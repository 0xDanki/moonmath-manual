\documentclass[a4paper,oneside,12pt]{book}

%margins
\usepackage[a4paper, margin=2.5cm]{geometry}
\usepackage[utf8]{inputenc}
\usepackage[english]{babel}
\usepackage[]{amsthm} %lets us use \begin{proof}
\usepackage[]{amssymb} %gives us the character \varnothing
\usepackage{setspace}
\usepackage{hyperref}
\doublespacing

\begin{document}

\subsection{CIRCOM}

In \cite{MunozTapia2022},
 Mu\~{n}oz {\textit{et al.}} presented a programming language
and a compiler for designing arithmetic circuits that are compiled to R1CS. The name of this new language is \verb|circom|. To be more precise, programmers can design their own arithmetic circuits using \verb|circom| and the compiler outputs not only the R1CS description, but also WebAssembly and \verb|C++| programs to efficiently
compute all values of the circuit. Moreover, there are multiple circuit templates in \verb|CIRCOMLIB| which is an open-source library of \verb|circom|. 

The \verb|circom| installation guide for Linux can be found on the website\footnote{\url{https://docs.circom.io/getting-started/installation/}}. In this book, we give some examples of how to writing arithmetic circuits using \verb|circom|. For proving the circuits, we use \verb|snarkjs| library. It is a JavaScript and Pure Web Assembly implementation of ZK-SNARK schemes. It uses the Groth16 Protocol (3 point only and 3 pairings) and PLONK. We choose the Groth16 protocol which is defined in Section\ref{sec:gorth_16} to prove our circuits. Also in \verb|circomlib|, Baby-Jubjub Curve\cite{jubjub} (BN128) is used for all arithmetic operations defined on the prime field $\mathbb{F}_p$ where, \[p = 21888242871839275222246405745257275088548364400416034343698204186575808495617\] We just give the definition of terms "signals", "templates" and "components" to understand our examples easily. The \verb|circom| language details can also be found on the website\footnote{https://docs.circom.io/circom-language/signals/}. 
\begin{itemize}
	\item "signals" are fields elements in $\mathbb{F}_p$. The arithmetic circuits built using \verb|circom| operate on signals. Signals are immutable and they can be defined as input or output. Input signals are private (if not stated otherwise using the keyword public) and all output signals are public. The rest of signals are all private and cannot be made public.
	\item "templates" is an algorithm to create generic circuits in \verb|circom|. The template is a new circuit object, which can be used to compose other circuits.
	\item "components" define an arithmetic circuit such that they receive input signals and produce output signals and  intermediate signals. Additionally, they can produce a set of constraints. Components are also immutable.
\end{itemize}

\subsubsection{2-Multiplication Example}
\noindent We give three circuits example. The first one is the 2-input multiplier circuit. The purpose of this circuit is to allow us to prove to someone that we know two numbers (\verb|a| and \verb|b|) that multiply them to give \verb|c|, without revealing \verb|a| and \verb|b|. This means that we are able to factor an integer \verb|c|.  

\noindent \textbf{Writing the circuit:} Firstly, we create a new file "\verb|mult2.circom|" and it contains the following template:

\begin{verbatim}
template Multiplier() {
    signal private input a;
    signal private input b;
    signal output c;

    c <== a*b;
}
component main = Multiplier();

\end{verbatim}

\noindent \textbf{Compiling the circuit:} In our circuit, we have two private input signals (\verb|a| and \verb|b|) and one output signal (\verb|c|). Now, we can compile the circuit using the command below:

\begin{verbatim}
circom mult2.circom --r1cs --wasm --sym
\end{verbatim}

\noindent This command generates three different files such that \verb|mult2.r1cs|, \verb|mult2.wasm| and \verb|mult2.sym|:
\begin{itemize}
	\item \verb|mult2.r1cs| contains the R1CS constraint system of the circuit in binary format.
	\item \verb|mult2.wasm| is a wasm code in the directory \verb|mult2_js| which contains files needed to generate the witness.
	\item \verb|mult2.sym| is a symbols file required for debugging or for printing the constraint system in an annotated mode.
\end{itemize}

\noindent From the command line run \verb|snarkjs info -r mult2.r1cs| gives us the details below:

\begin{verbatim}
[INFO]  snarkJS: Curve: bn-128
[INFO]  snarkJS: # of Wires: 4
[INFO]  snarkJS: # of Constraints: 1
[INFO]  snarkJS: # of Private Inputs: 2
[INFO]  snarkJS: # of Public Inputs: 0
[INFO]  snarkJS: # of Labels: 4
[INFO]  snarkJS: # of Outputs: 1
\end{verbatim}

\noindent \textbf{Computing the witness:} The set of inputs, intermediate signals and output is called witness. Now, we can compute the witness. We create a file named \verb|input.json| containing the inputs written in the standard json format. Using sagemath with command "\verb|random_prime(2^256)|", we generate two different 256 bit prime number and we add them in a file \verb|input.json| such that:

\begin{verbatim}
{"a":
 239998594368909167503748967592306645199, 
"b":
 17989387481342846867879040835511292829}
\end{verbatim}

\noindent To generate witness, enter in the directory \verb|mult2_js|, add the input in a file \verb|input.json|:

\begin{verbatim}
node generate_witness.js mult2.wasm input.json witness.wtns
\end{verbatim}

\noindent \verb|witness.wtns| file is encoded in a binary format compatible with snarkjs. Now, we use \verb|snarkjs| library to generate and validate a proof for our input. 

\noindent \textbf{Proving and verifying the circuit:} Firstly, we generate a trusted setup. As we said before, we use Groth16 zk-SNARK protocol. The trusted setup consists of 2 parts:

\begin{enumerate}
	\item The powers of $\tau$, which is independent of the circuit.
	\item The phase 2, which depends on the circuit.
\end{enumerate}

\noindent It can be started a new "powers of $\tau$" ceremony as follows:
\begin{verbatim}
snarkjs powersoftau new bn128 12 pot12_0000.ptau -v

[DEBUG] snarkJS: Calculating First Challenge Hash
[DEBUG] snarkJS: Calculate Initial Hash: tauG1
[DEBUG] snarkJS: Calculate Initial Hash: tauG2
[DEBUG] snarkJS: Calculate Initial Hash: alphaTauG1
[DEBUG] snarkJS: Calculate Initial Hash: betaTauG1
[DEBUG] snarkJS: Blank Contribution Hash:
786a02f7 42015903 c6c6fd85 2552d272
912f4740 e1584761 8a86e217 f71f5419
d25e1031 afee5853 13896444 934eb04b
903a685b 1448b755 d56f701a fe9be2ce
[INFO]  snarkJS: First Contribution Hash:
9e63a5f6 2b96538d aaed2372 481920d1
a40b9195 9ea38ef9 f5f6a303 3b886516
0710d067 c09d0961 5f928ea5 17bcdf49
ad75abd2 c8340b40 0e3b18e9 68b4ffef

snarkjs powersoftau contribute pot12_0000.ptau pot12_0001.ptau --name="1st_cont" -v
\end{verbatim}

\noindent Now, we are ready to start the phase 2. The following command start the generation of this phase:


\verb|snarkjs powersoftau prepare phase2 pot12_0001.ptau pot12_final.ptau -v|
\newpage
\noindent Next, we generate a \verb|.zkey| file that will contain the proving and verification keys together with all phase 2 contributions. Execute the following command to start a new zkey:

\verb|snarkjs groth16 setup mult2.r1cs pot12_final.ptau mult2_0000.zkey|

\begin{verbatim}
[INFO]  snarkJS: Reading r1cs
[INFO]  snarkJS: Reading tauG1
[INFO]  snarkJS: Reading tauG2
[INFO]  snarkJS: Reading alphatauG1
[INFO]  snarkJS: Reading betatauG1
[INFO]  snarkJS: Circuit hash: 
7f6efb81 1b8c6af5 4124edcc 6c50bf0f
5e2f3ad7 950579c8 79a62ae6 a807d1ed
453eb5ee 7edfd747 000fb9b7 b386fe92
b8aac3c3 b1261694 bd097a31 b2d65e3b
\end{verbatim}

\noindent As "MMM", we contribute to the phase 2 of the ceremony:

\verb|snarkjs zkey contribute mult2_0000.zkey mult2_0001.zkey --name="MMM" -v|

\noindent Export the verification key:

\verb|snarkjs zkey export verificationkey mult2_0001.zkey verification_key.json|

\noindent Once the witness is computed and the trusted setup is already executed, we can generate a zk-proof associated to the circuit and the witness:

\verb|snarkjs groth16 prove mult2_0001.zkey witness.wtns proof.json public.json|

\noindent We have two outputs, namely "\verb|proof.json|" and "\verb|public.json|".
\newpage
\begin{verbatim}
"proof.json"
{
"pi_a": [
"21406187904196223586918123752207045586496226461416441954146450709616200888871",
"17510392692218158235134452508316524756230979392010695079770294777136117847241",
"1"
],
"pi_b": [
[
"2090071021965257007758242978555767324630916830654503214195463722378011152994",
"2100135140908832594403395311379798745925373552245316771793435015163339124462"
],
[
"5384389015145349229307246080642002889088338815693005620453006291210288346164",
"4880896060730701248583728133667540283665869158434877502325001520456551721539"
],
[
"1",
"0"
]
],
"pi_c": [
"9652966723444936806918000865316219888229990889667853066455299460846564791978",
"21217226324776045682959902364886269573607147680689921164456952969619999791404",
"1"
],
"protocol": "groth16",
"curve": "bn128"
}
\end{verbatim}

\begin{verbatim}
"public.json"
[
"4317427709079934374760538971420816277665383577721473478792976159773288300544"
]
\end{verbatim}
\noindent Lastly, to verify the proof, execute the following command:

\verb|snarkjs groth16 verify verification_key.json public.json proof.json|

\verb|[INFO]  snarkJS: OK!|

\noindent It is also possible to generate a Solidity verifier that allows verifying proofs on Ethereum blockchain as follows:

\verb|snarkjs zkey export solidityverifier mult2_0001.zkey verifier.sol|

\verb|snarkjs generatecall|

\begin{verbatim}
["0x2a221f1a579d40f7272cc84a807ad7e09ba2930cb07fd8234ec94e3da75d0582",
 "0x0dae93b52d6e416dc5c253c27c22da2095e2b8b48e9292a64a68b847219b962c"],
 [["0x3006d45355660ace3d894a61ebdc31bfc9510b9598416da27f89f96cc67045db",
 "0x15dacb632925d749f642ee485d3fa6a5a45b947af15642c53ab654e0690257bb"],
 ["0x200b7084fcbbe82a0b5d5f3920686dfc9d68c5e7311b62822e300deb83fcd9ba",
  "0x0375b55b7a404b1abaeaa4b395e62ba9689959fa5122891ae770e75517001a24"]],
 ["0x2a0a77153ffd5ad62aca8e70b38211a8daa35dd4cec535fe97cd96eb37ff0659",
  "0x0aabfe601a7f8b9f1d741ffd3b3b4f08f1e4ede64bc27e8196e0eea09362429f"],
 ["0x10043fd720899394f8c6503b52d1cf31257756bd1cc0c678bcc5ed4c09f510f7"]
\end{verbatim}

\subsubsection{3-Factorization Example}
\noindent Now we write a circuit for the 3-factorization. (Example 111 in the MMM). The purpose of this circuit is that given \verb|y| provide three numbers \verb|x1|, \verb|x2|, \verb|x3| such that $y = x_1x_2x_3$. This means that we are able to factor \verb|y|.  

\noindent \textbf{Writing the circuit:} Firstly, we create a new file "\verb|mult3.circom|" and it contains also the previous code "\verb|mult2.circom|":

\begin{verbatim}
pragma circom 2.0.0;
template Multiplier2() {
signal input a;
signal input b;
signal output c;
c <== a*b;}

template Multiplier3 () {

signal input x1;
signal input x2;
signal input x3;
signal output y;
component mult1 = Multiplier2();
component mult2 = Multiplier2();

mult1.a <== x1;
mult1.b <== x2;
mult2.a <== mult1.c;
mult2.b <== x3;
y <== mult2.c;
}
component main = Multiplier3();
\end{verbatim}

\noindent \textbf{Compiling the circuit:} In our circuit, we have three private input signals (\verb|x1|, \verb|x2| and \verb|x3|) and one output signal (\verb|y|). Now, we can compile the circuit using the command below:

\begin{verbatim}
circom mult3.circom --r1cs --wasm --sym --c
\end{verbatim}

\noindent This command generates three different files such that \verb|mult3.r1cs|, \verb|mult3.wasm| and \verb|mult3.sym|.

\noindent From the command line run \verb|snarkjs info -r mult3.r1cs| gives us the details below:

\begin{verbatim}
[INFO]  snarkJS: Curve: bn-128
[INFO]  snarkJS: # of Wires: 6
[INFO]  snarkJS: # of Constraints: 2
[INFO]  snarkJS: # of Private Inputs: 3
[INFO]  snarkJS: # of Public Inputs: 0
[INFO]  snarkJS: # of Labels: 11
[INFO]  snarkJS: # of Outputs: 1
\end{verbatim}

\noindent \textbf{Computing the witness:} We create a file named \verb|input.json| containing the inputs written in the standard json format. Using sagemath with command "\verb|random_prime(2^128)|", we generate three different 128 bit prime number and we add them in a file \verb|input.json| such that:

\begin{verbatim}
{"x1": 59658793945365707096775385383943529221, 
"x2": 9197597494235733864761252978256366097,
"x3":21695142336767020381719608537329614499 }
\end{verbatim}

\noindent To generate witness, enter in the directory \verb|mult3_js|, add the input in a file \verb|input.json|:

\begin{verbatim}
node generate_witness.js mult3.wasm input.json witness.wtns
\end{verbatim}

\noindent \verb|witness.wtns| file is encoded in a binary format compatible with snarkjs. Now, we use \verb|snarkjs| library to generate and validate a proof for our input. 

\noindent \textbf{Proving and verifying the circuit:} We can start a new "powers of $\tau$" ceremony as follows:
\begin{verbatim}
snarkjs powersoftau new bn128 12 pot12_0000.ptau -v
snarkjs powersoftau contribute pot12_0000.ptau pot12_0001.ptau --name="1st_cont" -v
\end{verbatim}

\noindent Now, we are ready to start the phase 2. The following command start the generation of this phase:


\verb|snarkjs powersoftau prepare phase2 pot12_0001.ptau pot12_final.ptau -v|
\newpage
\noindent Next, we generate a \verb|.zkey| file that will contain the proving and verification keys together with all phase 2 contributions. Execute the following command to start a new zkey:

\verb|snarkjs groth16 setup mult3.r1cs pot12_final.ptau mult3_0000.zkey|

\noindent As "MMM", we contribute to the phase 2 of the ceremony:

\verb|snarkjs zkey contribute mult3_0000.zkey mult3_0001.zkey --name="MMM" -v|

\noindent Export the verification key:

\verb|snarkjs zkey export verificationkey mult3_0001.zkey verification_key.json|

\noindent Once the witness is computed and the trusted setup is already executed, we can generate a zk-proof associated to the circuit and the witness:

\verb|snarkjs groth16 prove mult3_0001.zkey witness.wtns proof.json public.json|

\noindent We have two outputs, namely "\verb|proof.json|" and "\verb|public.json|".

\begin{verbatim}
{
"pi_a": [
"16745812350712194599912818885111850581140129102802781103978195448222742721982",
"2930220674110530044572393269845479082581725705447255430373346826416347319433",
"1"
],
"pi_b": [
[
"5884958842781712287459166407321120679731237160489467479981941478408621125291",
"908154676307550301490130304968842226724287626448508510713338586945928079132"
],
[
"6031864672109082326061204029412974077200625770550844508413016404580708499805",
"18610897479695624478152282790429593926513626014501302970467057976691179455214"
],
[
"1",
"0"
]],
"pi_c": [
"4008230898800091114550678437437282196492792990719145660073689042559906893420",
"21357079849090056859534635787041592226330777406166136445206627531091878143647",
"1"
],
"protocol": "groth16",
"curve": "bn128"
}
\end{verbatim}

\begin{verbatim}
"public.json"
[
"11287789420151883716610961972871231842187219266781340012158803956890733917065"
]
\end{verbatim}
\noindent Lastly, to verify the proof, execute the following command:

\verb|snarkjs groth16 verify verification_key.json public.json proof.json|

\verb|[INFO]  snarkJS: OK!|

\noindent Generate a Solidity verifier that allows verifying proofs on Ethereum blockchain as follows:

\verb|snarkjs zkey export solidityverifier mult3_0001.zkey verifier.sol|

\verb|snarkjs generatecall|

\begin{verbatim}
["0x2505cb3db1f802a045dde60a93ae0dbaa2336f3347d38c3f794b94740bf7b5be",
 "0x067a7235790ae03570fe366d1f5ff47ffb11f15cc3574dc3e02ac3b14d8e8c89"],
 [["0x0201ff511c81c488d480c50977962d76e1cd3e20c339175df5b2f8ff41e4a71c"
 , "0x0d02c4c75ca456064866cf441400ddf03f57a2b78d838b5de9eff5f9b823eeab"],
 ["0x2925650a03d710d8b28d9f36705d57974c01a28fc9eb089f3a1bf1cfaead02ee",
  "0x0d55ea176b445a896219e768a6ef899a6cc156cf42cea9ce4f3f1da3a3a5c15d"]],
  ["0x08dc940b0834364701772e78275164a568bf0cf9d8655eb28aa53d26b3644e6c",
   "0x2f37adc75ea2d70f03f32268c8b5bd62a7d3084a4351e4708ca175f02f80e69f"],
   ["0x18f4a99372ec45072aadea583c4bf322f81afb3ded204af3651f6cbe32243389"]
\end{verbatim}

\subsubsection{Baby JubJub Example}
\noindent Now we write a circuit for Baby JubJub curve. The purpose of this circuit is proving a pair $(x, y)$ of field elements from $\mathbb{F}_p$ to be a point on the Baby JubJub curve in its Twisted Edwards form.
Firstly, we give a brief information about Baby JubJub curve. To verify a Groth16 proof, it is necessary to use an elliptic curve. The curve is bn128 (also referred as BN254), which has prime order $p$. With this curve, it is possible to generate and validate proofs of any $\mathbb{F}_p$-arithmetic circuit. bn128 has a Twisted Edwards form such that:
\[ax^2 + y^2 = 1 + dx^2y^2\] where $a = 168700, d = 168696$.

\noindent \textbf{Writing the circuit:} Firstly, we create a new file "\verb|babyjub.circom|" such that: 

\begin{verbatim}
pragma circom 2.0.0;
template BabyCheck() {
signal input x;
signal input y;

signal x2;
signal y2;
signal dx2y2;

var a = 168700;
var d = 168696;

x2 <== x*x;
y2 <== y*y;
dx2y2 <== d*x2*y2;

(a*x2) + (y2 - 1) - dx2y2 === 0;}

component main {public[x,y]}= BabyCheck();
\end{verbatim}

\noindent \textbf{Compiling the circuit:} In our circuit, we have two public input signals (\verb|x|, \verb|y|), three private constraint signals (\verb|x2|, \verb|y2|, \verb|dx2y2|). Now, we can compile the circuit using the command below:

\begin{verbatim}
circom babyjb.circom --r1cs --wasm --sym --c
\end{verbatim}

\noindent This command generates three different files such that \verb|babyjub.r1cs|, \verb|babyjub.wasm| and \verb|babyjub.sym|.

\noindent From the command line run \verb|snarkjs info -r babyjub.r1cs| gives us the details below:

\begin{verbatim}
[INFO]  snarkJS: Curve: bn-128
[INFO]  snarkJS: # of Wires: 5
[INFO]  snarkJS: # of Constraints: 3
[INFO]  snarkJS: # of Private Inputs: 0
[INFO]  snarkJS: # of Public Inputs: 2
[INFO]  snarkJS: # of Labels: 6
[INFO]  snarkJS: # of Outputs: 0
\end{verbatim}

\noindent \textbf{Computing the witness:} We create a file named \verb|input.json| containing the inputs written in the standard json format. We choose the point $(x,y)=(0,1)$ as an input:

\begin{verbatim}
{"x": 0 , "y": 1}
\end{verbatim}

\noindent To generate witness, enter in the directory \verb|babyjub_js|, add the input in a file \verb|input.json|:

\begin{verbatim}
node generate_witness.js babyjub.wasm input.json witness.wtns
\end{verbatim}

\noindent \verb|witness.wtns| file is encoded in a binary format compatible with snarkjs. Now, we use \verb|snarkjs| library to generate and validate a proof for our input. 

\noindent \textbf{Proving and verifying the circuit:} We can start a new "powers of $\tau$" ceremony as follows:
\begin{verbatim}
snarkjs powersoftau new bn128 12 pot12_0000.ptau -v
snarkjs powersoftau contribute pot12_0000.ptau pot12_0001.ptau --name="1st_cont" -v
\end{verbatim}

\noindent Now, we are ready to start the phase 2. The following command start the generation of this phase:


\verb|snarkjs powersoftau prepare phase2 pot12_0001.ptau pot12_final.ptau -v|

\noindent Next, we generate a \verb|.zkey| file that will contain the proving and verification keys together with all phase 2 contributions. Execute the following command to start a new zkey:

\verb|snarkjs groth16 setup babyjub.r1cs pot12_final.ptau babyjub_0000.zkey|

\noindent As "MMM", we contribute to the phase 2 of the ceremony:

\verb|snarkjs zkey contribute babyjub_0000.zkey babyjub_0001.zkey --name="MMM" -v|

\noindent Export the verification key:

\verb|snarkjs zkey export verificationkey babyjub_0001.zkey verification_key.json|

\noindent Once the witness is computed and the trusted setup is already executed, we can generate a zk-proof associated to the circuit and the witness:

\verb|snarkjs groth16 prove babyjub_0001.zkey witness.wtns proof.json public.json|

\noindent We have two outputs, namely "\verb|proof.json|" and "\verb|public.json|".

\begin{verbatim}
"proof.json"
{
"pi_a": [
"20518903489287660620661153311390007323011620749185510399461943193146649102472",
"15307223541957067225744138260761203687009837259992128682557497517222743661548",
"1"
],
"pi_b": [
[
"14274617841320081523853305267827272163893668222989444791093646331034150484531",
"7358916457896414291520611398710118572179838678670116677571443068983066759462"
],
[
"14696822342728687033692631337429168156499204402406531544650781570954192796088",
"6555775806980740688732910354015556043588740413196388967589336518129553848787"
],
[
"1",
"0"
]
],
"pi_c": [
"21056013167231969296572471902819508629642169841186711598183732562036802928843",
"879681114039454498023611074575505017994830830641599083218485286916318893313",
"1"
],
"protocol": "groth16",
"curve": "bn128"
}
\end{verbatim}

\begin{verbatim}
"public.json"
[
"0",
"1"
]
\end{verbatim}
\noindent Lastly, to verify the proof, execute the following command:

\verb|snarkjs groth16 verify verification_key.json public.json proof.json|

\verb|[INFO]  snarkJS: OK!|

\noindent Generate a Solidity verifier that allows verifying proofs on Ethereum blockchain as follows:

\verb|snarkjs zkey export solidityverifier babyjub_0001.zkey verifier.sol|

\verb|snarkjs generatecall|

\begin{verbatim}
"0x2d5d49ae473cf4259e389b987552151e794ec3ccd8a14adc25d333457d0e4888",
"0x21d794dd1e2480d7f94e3344b2a48729542eaf1b06a55cc8b355e2e1be1a5bec"]
,[["0x1044ffc5e15796548105ea38caa624d52c9be6c54c5f49fd954316463fdc1126",
 "0x1f8f25baf51619cb6e5dc04e92c1052fb5091095b5abdc78b6fd4d48658dba33"],
 ["0x0e7e70077532e073bbc2cfadb3044aaae1ac855276af62bedbaa8213a7fd39d3",
  "0x207e1b4e4ac6c7988a7cb6384cefad0303c417b5ee6c83a8b4def4c1ce54b9b8"]],
  ["0x2e8d47f77d0fbf2f9cab8e00d79e1644b30dc612029d728ec12edeee34cf40cb",
  "0x01f1e1c1fd06ef763a0e4ee13b423771d882e0893b1a3a75baa2a110ed5d1501"],
  ["0x0000000000000000000000000000000000000000000000000000000000000000",
  "0x0000000000000000000000000000000000000000000000000000000000000001"]
\end{verbatim}



\begin{thebibliography}{99}
\bibitem{MunozTapia2022} Muñoz-Tapia, Jose L. Belles, Marta Isabel, Miguel Rubio, Albert Baylina, Jordi (2022). CIRCOM: \textit{A Robust and Scalable Language for Building Complex Zero-Knowledge Circuits}. TechRxiv. Preprint. https://doi.org/10.36227/techrxiv.19374986.v1. 	

\bibitem{jubjub} Barry WhiteHat, Jordi Baylina, and Marta Belles. \textit{Baby JubJub Elliptic Curve}, 2019. \url{https://iden3-docs.readthedocs.io/en/latest/_downloads/
33717d75ab84e11313cc0d8a090b636f/Baby-Jubjub.pdf}.
\end{thebibliography}

%@article{Muñoz-Tapia2022, author = "Jose L. Muñoz-Tapia and Marta Belles and Miguel Isabel and Albert Rubio and Jordi Baylina", title = "{CIRCOM: A Robust and Scalable Language for Building Complex Zero-Knowledge Circuits}", year = "2022", month = "3", url = "https://www.techrxiv.org/articles/preprint/CIRCOM_A_Robust_and_Scalable_Language_for_Building_Complex_Zero-Knowledge_Circuits/19374986", doi = "10.36227/techrxiv.19374986.v1" } 

\end{document}
