\begin{center}\Large
Creative Commons Attribution-NonCommercial-NoDerivatives 4.0 International Public License
\end{center}


By exercising the Licensed Rights (defined below), You accept and agree to be bound by the terms and conditions of this Creative Commons Attribution-NonCommercial-NoDerivatives 4.0 International Public License (``Public License''). To the extent this Public License may be interpreted as a contract, You are granted the Licensed Rights in consideration of Your acceptance of these terms and conditions, and the Licensor grants You such rights in consideration of benefits the Licensor receives from making the Licensed Material available under these terms and conditions.

\subsection*{Section 1 – Definitions.}
\begin{itemize}
\item[a.] \textbf{Adapted Material} means material subject to Copyright and Similar Rights that is derived from or based upon the Licensed Material and in which the Licensed Material is translated, altered, arranged, transformed, or otherwise modified in a manner requiring permission under the Copyright and Similar Rights held by the Licensor. For purposes of this Public License, where the Licensed Material is a musical work, performance, or sound recording, Adapted Material is always produced where the Licensed Material is synched in timed relation with a moving image.

\item[b.] \textbf{Copyright and Similar Rights} means copyright and/or similar rights closely related to copyright including, without limitation, performance, broadcast, sound recording, and Sui Generis Database Rights, without regard to how the rights are labeled or categorized. For purposes of this Public License, the rights specified in Section \href{https://creativecommons.org/licenses/by-nc-nd/4.0/legalcode#s2b}{2(b)(1)-(2)} are not Copyright and Similar Rights.

\item[c.] \textbf{Effective Technological Measures} means those measures that, in the absence of proper authority, may not be circumvented under laws fulfilling obligations under Article 11 of the WIPO Copyright Treaty adopted on December 20, 1996, and/or similar international agreements.

\item[d.] \textbf{Exceptions and Limitations} means fair use, fair dealing, and/or any other exception or limitation to Copyright and Similar Rights that applies to Your use of the Licensed Material.

\item[e.] \textbf{Licensed Material} means the artistic or literary work, database, or other material to which the Licensor applied this Public License.

\item[f.] \textbf{Licensed Rights} means the rights granted to You subject to the terms and conditions of this Public License, which are limited to all Copyright and Similar Rights that apply to Your use of the Licensed Material and that the Licensor has authority to license.

\item[g.] \textbf{Licensor} means the individual(s) or entity(ies) granting rights under this Public License.

\item[h.] \textbf{NonCommercial} means not primarily intended for or directed towards commercial advantage or monetary compensation. For purposes of this Public License, the exchange of the Licensed Material for other material subject to Copyright and Similar Rights by digital file-sharing or similar means is NonCommercial provided there is no payment of monetary compensation in connection with the exchange.

\item[i.] \textbf{Share} means to provide material to the public by any means or process that requires permission under the Licensed Rights, such as reproduction, public display, public performance, distribution, dissemination, communication, or importation, and to make material available to the public including in ways that members of the public may access the material from a place and at a time individually chosen by them.

\item[j.] \textbf{Sui Generis Database Rights} means rights other than copyright resulting from Directive 96/9/EC of the European Parliament and of the Council of 11 March 1996 on the legal protection of databases, as amended and/or succeeded, as well as other essentially equivalent rights anywhere in the world.

\item[k.] \textbf{You} means the individual or entity exercising the Licensed Rights under this Public License. Your has a corresponding meaning.
\end{itemize}

\subsection*{Section 2 – Scope.}
\begin{itemize}
\item[a.] \textbf{License grant.}
	\begin{enumerate}
	\item Subject to the terms and conditions of this Public License, the Licensor hereby grants You a worldwide, royalty-free, non-sublicensable, non-exclusive, irrevocable license to exercise the Licensed Rights in the Licensed Material to:
	\begin{enumerate}
		\item[A.] reproduce and Share the Licensed Material, in whole or in part, for NonCommercial purposes only; and
		\item[B.] produce and reproduce, but not Share, Adapted Material for NonCommercial purposes only.
	\end{enumerate}
	\item \underline{Exceptions and Limitations.} For the avoidance of doubt, where Exceptions and Limitations apply to Your use, this Public License does not apply, and You do not need to comply with its terms and conditions.
	\item \underline{Term.} The term of this Public License is specified in Section \href{https://creativecommons.org/licenses/by-nc-nd/4.0/legalcode#s6a}{6(a)}.
	\item \underline{Media and formats; technical modifications allowed.} The Licensor authorizes You to exercise the Licensed Rights in all media and formats whether now known or hereafter created, and to make technical modifications necessary to do so. The Licensor waives and/or agrees not to assert any right or authority to forbid You from making technical modifications necessary to exercise the Licensed Rights, including technical modifications necessary to circumvent Effective Technological Measures. For purposes of this Public License, simply making modifications authorized by this Section \href{https://creativecommons.org/licenses/by-nc-nd/4.0/legalcode#s2a4}{2(a)(4)} never produces Adapted Material.
	\item \underline{Downstream recipients.}
		\begin{enumerate}
		\item[A.] \underline{Offer from the Licensor – Licensed Material.} Every recipient of the Licensed Material automatically receives an offer from the Licensor to exercise the Licensed Rights under the terms and conditions of this Public License.
		\item[B.] \underline{No downstream restrictions.} You may not offer or impose any additional or different terms or conditions on, or apply any Effective Technological Measures to, the Licensed Material if doing so restricts exercise of the Licensed Rights by any recipient of the Licensed Material.
		\end{enumerate}
\item \underline{No endorsement.} Nothing in this Public License constitutes or may be construed as permission to assert or imply that You are, or that Your use of the Licensed Material is, connected with, or sponsored, endorsed, or granted official status by, the Licensor or others designated to receive attribution as provided in Section \href{https://creativecommons.org/licenses/by-nc-nd/4.0/legalcode#s3a1Ai}{3(a)(1)(A)(i)}.
	\end{enumerate}
\item[b.] \textbf{Other rights.}
	\begin{enumerate}
	\item Moral rights, such as the right of integrity, are not licensed under this Public License, nor are publicity, privacy, and/or other similar personality rights; however, to the extent possible, the Licensor waives and/or agrees not to assert any such rights held by the Licensor to the limited extent necessary to allow You to exercise the Licensed Rights, but not otherwise.
	\item Patent and trademark rights are not licensed under this Public License.
	\item To the extent possible, the Licensor waives any right to collect royalties from You for the exercise of the Licensed Rights, whether directly or through a collecting society under any voluntary or waivable statutory or compulsory licensing scheme. In all other cases the Licensor expressly reserves any right to collect such royalties, including when the Licensed Material is used other than for NonCommercial purposes.
	\end{enumerate}
\end{itemize}

\subsection*{Section 3 – License Conditions.}
Your exercise of the Licensed Rights is expressly made subject to the following conditions.
\begin{itemize}
\item[a.] \textbf{Attribution.}
	\begin{enumerate}
	\item If You Share the Licensed Material, You must:
		\begin{enumerate}
		\item[A.] retain the following if it is supplied by the Licensor with the Licensed Material:
			\begin{enumerate}
			\item identification of the creator(s) of the Licensed Material and any others designated to receive attribution, in any reasonable manner requested by the Licensor (including by pseudonym if designated);
			\item a copyright notice;
			\item a notice that refers to this Public License;
			\item a notice that refers to the disclaimer of warranties;
			\item a URI or hyperlink to the Licensed Material to the extent reasonably practicable;
			\end{enumerate}
		\item[B.] indicate if You modified the Licensed Material and retain an indication of any previous modifications; and
		\item[C.] indicate the Licensed Material is licensed under this Public License, and include the text of, or the URI or hyperlink to, this Public License.
		\end{enumerate}
For the avoidance of doubt, You do not have permission under this Public License to Share Adapted Material.
\item You may satisfy the conditions in Section \href{https://creativecommons.org/licenses/by-nc-nd/4.0/legalcode#s3a1A}{3(a)(1)} in any reasonable manner based on the medium, means, and context in which You Share the Licensed Material. For example, it may be reasonable to satisfy the conditions by providing a URI or hyperlink to a resource that includes the required information.
\item If requested by the Licensor, You must remove any of the information required by Section \href{https://creativecommons.org/licenses/by-nc-nd/4.0/legalcode#s3a1A}{3(a)(1)(A)} to the extent reasonably practicable.
	\end{enumerate}
\end{itemize}

\subsection*{Section 4 – Sui Generis Database Rights.}
Where the Licensed Rights include Sui Generis Database Rights that apply to Your use of the Licensed Material:
\begin{itemize}
\item[a.]
for the avoidance of doubt, Section \href{https://creativecommons.org/licenses/by-nc-nd/4.0/legalcode#s2a1}{2(a)(1)} grants You the right to extract, reuse, reproduce, and Share all or a substantial portion of the contents of the database for NonCommercial purposes only and provided You do not Share Adapted Material;
\item[b.] if You include all or a substantial portion of the database contents in a database in which You have Sui Generis Database Rights, then the database in which You have Sui Generis Database Rights (but not its individual contents) is Adapted Material; and
\item[c.] You must comply with the conditions in Section \href{https://creativecommons.org/licenses/by-nc-nd/4.0/legalcode#s3a}{3(a)} if You Share all or a substantial portion of the contents of the database.
\end{itemize}
For the avoidance of doubt, this Section \href{https://creativecommons.org/licenses/by-nc-nd/4.0/legalcode#s4}{4} supplements and does not replace Your obligations under this Public License where the Licensed Rights include other Copyright and Similar Rights.

\subsection*{Section 5 – Disclaimer of Warranties and Limitation of Liability.}
\begin{itemize}
\item[\textbf{a.}] \textbf{Unless otherwise separately undertaken by the Licensor, to the extent possible, the Licensor offers the Licensed Material as-is and as-available, and makes no representations or warranties of any kind concerning the Licensed Material, whether express, implied, statutory, or other. This includes, without limitation, warranties of title, merchantability, fitness for a particular purpose, non-infringement, absence of latent or other defects, accuracy, or the presence or absence of errors, whether or not known or discoverable. Where disclaimers of warranties are not allowed in full or in part, this disclaimer may not apply to You.}
\item[\textbf{b.}] \textbf{To the extent possible, in no event will the Licensor be liable to You on any legal theory (including, without limitation, negligence) or otherwise for any direct, special, indirect, incidental, consequential, punitive, exemplary, or other losses, costs, expenses, or damages arising out of this Public License or use of the Licensed Material, even if the Licensor has been advised of the possibility of such losses, costs, expenses, or damages. Where a limitation of liability is not allowed in full or in part, this limitation may not apply to You.}
\item[c.] The disclaimer of warranties and limitation of liability provided above shall be interpreted in a manner that, to the extent possible, most closely approximates an absolute disclaimer and waiver of all liability.
\end{itemize}

\subsection*{Section 6 – Term and Termination.}
\begin{itemize}
\item[a.] This Public License applies for the term of the Copyright and Similar Rights licensed here. However, if You fail to comply with this Public License, then Your rights under this Public License terminate automatically.
\item[b.] Where Your right to use the Licensed Material has terminated under Section \href{https://creativecommons.org/licenses/by-nc-nd/4.0/legalcode#s6a}{6(a)}, it reinstates:
	\begin{enumerate}
	\item automatically as of the date the violation is cured, provided it is cured within 30 days of Your discovery of the violation; or
	\item upon express reinstatement by the Licensor.
	\end{enumerate}
For the avoidance of doubt, this Section \href{https://creativecommons.org/licenses/by-nc-nd/4.0/legalcode#s6b}{6(b)} does not affect any right the Licensor may have to seek remedies for Your violations of this Public License.
\item[c.] For the avoidance of doubt, the Licensor may also offer the Licensed Material under separate terms or conditions or stop distributing the Licensed Material at any time; however, doing so will not terminate this Public License.
\item[d.] Sections \href{https://creativecommons.org/licenses/by-nc-nd/4.0/legalcode#s1}{1}, \href{https://creativecommons.org/licenses/by-nc-nd/4.0/legalcode#s5}{5}, \href{https://creativecommons.org/licenses/by-nc-nd/4.0/legalcode#s6}{6}, \href{https://creativecommons.org/licenses/by-nc-nd/4.0/legalcode#s7}{7}, and \href{https://creativecommons.org/licenses/by-nc-nd/4.0/legalcode#s8}{8} survive termination of this Public License.
\end{itemize}

\subsection*{Section 7 – Other Terms and Conditions.}
\begin{itemize}
\item[a.] The Licensor shall not be bound by any additional or different terms or conditions communicated by You unless expressly agreed.
\item[b.] Any arrangements, understandings, or agreements regarding the Licensed Material not stated herein are separate from and independent of the terms and conditions of this Public License.
\end{itemize}

\subsection*{Section 8 – Interpretation.}
\begin{itemize}
\item[a.] For the avoidance of doubt, this Public License does not, and shall not be interpreted to, reduce, limit, restrict, or impose conditions on any use of the Licensed Material that could lawfully be made without permission under this Public License.
\item[b.] To the extent possible, if any provision of this Public License is deemed unenforceable, it shall be automatically reformed to the minimum extent necessary to make it enforceable. If the provision cannot be reformed, it shall be severed from this Public License without affecting the enforceability of the remaining terms and conditions.
\item[c.] No term or condition of this Public License will be waived and no failure to comply consented to unless expressly agreed to by the Licensor.
\item[d.] Nothing in this Public License constitutes or may be interpreted as a limitation upon, or waiver of, any privileges and immunities that apply to the Licensor or You, including from the legal processes of any jurisdiction or authority.
\end{itemize}

Creative Commons is not a party to its public licenses. Notwithstanding, Creative Commons may elect to apply one of its public licenses to material it publishes and in those instances will be considered the “Licensor.” The text of the Creative Commons public licenses is dedicated to the public domain under the \href{https://creativecommons.org/publicdomain/zero/1.0/legalcode}{\textsl{CC0 Public Domain Dedication}}. Except for the limited purpose of indicating that material is shared under a Creative Commons public license or as otherwise permitted by the Creative Commons policies published at \href{https://creativecommons.org/policies}{creativecommons.org/policies}, Creative Commons does not authorize the use of the trademark “Creative Commons” or any other trademark or logo of Creative Commons without its prior written consent including, without limitation, in connection with any unauthorized modifications to any of its public licenses or any other arrangements, understandings, or agreements concerning use of licensed material. For the avoidance of doubt, this paragraph does not form part of the public licenses.

Creative Commons may be contacted at \href{creativecommons.org}{creativecommons.org}.

\begin{center}\Large 
Sideletter for Contributions to the MoonMath Manual To zk-SNARKs (the ``\textbf{Sideletter}'')
\end{center}

between Least Authority TFA GmbH, Thaerstraße 28a, 10249 Berlin (hereinafter referred to as ``\textbf{Least Authority}'') and any natural person or legal entity submitting Contributions to the MoonMath Manual (hereinafter referred to as ``\textbf{You}'' or ``\textbf{Your}''). 
\begin{center}\bfseries
Preamble
\end{center}
\begin{itemize}
\item[(A)] Least Authority is the initial creator of the so-called MoonMath Manual To zk-SNARKs (the ``\textbf{Manual}'') which serves as a resource for anyone interested in understanding and unlocking the potential of the so-called ``zk-SNARK'' technology (``\textbf{zk-SNARK}''). The acronym zk-SNARK stands for “Zero-Knowledge Succinct Non-Interactive Argument of Knowledge” and refers to a cryptographic technique where one can prove possession of certain information without revealing the information itself. Most explanations struggle to clarify how and why they work. Resources are scattered across blog posts and Github libraries. This results in a high barrier to entry, thereby slowing the widespread adoption of zk-SNARKs and associated privacy-enhancing technologies. 
\item[(B)] Least Authority wants to change that with the Manual by continuing the Manual as a community-based project to collect useful and practical information on the zk-SNARK. Third-party authors like You shall be able to contribute parts, ideas and practical information to the Manual. 
\item[(C)] The Manual itself is licensed under the Creative Commons Public License, version \textsl{Attri\-bution-NonCommercial-NoDerivatives 4.0 International} (``\textbf{CC BY-NC-ND-4.0}''), which allows usage and distribution as well as modification of the Manual. However, if You modify the Manual or create ``\textbf{Adapted Material}'' of the Manual in the sense of Section 1.a. of the CC BY-NC-ND-4.0, those are not allowed to be distributed by You because Section 3.a.1. subsection 2 of the CC BY-NC-ND-4.0 prohibits the distribution of Adapted Material. 
\item[(D)] If You wish to participate in the Manual, You can submit Adapted Material on the Manual as well as material created independently from the Manual (``\textbf{Independent Creations}'') to Least Authority. If You are interested in adding a major Contribution to the Manual, please contact Least Authority directly under \href{mailto:mmm@leastauthority.com}{mmm@leastauthority.com} and we can discuss if Your contribution can be handled individually with different terms.
\item[(E)] Subject of this Sideletter shall be the licensing of Your Contribution to Least Authority. 
\end{itemize}
Now it is agreed as follows:

\begin{center}
$\S$1\\\bfseries
License on Your Submitted Contribution
\end{center}

\begin{itemize}
\item[(1)] You can contribute any written work, graphic, calculation method, compilation of information, database, or any other work of authorship, including any modifications or additions to the Manual that is created by You by submitting it to Least Authority for the purpose of the inclusion in the Manual, regardless of whether it is an Independent Creation or Adapted Material (each of them a ``\textbf{Contribution}''). ``\textbf{Submission}'' in this sense includes any form of electronic, verbal, or written communication sent to Least Authority under \href{mailto:mmm@leastauthority.com}{mmm@leastauthority.com} or uploaded to https://github.com/LeastAuthority/moonmath-manual. For clarity: Least Authority is not obligated to include Your Contribution in the Manual.

\item[(2)] You hereby grant Least Authority a perpetual, worldwide, non-exclusive, sublicensable, irrevocable and royalty-free right to use, modify, edit, make publicly available and distribute Your Contribution in tangible and intangible form or any other way now known or in the future developed in their original or modified way (within the limits of the prohibition of defacement), as well as to combine it in the original or modified way with or into the Manual (``\textbf{License}''). The License does at least include all rights required to license the Contribution under the CC BY-NC-ND-4.0 and in particular allows Least Authority to use, modify, edit, make publicly available and distribute in tangible and intangible form or any other way now known or in the future developed the Contribution as part of the Manual. Least Authority hereby accepts the grant of the License. 
\item[(3)] If Least Authority decides that Your Contribution or parts thereof shall be included in the Manual, Least Authority will ensure the following: 
	\begin{itemize}
	\item[a)] the Contribution as part of the Manual is licensed under the CC BY-NC-ND-4.0, 
	\item[b)]You will be identified as a Contributor (including by pseudonym if designated) in the Manual. 
	\end{itemize}
The rule § 1 (3) b) only applies if Your name or pseudonym is supplied with the Contribution. 
\item[(4)] In case Least Authority decides that only parts or revisions of Your Contribution will be included in the Manual, Least Authority will inform You within a reasonable period of time and obtain Your consent to license the parts / revisions of Your Contribution corresponding to §1 (2). No consent is needed if only editorial changes are made by Least Authority. In case You decide to submit Your Contribution with no information to contact You, this clause § 1 (4) shall not apply since Least Authority has no possibility to obtain Your consent.      
\item[(5)] In case Least Authority decides that Your Contribution will not be part of the Manual, Least Authority shall use reasonable means to inform you on its decision within a reasonable period of time after Your Submission. The License You granted to Least Authority ends with the decision by Least Authority not to include the Contributions into the Manual. 
\end{itemize}

\begin{center}
$\S$2\\\bfseries
Disclaimer
\end{center}
\begin{itemize}
\item[(1)] Unless otherwise separately undertaken by You, to the extent possible, You offer the Contribution as-is and as-available, and make no representations or warranties of any kind concerning the Contribution, whether express, implied, statutory, or other. This includes, without limitation, warranties of title, merchantability, fitness for a particular purpose, non-infringement, absence of latent or other defects, accuracy, or the presence or absence of errors, whether or not known or discoverable. Where disclaimers of warranties are not allowed in full or in part, this disclaimer may not apply to You.
\item[(2)] To the extent possible, in no event will You be liable to us on any legal theory (including, without limitation, negligence) or otherwise for any direct, special, indirect, incidental, consequential, punitive, exemplary, or other losses, costs, expenses, or damages arising out of this Side Letter or use of the Contribution, even if You have been advised of the possibility of such losses, costs, expenses, or damages. Where a limitation of liability is not allowed in full or in part, this limitation may not apply to You.
\item[(3)] The disclaimer of warranties and limitation of liability provided above shall be interpreted in a manner that, to the extent possible, most closely approximates an absolute disclaimer and waiver of all liability.
\end{itemize}

\begin{center}
$\S$3\\\bfseries
Miscellaneous
\end{center}

\begin{itemize}
\item[(1)] This Sideletter is valid without signature. It is concluded between You and Least Authority at the time of the submission of the Contribution by You to Least Authority. 
\item[(2)] This Sideletter and its interpretation and any non-contractual obligations in connection with it are subject to German substantive law. The UN Convention on Contracts for the International Sale of Goods (CISG) shall not apply.
\item[(3)] English language terms used in this Sideletter describe German legal concepts only and shall not be interpreted by reference to any meaning attributed to them in any jurisdiction other than Germany. Where a German term has been inserted in brackets and/or italics it alone (and not the English term to which it relates) shall be authoritative for the purpose of the interpretation of the relevant term whenever it is used in this Agreement.
\item[(4)] Should one or more provisions of this Sideletter be or become invalid or unenforceable in whole or in part, this shall not affect the validity and enforceability of the remaining provisions of this Sideletter. In place of any Standard Terms of Business (\textsl{Allgemeine Geschäftsbedingungen}) which are invalid or not incorporated in the Sideletter the statutory provisions shall apply (§ 306 (2) of the German Civil Code (BGB)). In all other cases, the parties shall agree a valid provision to replace the invalid or unenforceable provision which reflects as closely as possible the original economic purpose, provided a supplementary interpretation of the Sideletter (\textsl{ergänzende Vertragsauslegung}) does not have precedence or is not possible.
\item[(5)] Amendments and additions to this Sideletter shall be valid only if made in writing. This also applies to any amendment to this written form clause. 
\item[(6)] Any disputes arising out of or in connection with this Sideletter, including disputes on its conclusion, binding effects, amendment and termination, shall be dealt with exclusively by the competent court in Berlin, Germany, if legally possible.
\end{itemize}
