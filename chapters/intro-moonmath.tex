\chapter{Introduction}
% ATTENTION! THIS IS ALL COPY PASTED FROM SOMEWHERE; SO CAN ONLY BE USED AS GUIDENCE AND NEEDS TO BE REWRITTEN

This is dump from other papers as inspiration for the intro:

% Climate Dao whitepaper

Zero knowledge proofs are a class of cryptographic protocols in which one can prove
honest computation without revealing the inputs to that computation. A simple high-level
example of a zero-knowledge proof is the ability to prove one is of legal voting age
without revealing the respective age. In a typical zero knowledge proof system, there
are two participants: a prover and a verifier. A prover will present a mathematical proof
of computation to a verifier to prove honest computation. The verifier will then confirm
whether the prover has performed honest computation based on predefined methods.
Zero knowledge proofs are of particular interest to public blockchain activities as the
verifier can be codified in smart contracts as opposed to trusted parties or third-party
intermediaries.

% https://docs.zkproof.org/reference.pdf

Zero-knowledge proofs (ZKPs) are an important privacy-enhancing tool from cryptography. Theyallow proving the veracity of a statement, related to confidential data, without revealing any in-formation beyond the validity of the statement. ZKPs were initially developed by the academiccommunity in the 1980s, and have seen tremendous improvements since then. They are now ofpractical feasibility in multiple domains of interest to the industry, and to a large community ofdevelopers and researchers. ZKPs can have a positive impact in industries, agencies, and for per-sonal use, by allowing privacy-preserving applications where designated private data can be madeuseful to third parties, despite not being disclosed to them. 

ZKP systems involve at least two parties: a prover and a verifier. The goal of the prover is toconvince the verifier that a statement is true, without revealing any additional information. Forexample, suppose the prover holds a birth certificate digitally signed by an authority. In orderto access some service, the prover may have to prove being at least 18 years old, that is, thatthere exists a birth certificate, tied to the identify of the prover and digitally signed by a trustedcertification authority, stating a birthdate consistent with the age claim. A ZKP allows this, withoutthe prover having to reveal the birthdate.


\section{Target audience}

% copy pasted from https://claritybook.netlify.app/ch00-00-introduction.html
% need adoption to our case
This book is accessible for both beginners and experienced developers alike. Concepts are gradually introduced in a logical and steady pace. Nonetheless, the chapters lend themselves rather well to being read in a different order. More experienced developers might get the most benefit by jumping to the chapters that interest them most. If you like to learn by example, then you should go straight to the chapter on Using Clarinet.

It is assumed that you have a basic understanding of programming and the underlying logical concepts. The first chapter covers the general syntax of Clarity but it does not delve into what programming itself is all about. If this is what you are looking for, then you might have a more difficult time working through this book unless you have an (undiscovered) natural affinity for such topics. Do not let that dissuade you though, find an introductory programming book and press on! The straightforward design of Clarity makes it a great first language to pick up.

\section{The Zoo of Zero-Knowledge Proofs}

{First, a list of zero-knowledge proof systems:

\begin{enumerate}
	\item Pinocchio (2013): \href{https://eprint.iacr.org/2013/279.pdf}{{Paper}}
	\begin{itemize}[label={--}]
		\item Notes: trusted setup
	\end{itemize}

	\item BCGTV (2013): \href{https://eprint.iacr.org/2013/507.pdf}{{Paper}}
	\begin{itemize}[label={--}]
		\item Notes: trusted setup, implementation
	\end{itemize}

	\item BCTV (2013): \href{https://eprint.iacr.org/2013/879.pdf}{{Paper}}
	\begin{itemize}[label={--}]
		\item Notes: trusted setup, implementation
	\end{itemize}

	\item Groth16 (2016): 	\href{https://eprint.iacr.org/2016/260.pdf}{Paper}
	\begin{itemize}[label={--}]
		\item Notes: trusted setup
		\item Other resources: \href{https://www.gakonst.com/zksummit2019.pdf}{Talk in 2019 by Georgios Konstantopoulos}
	\end{itemize}

	\item GM17 (207): 	\href{https://eprint.iacr.org/2017/540.pdf}{Paper}
	\begin{itemize}[label={--}]
		\item Notes: trusted setup
		\item Other resources: later \href{https://eprint.iacr.org/2018/187}{Simulation extractability in ROM, 2018}
	\end{itemize}

	\item Bulletproofs (2017): \href{https://eprint.iacr.org/2017/1066.pdf}{Paper}
	\begin{itemize}[label={--}]
		\item Notes: no trusted setup
		\item Other resources: \href{https://eprint.iacr.org/2016/263.pdf}{Polynomial Commitment Scheme on DL, 2016} and \href{https://www.iacr.org/archive/asiacrypt2010/6477178/6477178.pdf}{KZG10, Polynomial Commitment Scheme on Pairings, 2010}
	\end{itemize}

	\item Ligero (2017): \href{https://acmccs.github.io/papers/p2087-amesA.pdf}{Paper}
	\begin{itemize}[label={--}]
		\item Notes: no trusted setup
		\item Other resources: 
	\end{itemize}

	\item Hyrax (2017): \href{https://eprint.iacr.org/2017/1132.pdf}{Paper}
	\begin{itemize}[label={--}]
		\item Notes: no trusted setup
		\item Other resources: 
	\end{itemize}

	\item STARKs (2018): \href{https://eprint.iacr.org/2018/046.pdf}{Paper}
	\begin{itemize}[label={--}]
		\item Notes: no trusted setup 
		\item Other resources: 
	\end{itemize}

	\item Aurora (2018): \href{https://eprint.iacr.org/2018/828.pdf}{Paper}
	\begin{itemize}[label={--}]
		\item Notes: transparent SNARK
		\item Other resources:
	\end{itemize}

	\item Sonic (2019): \href{https://eprint.iacr.org/2019/099.pdf}{Paper}
	\begin{itemize}[label={--}]
		\item Notes: SNORK - SNARK with universal and updateable trusted setup, PCS-based
		\item Other resources: \href{https://www.benthamsgaze.org/2019/02/07/introducing-sonic-a-practical-zk-snark-with-a-nearly-trustless-setup/}{Blog post by Mary Maller from 2019} and \href{https://eprint.iacr.org/2018/280}{work on updateable and universal setup from 2018}
	\end{itemize}

	\item Libra (2019): \href{https://eprint.iacr.org/2019/317}{Paper}
	\begin{itemize}[label={--}]
		\item Notes: trusted setup
		\item Other resources:
	\end{itemize}

	\item Spartan (2019): \href{https://eprint.iacr.org/2019/550.pdf}{Paper}
	\begin{itemize}[label={--}]
		\item Notes: transparent SNARK
		\item Other resources:
	\end{itemize}

	\item PLONK (2019): \href{https://eprint.iacr.org/2019/953.pdf}{Paper}
	\begin{itemize}[label={--}]
		\item Notes: SNORK, PCS-based
		\item Other resources: \href{https://www.plonk.cafe/t/welcome-to-discussion-of-plonk-related-research/24}{Discussion on Plonk systems} and \href{https://github.com/Fluidex/awesome-plonk}{Awesome Plonk list}
	\end{itemize}

	\item Halo (2019): \href{https://eprint.iacr.org/2019/1021}{Paper}
	\begin{itemize}[label={--}]
		\item Notes: no trusted setup, PCS-based, recursive
		\item Other resources: 
	\end{itemize}

	\item Marlin (2019): \href{https://eprint.iacr.org/2019/1047.pdf}{Paper}
	\begin{itemize}[label={--}]
		\item Notes: SNORK, PCS-based
		\item Other resources: \href{https://github.com/arkworks-rs/marlin}{Rust Github}
	\end{itemize}

	\item Fractal (2019): \href{https://eprint.iacr.org/2019/1076.pdf}{Paper}
	\begin{itemize}[label={--}]
		\item Notes: Recursive, transparent SNARK
		\item Other resources: 
	\end{itemize}

	\item SuperSonic (2019): \href{https://eprint.iacr.org/2019/1229.pdf}{Paper}
	\begin{itemize}[label={--}]
		\item Notes: transparent SNARK, PCS-based
		\item Other resources: \href{https://eprint.iacr.org/2021/358}{Attack on DARK compiler in 2021}
	\end{itemize}

	\item Redshift (2019): \href{https://eprint.iacr.org/2019/1400}{Paper}
	\begin{itemize}[label={--}]
		\item Notes: SNORK, PCS-based
		\item Other resources: 
	\end{itemize}



\end{enumerate}

\textbf{Other resources on the zoo: } \href{https://github.com/matter-labs/awesome-zero-knowledge-proofs}{Awesome ZKP list on Github}, \href{https://zkp.science/}{ZKP community} with the \href{https://docs.zkproof.org/reference.pdf}{reference document}

}

\paragraph{To Do List}
\begin{itemize}
	\item Make table for prover time, verifier time, and proof size
	\item Think of categories - \textit{Achieved Goals}: Trusted setup or not, Post-quantum or not, \dots
	\item Think of categories - \textit{Mathematical background}: Polynomial commitment scheme, \dots
	\item \dots while we discuss the points above, we should also discuss a common notation/language for all these things. (E.g. transparent SNARK/no trusted setup/STARK)
\end{itemize}

\paragraph{Points to cover while writing}
\begin{itemize}
	\item Make a historical overview over the "discovery" of the different ZKP systems
	\item Make reader understand what paper is build on what result etc. - the tree of publications!
	\item Make reader understand the different terminology, e.g. SNARK/SNORK/STARK, PCS, R1CS, updateable, universal, $\dots$
	\item Make reader understand the mathematical assumptions - and what this means for the zoo.
	\item Where will the development/evolution go? What are bottlenecks?
\end{itemize}

\vspace*{1em}
{\footnotesize
\hspace*{-1em}\textbf{Other topics I fell into while compiling this list}
\begin{itemize}
	\item Vector commitments: \url{https://eprint.iacr.org/2020/527.pdf}
	\item Snarkl: \url{http://ace.cs.ohio.edu/~gstewart/papers/snaarkl.pdf}
	\item Virgo?: \url{https://people.eecs.berkeley.edu/~kubitron/courses/cs262a-F19/projects/reports/project5_report_ver2.pdf}
\end{itemize} 
}


