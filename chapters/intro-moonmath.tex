\chapter{Introduction}

This is dump from other papers as inspiration for the intro:

% https://docs.zkproof.org/reference.pdf

Zero-knowledge proofs (ZKPs) are an important privacy-enhancing tool from cryptography. Theyallow proving the veracity of a statement, related to confidential data, without revealing any in-formation beyond the validity of the statement. ZKPs were initially developed by the academiccommunity in the 1980s, and have seen tremendous improvements since then. They are now ofpractical feasibility in multiple domains of interest to the industry, and to a large community ofdevelopers and researchers. ZKPs can have a positive impact in industries, agencies, and for per-sonal use, by allowing privacy-preserving applications where designated private data can be madeuseful to third parties, despite not being disclosed to them. 

ZKP systems involve at least two parties: a prover and a verifier. The goal of the prover is toconvince the verifier that a statement is true, without revealing any additional information. Forexample, suppose the prover holds a birth certificate digitally signed by an authority. In orderto access some service, the prover may have to prove being at least 18 years old, that is, thatthere exists a birth certificate, tied to the identify of the prover and digitally signed by a trustedcertification authority, stating a birthdate consistent with the age claim. A ZKP allows this, withoutthe prover having to reveal the birthdate.


\section{Target audience}

% copy pasted from https://claritybook.netlify.app/ch00-00-introduction.html
% need adoption to our case
This book is accessible for both beginners and experienced developers alike. Concepts are gradually introduced in a logical and steady pace. Nonetheless, the chapters lend themselves rather well to being read in a different order. More experienced developers might get the most benefit by jumping to the chapters that interest them most. If you like to learn by example, then you should go straight to the chapter on Using Clarinet.

It is assumed that you have a basic understanding of programming and the underlying logical concepts. The first chapter covers the general syntax of Clarity but it does not delve into what programming itself is all about. If this is what you are looking for, then you might have a more difficult time working through this book unless you have an (undiscovered) natural affinity for such topics. Do not let that dissuade you though, find an introductory programming book and press on! The straightforward design of Clarity makes it a great first language to pick up.

