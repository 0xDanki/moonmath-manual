\chapter{Software Used in This Book}

\section{Sagemath}
\label{sagemath_setup}
It order to provide an interactive learning experience, and to allow getting hands-on with the concepts described in this book, we give examples for how to program them in the \href{https://www.sagemath.org/}{SageMath} programming language. Sage is based on the learning-friendly programming language \href{https://www.python.org/}{Python},  extended and optimized for computations involving algebraic objects. Therefore, we recommend installing Sage before diving into the following chapters.

The installation steps for various system configurations are described on the \href{https://doc.sagemath.org/html/en/installation/index.html}{Sage website}. Note that we use Sage version 9, so if you are using Linux and your package manager only contains version 8, you may need to choose a different installation path, such as using prebuilt binaries. If you are not familiar with SageMath, we recommend you consult the \href{https://doc.sagemath.org/html/en/tutorial/index.html}{Sage Tutorial}.

\section{Circom}
\label{circom_setup}
It order to compile our pen-and-paper calculations into real world zk-SNARKs, we give examples for how to implement them in the \href{https://iden3.io/circom}{Circom} domain-specific language. Circom is a circuit programming language and a compiler that allows programmers to design arithmetic circuits. It uses snarkjs as its underlying zero knowledge proof system. The installation steps are described at \href{https://docs.circom.io/getting-started/installation/#installing-circom}{Circom installation}.

