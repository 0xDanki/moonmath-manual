\chapter{Zk-Proof Systems}

Some philosophical stuff about compuational models for snarks. Bounded computability...

% https://docs.zkproof.org/reference.pdf

\section{Computational Models}
Proofs are the evidence of correctness of the assertions, and people can verify the cor-rectness by reading the proof. However, we obtain much more than the correctness itself:After you read one proof of an assertion, you know not only the correctness, but also why itis correct. Is it possible to solely show the correctness of an assertion without revealing theknowledge of proofs? It turns out that it is indeed possible, and this is the topic of today’slecture: Zero Knowledge Systems.
% from http://resources.mpi-inf.mpg.de/departments/d1/teaching/ss14/gitcs/notes6.pdf

\begin{example}[Generalized factorization snark]
\label{main_example_2_1}
As one of our major running examples we want to derive a zk-SNARK for the following generalized factorization problem: 

Given two numbers $a,b\in \mathbb{F}_{13}$, find two additional numbers $x,y\in \mathbb{F}_{13}$, such that
$$
(x\cdot y) \cdot a = b 
$$
and proof knowledge of those numbers, without actually revealing them.

Of course this example reduces to the classic factorization problem (over $\F_{13}$ by setting $y=1$)

This zero knowledge system deals with the following situation: "Given two publicly known numbers $a,b \in \mathbb{F}_{13}$ a proofer can show that they know two additional numbers $x,y\in \mathbb{F}_{13}$, such that $(x\cdot y) \cdot a = b$, without actually revealing $x$ or $y$." 

Of course our choice of what information to hide and what to reveal was completely arbitrary. Every other split would also be possible, but eventually gives a different problem. 

For example the task could be to not hide any of the variables.  Such 
a system has no zero knowledge and deals with verifiable computations: "A worker can proof that they multiplied three publicly known numbers $a,b,x \in \mathbb{F}_{13}$ and that the result is $z \in \mathbb{F}_{13}$, in such a way that no verifier has to repeat the computation."
\end{example}

\subsection{Formal Languages}
Roughly speaking a formal language is nothing but a set of words, that are strings of letters taken from some alphabet and formed according to some defining rules of that language. 

In computer science, formal languages are used for defining the grammar of programming languages in which the words of the language represent concepts that are associated with particular meanings or semantics. In computational complexity theory, decision problems are typically defined as formal languages, and complexity classes are defined as the sets of the formal languages that can be parsed by machines with limited computational power. 

\begin{definition}[Formal Language]
\label{def_formal_language}
 Let $\Sigma$ be a set and $\Sigma^*$ the set of all finite strings of elements from $\Sigma$. Then a \textbf{formal language} $L$ is a subset of $\Sigma^*$. The set $\Sigma$ is called the \textbf{alphabet} of $L$ and elements from $L$ are called \textbf{words}. The rules that specify which strings from $\Sigma^*$ belong to $L$ are called the \textbf{grammar} of $L$. 

In the context of proofing systems we often call words \textbf{statements}.
\end{definition}

\begin{example}[Generalized factorization snark]
\label{main_example_2_2}
Consider example \ref{main_example_2_1} again. Definition \ref{def_formal_language} is not quite suitable yet to define the example, since there is not distinction between public input and private input.

However if we assume for the moment that the task in example \ref{main_example_2_1} is to simply find $a,b,x,y\in \F_{13}$ such that that $x\cdot y\cdot a\cdot =b$, then we can define the entire solution set as a language $L_{factor}$ over the alphabet $\Sigma = \F_{13}$. We then say that a string $w\in \Sigma^*$ is a statement in our language $L_{factor}$ if and only if $w$ consists of 4 letters $w_1,w_2,w_3,w_4$ that satisfy the equation $w_1\cdot w_2\cdot w_3 =w_4$.
\end{example}

\begin{example}[Binary strings] If we take the set $\{0,1\}$ as our alphabet $\Sigma$ and imply no rules at all to form words in this set. Then our language $L$ is the set $\{0,1\}^*$ of all finite binary strings. So for example $(0,0,1,0,1,0,1,1,0)$ is a word in this language.
\end{example}

\begin{example}[Programing Language]
\end{example}

\begin{example}[Compiler]
\end{example}



As we have seen in general not all strings from an alphabet are words in a language. So an important question is, weather a given string belongs to a language or not. 

% https://www.claymath.org/sites/default/files/pvsnp.pdf
\begin{definition}[Relation, Statement, Instance and Witness] Let $\Sigma_I$ and $\Sigma_W$ be two alphabets. Then the binary relation $R\subset \Sigma_I^* \times \Sigma_W^*$ is called a \textbf{checking relation} for the language 
$$
L_R := \{(i,w) \in \Sigma_I^* \times \Sigma_W^*\;| R(i,w)\; \}
$$ 
of all \textbf{instances} $i\in \Sigma_I^*$ and \textbf{witnesses} $i\in \Sigma_I^*$, such that the \textbf{statement} $(i,w)$ satisfies the checking relation.
\end{definition}
\begin{remark}
% https://docs.zkproof.org/reference.pdf
To summarize the definition, a statement is nothing but a membership claim of the form $x\in L$. So statements are really nothing but strings in an alphabet that adhere to the rules of a language. 

However in the context of checking relations, there is another interpretations in terms of a knowledge claim of the form "In the scope of relation R, I know a witness for instance x." This is of particular importance in the context of zero knowledge proofing systems, where the instance represents public knowledge, while the witness represents the data that is hidden (the zero-knowledge part). 

For some cases, the knowledge and membership types of statements can be informally considered interchangeable, but formally there are technical reasons to distinguish between the two notions (See for example XXX
% https://docs.zkproof.org/reference.pdf
) 
\end{remark}
\begin{example}[Generalized factorization snark]
\label{main_example_2_3}
Consider example \ref{main_example_2_1} and our associate formal language \ref{main_example_2_2}. We can define another language $L_{zk-factor}$ for that example by defining the alphabet $\Sigma_I \times \Sigma_W$ to be $\F_{13} \times \F_{13}$ and the checking relation $R_{zk-factor}$ such that
$R(i,w)$ holds if and only if instance $i$ is a two letter string $i=(a,b)$ and witness $w$ is a two letter string $w=(x,y)$, such that the equation $x\cdot y \cdot a = b$ holds. 

So to summarize four elements $x,y,a,b\in \F_{13}$ form a statement 
$((x,y),(a,b))$ in $L_{zk-factor}$ with instance $(a,b)$ and witness $x,y$, precisely if, given $a$ and $b$, the values $x$ and $y$ are a solution to the generalized factorization problem $x\cdot y \cdot a = b$.
\end{example}




\begin{example}[SHA256 relation]
ssss
\end{example}

As the following example shows checking relations and their languages are quite general and able to express in particular the class of all terminating computer programs:
\begin{example}[Computer Program] Let $A$ be a terminating algorithm that transforms a binary string of inputs in finite execution steps into a binary output string. We can then interpret $A$ as a map 
$$
A :\{0,1\}^* \to \{0,1\}^*
$$
Algorithm $A$ then defines a relation
$R\subset \{0,1\}^* \times \{0,1\}^*$ in the following way: instance string $i\in \{0,1\}^*$ and witness string $w\in \{0,1\}^*$ satisfy the relation $R$, that is $R(i,w)$, if and only if $w$ is the result of algorithm $A$ executed on input instance $i$.
\end{example}

\subsection{Circuits} 
\begin{definition}[Circuits] Let $\Sigma_I$ and $\Sigma_W$ be two alphabets. Then a directed, acyclic graph $C$ is called a \textbf{circuit} over $\Sigma_I \times \Sigma_W$, if the graph has an ordering and every node has a label in the following way:
\begin{itemize}
\item Every source node (called input) has a letter from $\Sigma_I \times \Sigma_W$ as label.
\item Every sink node (called output) has a letter from $\Sigma_I \times \Sigma_W$ as label.
\item Every other node (called gate) with $j$ incoming edges has a label that consist of a function $f: \left(\Sigma_I \times \Sigma_W\right)^j \to \Sigma_I \times \Sigma_W$.
\end{itemize}
\end{definition}
\begin{remark}[Circuit-SAT] Every circuit with $n$ input nodes and $m$ output nodes can be seen a function that transforms strings of size $n$ from $\Sigma_I \times \Sigma_W$ into strings of size $m$ over the same alphabet. The transformation is done by sending the strings from a node along the outgoing edges to other nodes. If those nodes are gates, then the string is transformed according to the label.

By executing the previous transformation, every node of a circuit has an associated letter from $\Sigma_I \times \Sigma_W$ and this defines a checking relation over $\Sigma_I^* \times \Sigma_W^*$. To be more precise, let $C$ be a circuit with $n$ nodes and $(i,w) \in \Sigma_I^j \times \Sigma_W^k$ a string. Then $R_C(i,w)$ iff THE CIRCUIT IS SATISFIED WHEN ALL LABELS ARE ASSOCIATED TO ALL NODES IN THE CIRCUIT.... BUT MORE PRECISE

MODULO ERRORS. TO BE CONTINUED.....

An Assignment associates field elements to all edges (indices) in an algebraic circuit. An Assignment is valid, if the field element arise from executing the circuit. Every other assignment is invalid.

The checking relation for circuit-SAT then is satidfied if valid asignment (TODO: THE WITNESS/INSTANCE SPLITTING)

Valid assignments are proofs for proper circuit execution.
\end{remark}



So to summarize, algebraic circuits (over a field $\mathbb{F}$) are directed acyclic graphs, that express arbitrary, but bounded computation. Vertices with only outgoing edges (leafs, sources) represent inputs to the computation, vertices with only ingoing edges (roots, sinks) represent outputs from the computation and internal vertices represent field operations (Either addition or multiplication). It should be noted however that there are many circuits that can represent the same laguage...

Circuits have a notion of execution, where input values are send from leafs along edges, through internal vertices to roots.

\begin{remark}
Algebraic circuits are usually derived by  Compilers, that transform  higher languages to circuits. An example of such a compiler is XXX. Note: Different Compiler give very different circuit representations and Compiler optimization is important.
\end{remark}


\begin{example}[Generalized factorization snark]
\label{main_example_2_4}
Consider our generalized factorization example \ref{main_example_2_1} with associated language \ref{main_example_2_3}.

To write this example in circuit-SAT, consider the following function 
\[
f:\mathbb{F}_{13}\times\mathbb{F}_{13}\times\mathbb{F}_{13}\to\mathbb{F}_{13};(x_{1},x_{2},x_{3})\mapsto(x_{1}\cdot x_{2})\cdot x_{3}
\]

A valid circuit for $f:\mathbb{F}_{11}\times\mathbb{F}_{11}\times\mathbb{F}_{11}\to\mathbb{F}_{11};(x_{1},x_{2},x_{3})\mapsto(x_{1}\cdot x_{2})\cdot x_{3}$ is given by:

\[
\xymatrix{\star\ar^{in_1}[dr] &  & \star\ar_{in_2}[dl]\\
 & \star_{m_1}\ar^{mid_1}[drr] &   & & \star\ar_{in_3}[dl]\\
  &  &  & \star_{m_2}\ar_{out_1}[d]\\
  &  &  & \star
}
\]
with edge-index set $I:=\{in_{1},in_{2},in_{3},mid_{1},out_{1}\}$.

To given a valid assignment, consider the set $I_{valid}:=\{in_{1},in_{2},in_{3},mid_{1},out_{1}\} = \{2,3,4,6,10\}$

\[
\xymatrix{\star\ar^{2}[dr] &  & \star\ar_{3}[dl]\\
 & \star_{m_1}\ar^{6}[drr] &   & & \star\ar_{4}[dl]\\
  &  &  & \star_{m_2}\ar_{10}[d]\\
  &  &  & \star
}
\]
Appears from multiplying the input values at $m_1$, $m_2$ in $\mathbb{F}_{13}$, hence by executing the circuit.

Non valid assignment: $I_{err}:=\{in_{1},in_{2},in_{3},mid_{1},out_{1}\} =\{2,3,4,7,8\}$
\[
\xymatrix{\star\ar^{2}[dr] &  & \star\ar_{3}[dl]\\
 & \star_{m_1}\ar^{7}[drr] &   & & \star\ar_{4}[dl]\\
  &  &  & \star_{m_2}\ar_{8}[d]\\
  &  &  & \star
}
\]
Can not appear from multiplying the input values at $m_1$, $m_2$ in $\mathbb{F}_{13}$

To match the requirements of the inital task \ref{main_example_2_1}, we have to split the statement into instance and witness. So given index set $I:=\{in_{1},in_{2},in_{3},mid_{1},out_{1}\}$, we assume that every step in the computation other then $in_3$ and $out_1$ are part of the witness. So we choose:
\begin{itemize}
\item Instance $S=\{in_3, out_1\}$. 
\item Witness $W=\{in_1, in_2, mid_{1}\}$.
\end{itemize}
\end{example}

\begin{example}[Baby JubJub for BLS6-6]

\end{example}

\begin{example}[ECDH as a circuit]
over BLS6
\end{example}

\begin{example}[BLS Signature]
example of one layer recursion over MNT4 and MNT6
\end{example}


\begin{example}[Boolean Circuits]

\end{example}

\begin{example}[Algebraic (Aithmetic) Circuits]

\end{example}

Any program  can be reduced to  an arithmetic circuit  (a circuit that contains only addition and multiplication gates). A particular reduction can be found for example in [BSCG+13]



\subsection{Rank-1 Constraint Systems}

\begin{definition}[Rank-1 Constraint system]
Let $\F$ be a Galois field, $i,j,k$ three numbers and $A$, $B$ and $C$ three $(i+j+1) \times k$ matrices with coefficients in $\F$. Then any vector $x= (1,\phi,w)\in \F^{1+i+j}$ that satisfies the \textbf{rank-1 constraint system} (R1CS)
$$
Ax \odot Bx = Cx
$$
(where $\odot$ is the Hadamard/Schur product) is called a \textbf{statement} of that system, with \textbf{instance} $\phi$ and \textbf{witness} $w$.

We call $k$ the \textbf{number of constraints}, $i$ the \textbf{instance} size and $j$ the \textbf{witness} size.
\end{definition}

\begin{remark} Any Rank-1 constraint system defines a formal language in the following way: Consider the alphabets $\Sigma_I:= \F$ and $\Sigma_W:\F$. Then a checking relation $R_{R1CS} \subset \Sigma_I^i \times \Sigma_W^j \subset \Sigma_I^* \times \Sigma_W^*$ is defined by 
$$
R_{R1CS}(i,w) \Leftrightarrow (i,w)\text{ satisfies the R1CS}
$$
As shown in XXX such a checking relation defines a formal language. We call this language \textbf{R1CS satisfiability}.
\end{remark}

\begin{example}[Generalized factorization snark]
\label{main_example_2_4}
Defining the 5-dimensional affine vector $w =(1,in_1,in_2,in_3,m_1,out_1)$ for $in_1,in_2,in_3,m_1,out_1 \in \F_{13}$ and the $6\times ?$-matrices
$$
\begin{array}{lcr}
A = \begin{pmatrix}
0 & 1 & 0 & 0 & 0 & 0 \\ 
0 & 0 & 0 & 0 & 1 & 0
\end{pmatrix}, &
B = \begin{pmatrix}
0 & 0 & 1 & 0 & 0 & 0 \\ 
0 & 0 & 0 & 1 & 0 & 0
\end{pmatrix}, &
C = \begin{pmatrix}
0 & 0 & 0 & 0 & 1 & 0 \\ 
0 & 0 & 0 & 0 & 0 & 1
\end{pmatrix} 
\end{array}
$$
We can instantiate the general R1CS equation $Aw \odot Bw = Cw$ as
$$
\begin{pmatrix}
0 & 1 & 0 & 0 & 0 & 0 \\ 
0 & 0 & 0 & 0 & 1 & 0
\end{pmatrix} 
\begin{pmatrix}
1\\ in_1 \\ in_2 \\ in_3 \\ m_1 \\ out_1 
\end{pmatrix}\odot 
\begin{pmatrix}
0 & 0 & 1 & 0 & 0 & 0 \\ 
0 & 0 & 0 & 1 & 0 & 0
\end{pmatrix} 
\begin{pmatrix}
1\\ in_1 \\ in_2 \\ in_3 \\ m_1 \\ out_1 
\end{pmatrix} =
\begin{pmatrix}
0 & 0 & 0 & 0 & 1 & 0 \\ 
0 & 0 & 0 & 0 & 0 & 1
\end{pmatrix} 
\begin{pmatrix}
1\\ in_1 \\ in_2 \\ in_3 \\ m_1 \\ out_1 
\end{pmatrix}
$$
So evaluating all three matrix products and the Hadarmat prodoct we get two constraint equations
$$
\begin{array}{rcl}
in_1 \cdot in_2  &= & m_1 \\
m_1 \cdot in_3  &= & out_1 \\
\end{array}
$$
\end{example}

\subsection{Quadratic Arithmetic Programs}
As shown by [Pinocchio] rank-1 constraint systems can be transformed into so called quadratic  arithmetic  programs  assuming $\F$.

taken from the pinocchio paper. For proving arithmetic circuit-sat.  Given a R1CS QAPs transform potential solution vectors into two polynomials $p$ and $t$, such that $p$ is divisible by $t$ if and only if the vector is a solution to the R1CS. 

They are major building blocks for \textbf{succinct} proofs, since with high probability, the divisibility check can be performed in a single point of those polynomials. So computationally expensive polynomial division check is reduced TO WHAT? (IN FIELDS THERE IS ALWAYS DIVISIBILITY) 
% https://courses.cs.ut.ee/MTAT.07.022/2013_fall/uploads/Main/alisa-report

\begin{definition}[Quadratic Arithmetic Program]
Assume we have a Galois field $\F$, three numbers $i,j,k$ as well as three $(i+j+1) \times k$ matrices $A$, $B$ and $C$  with coefficients in $\F$ that define the R1CS
$Ax \odot Bx = Cx $ for some statement $x=(1,i,w)$ and let $m_1,\ldots,m_k\in \F$ be arbitrary field elements. 

Then a \textbf{quadratic arithmetic program} of the R1CS is the following set of polynomials over $\F$
$$
QAP = \left\{t\in \F[x],\left\{a_h,b_h,c_h\in \F[x]\right\}_{h=1}^{i+j+1}\right\}
$$
where $t(x) := \Pi_{l=1}^k (x- m_l)$ is a polynomial f degree $k$, called the \textbf{target polynomial} of the QAP and $a_h(x)$, $b_h(x)$ as well as $c_h(x)$ are the unique degree $k-1$ polynomials that are defined by the equations
$$
\begin{array}{lllr}
a_h(m_l)=A_{h,l} & b_h(m_l)=B_{h,l} & c_h(m_l)=C_{h,l} & h= 1, \ldots , i+j+1, l=1,\ldots,k 
\end{array}
$$  
\end{definition}
The major point is that R1CS-sat can be reformulated into the divisibility of a polynomials defined by any QAP.
\begin{theorem}
Assume that an R1CS and an associated QAP as defined in XXX are given. Then the affine vector $y=(1,i,w)$ is a solution to the R1CS, if and only if the polynomial
$$
p(x) = \left(\sum y_h\cdot a_h(x)\right)\cdot \left(\sum y_h\cdot b_h(x)\right)  - \sum y_h\cdot c_h(x) 
$$
is divisible by the target polynomial $t$.
\end{theorem}

The polynomials $a_h$, $b_h$ and $c_h$ are uniquely defined by the equations in XXX. However to actually compute them we need some algorithm like the Langrange XXX from XXX.

\begin{example}[Generalized factorization snark]
In this example we want to transform the R1CS from example \ref{main_example_2_3} into an associated QAP.

We start by choosing an arbitrary field element for every constraint in the R1CS, since we have $2$ constraints we choose $m_{1}=5$ and $m_{2}=7$

With this choice we get the target polynomial $t(x)=(x-m_1)(x-m_2)= (x-5)(x-7)= (x+8)(x+6)= x^2 + x +9$.

Since our statement has structure $w=(1, in_1,in_2,in_3,m_1,out_1)$ we have to compute the following degree $1$ polynomials

$\{a_{c},a_{in_{1}},a_{in_{2}},a_{in_{3}},a_{mid_{1}},a_{out}\}$
$\{b_{c},b_{in_{1}},b_{in_{2}},b_{in_{3}},b_{mid_{1}},b_{out}\}$
$\{c_{c},c_{in_{1}},c_{in_{2}},c_{in_{3}},c_{mid_{1}},c_{out}\}$

\item Apply QAP rule XXX to the $a_{k\in I}$ polynomials gives
$$
\begin{array}{llllll}
a_{c}(5)=0, & a_{in_{1}}(5)=1, & a_{in_{2}}(5)=0, & a_{in_{3}}(5)=0, & a_{mid_{1}}(5)=0, & a_{out}(5)=0 \\
a_{c}(7)=0, & a_{in_{1}}(7)=0, & a_{in_{2}}(7)=0, & a_{in_{3}}(7)=0, & a_{mid_{1}}(7)=1, & a_{out}(7)=0\\
\\
b_{c}(5)=0, & b_{in_{1}}(5)=0, & b_{in_{2}}(5)=1, & b_{in_{3}}(5)=0, & b_{mid_{1}}(5)=0, & b_{out}(5)=0 \\
b_{c}(7)=0, & b_{in_{1}}(7)=0, & b_{in_{2}}(7)=0, & b_{in_{3}}(7)=1, & b_{mid_{1}}(7)=0, & b_{out}(7)=0\\
\\
c_{c}(5)=0, & c_{in_{1}}(5)=0, & c_{in_{2}}(5)=0, & c_{in_{3}}(5)=0, & c_{mid_{1}}(5)=1, & c_{out}(5)=0 \\
c_{c}(7)=0, & c_{in_{1}}(7)=0, & c_{in_{2}}(7)=0, & c_{in_{3}}(7)=0, & c_{mid_{1}}(7)=0, & c_{out}(7)=1
\end{array}
$$

Since our polynomials are of degree $1$ only we don't have to invoke Langrange method but can deduce the solutions right away. 

Polynomials are defined on the two values $5$ and $7$ here.
Linear Polynomial $f(x)=m\cdot x + b$ is fully determined by this. Derive the general equation:
\begin{itemize}                        
\item  $5m+b=f(5)$  and $7m+b=f(7)$  
\item  $b=f(5)-5m$ and  $b=f(7)-7m$   
\item  $b=f(5)+8m$ and  $b=f(7)+6m$  
\item  $f(5)+8m=f(7)+6m$              
\item  $8m-6m=f(7)-f(5)$               
\item  $2m=f(7)+ 12f(5)$              
\item  $7\cdot 2m=7(f(7)+12f(5))$              
\item  $m=7(f(7)+12f(5))$ 
\item             
\item  $b=f(5)+8m$                   
\item  $b=f(5)+8\cdot(7(f(7)+12f(5)))$
\item  $b=f(5)+4(f(7)+12f(5))$ 
\item  $b=f(5)+4f(7)+9f(5)$ 
\item  $b= 10f(5)+4f(7)$ 
\end{itemize}
Gives the general equation: $f(x)=7(f(7)+12f(5))x+10f(5)+4f(7)$

For $a_{in_1}$ the computation looks like this:
\begin{itemize}
\item $ a_{in_{1}}(x) = 7(a_{in_{1}}(7)+12a_{in_{1}}(5))x+ 
10a_{in_{1}}(5)+4a_{in_{1}}(7)=$
\item $7(0 + 12\cdot 1)x+ 
10\cdot 1 +4\cdot 0 =$
\item $7\cdot 12 x + 10=$
\item $6x+10$
\end{itemize}
\begin{itemize}
\item $ a_{mid_{1}}(x) = 7(a_{mid_{1}}(7)+12a_{mid_{1}}(5))x+ 
10a_{mid_{1}}(5)+4a_{mid_{1}}(7)=$
\item $7(1 + 12\cdot 0)x+ 10\cdot 0 +4\cdot 1=$
\item $7\cdot 1x +4=$
\item $7x+4 $
\end{itemize}


\begin{tabular}{|l|l|l|}\hline 
$a_{c}(x)=0 $ &$ b_{c}(x)=0   $ & $c_{c}(x)=0$ \tabularnewline\hline 
$a_{in_{1}}(x)=6x+10 $ &$ b_{in_{1}}(x)=0   $ & $c_{in_1}(x)=0$ \tabularnewline\hline 
$a_{in_{2}}(x)=0    $ &$ b_{in_{2}}(x)=6x+10$ & $c_{in_2}(x)=0$ \tabularnewline\hline 
$a_{in_{3}}(x)=0    $ &$ b_{in_{3}}(x)=7x+4$ & $c_{in_{3}}(x)=0$ \tabularnewline\hline 
$a_{mid_{1}}(x)=7x+4$ &$ b_{mid_{1}}(x)=0  $ & $c_{mid_{1}}(x)=6x+10$ \tabularnewline\hline 
$a_{out}(x)=0       $ &$ b_{out}(x)=0      $ & $c_{out}(x)=7x+4$ \tabularnewline\hline 
\end{tabular}
This gives the quadratic arithmetic program for our generalized factorization snark as
$$QAP=\{x^{2}+x+9,\{0,6x+10,0,0,7x+4,0\},\{0,0,6x+10,7x+4,0,0\},\{0,0,0,0,6x+10,7x+4\}\}$$

Now as we recall, the main point for using QAPs in snarks is the fact, that solutions to R1CS are in 1:1 correspondence to the divisibility of a polynomial $p$, constructed from a R1CS solution and the polynomials of the QAP and the target polynomial.

So lets see this in our example. We already know from example XXX, that 
$I=\{1,2,3,4,6,11\}$ is a solution to the R1CS XXX of our problem. To see how this translates to polyinomial divisibility we compute the polynomial $p_I$ by
\begin{align*}
p_I(x)& = (\sum_{h\in |I|} I_h\cdot a_h(x))\cdot 
(\sum_{h\in |I|} I_h\cdot b_h(x)) - 
(\sum_{h\in |I|} I_h\cdot c_h(x)) \\
= & (2(6x+10)+6(7x+4))\cdot(3(6x+10)+4(7x+4))-(6(6x+10)+11(7x+4)) \\
= & ((12x+7)+(3x+11))\cdot((5x+4)+(2x+3))-((10x+8)+(12x+5)) \\
= & (2x+5)\cdot(7x+7)-(9x) \\
= & (x^{2}+2\cdot7x+5\cdot7x+5\cdot7)-(9x) \\
= & (x^{2}+x+9x+9)-(9x) \\
= & x^{2}+x+9
\end{align*}
And as we can see in this particular example $p_I(x)$ is equal to the target polynomial $t(x)$ and hence it is divisible by $t$ with $p/t=1$.

To give a counter example we already know from XXX that $I=\{1,2,3,4,8, 2\}$ is not a solution to our R1CS. To see how this translates to polyinomial divisibility we compute the polynomial $p_I$ by
\begin{align*}
p_I(x)& = (\sum_{h\in |I|} I_h\cdot a_h(x))\cdot 
(\sum_{h\in |I|} I_h\cdot b_h(x)) - 
(\sum_{h\in |I|} I_h\cdot c_h(x)) \\
= & (2(6x+10)+6(7x+4))\cdot(3(6x+10)+4(7x+4))-(6(6x+10)+11(7x+4)) \\
= & 8x^{2}+11x+3
\end{align*}
This polynomial is not divisible by the target polynomial $t$ since
Not divisible by $t$: $(8x^{2}+11x+3)/(x^{2}+x+9) =8+\frac{3x+8}{x^{2}+x+9} $
\end{example}


\subsection{Quadratic span programs}

\section{proof system}
Now a \textit{proof system} is nothing but a game between two parties, where one parties task is to convince the other party, that a given string over some alphabet is a statement is some agreed on language. To be more precise. Such a system is more over \textit{zero knowledge} if this possible without revealing any information about the (parts of) that string.
\begin{definition}[(Interactive) Proofing System]
% https://link.springer.com/content/pdf/10.1007/BF00195207.pdf
Let $L$ be some formal language over an alphabet $\Sigma$. Then an \textbf{interactive proof system} for $L$ is a pair $(P,V)$ of two probabilistic interactive algorithms, where $P$ is called the \textbf{prover} and $V$ is called the \textbf{verifier}. 

Both algorithms are able to send messages to one another. Each algorithm only sees its own state, some shared initial state and the communication messages. 

The verifier is bounded to a number of steps which is polynomial in the size of the shared initial state, after which it stops in an accept state or in a reject state. We impose no restrictions on the local computation conducted by the prover. 

We require that, whenever the verifier is executed the following two conditions hold:
\begin{itemize}
\item (Completeness) If a string $x\in \Sigma^*$ is a member of language $L$, that is $x\in L$ and both prover and verifier follow the protocol; the verifier will accept.
\item (Soundness) If a string $x\in \Sigma^*$ is not a member of language $L$, that is $x\notin L$ and the verifier follows the protocol; the verifier will not be convinced.
\item (Zero-knowledge) If a string $x\in \Sigma^*$ is a member of language $L$, that is $x\in L$ and the prover follows the protocol; the verifier will not learn anything about $x$ but $x\in L$.
\end{itemize}
\end{definition}

In the context of zero knowledge proving systems definition XXX gets a slight adaptation:
\begin{itemize}
\item Instance: Input commonly known to both prover (P) and verifier (V), and used to support the statement of what needs to be proven. This common input may either be local to the prover-verifier interaction, or public in the sense of being known by external parties (Some scientific articles use "instance" and "statement" interchangeably, but we distinguish between the two.).
\item Witness: Private input to the prover. Others may or may not know something about the witness.
\item Relation: Specification of relationship between instances and witness. A relation can be viewed as a set of permissible pairs (instance, witness).
\item Language: Set of statements that appear as a permissible pair in the given relation.
\item Statement:Defined by instance and relation. Claims the instance has a witness in the relation(which is either true or false).
\end{itemize}

The following subsections define ways to describe checking relations that are particularly useful in the context of zero knowledge proofing systems

\subsection{Succinct NIZK}
Blum, Feldman and Micali
% Manuel  Blum,  Paul  Feldman,  and  Silvio  Micali.   Non-interactive  zero-knowledge  and  itsapplications.  InSTOC, pages 103–112, 1988.
 extended the notion tonon-interactivezero-knowledge(NIZK)  proofs in the  common  reference  string  model.  NIZK  proofs  are  useful  in  theconstruction of non-interactive cryptographic schemes, e.g., digital signatures and CCA-secure public key encryption.
 
\begin{definition} 
Let $\mathcal{R}$ be a relation generator that given a security parameter $\lambda$ in unary returns a polynomial time decidable binary relation $R$. For pairs $(i,w)\in R$ we call $i$ the instance\footnote{Note that in Groth16 this is called the statement. We think the term instance is more consistent with SOMETHING. } and $w$ the witness. We define $R_\lambda$ to be the set of possible relations $R$ the relation generator may output given $1^\lambda$. We will in the following for notational simplicity assume $\lambda$ can be deduced from the description of $R$. The relation generator may also output some side information, an auxiliary input $z$, which will be given to the adversary. An efficient prover publicly verifiable non-interactive argument for $R$ is a quadruple of probabilistic polynomial algorithms $(\textsc{Setup},\textsc{Prove},\textsc{Vfy},\textsc{Sim})$ such 
\begin{itemize}
\item Setup: $(CRS,\tau)\rightarrow Setup(R)$: The setup produces a common reference string $CRS$ and a simulation trapdoor $\tau$ for the relation $R$.
\item Proof: $\pi\rightarrow Prove(R,CRS,i,w)$: The prover algorithm takes as input a common reference string $CRS$ and a statement $(i,w)\in R$ and returns an argument $\pi$.
\item Verify: $0/1\rightarrow Vfy(R,CRS,i,\pi)$: The  verification algorithm  takes as input a common reference string $CRS$, an instance $i$ and an argument $\pi$ and returns 0 (reject) or 1 (accept).
\item $\pi\rightarrow Sim(R,\tau,i)$: The simulator takes as input a simulation trapdoor $\tau$ and instance $i$ and returns an argument $\pi$. 
\end{itemize}
\end{definition}

\subsubsection{Groth16}
Groth’s  constant  size  NIZK  argument  is  based  on  constructing  a  set  of  polynomial equations and using pairings to efficiently verify these equations. Gennaro, Gentry,Parno and Raykova [Pinocchio] found an insightful construction of polynomial equations based on Lagrange interpolation polynomials yielding a pairing-based NIZK argumentwith a common reference string size proportional to the size of the statement and wit-ness.

It constructs a snark  for arithmetic circuit satisfiability, where a proof consists of only 3 group elements. In addition to being small, the proof is also easy to verify. The verifier just needs to compute a number of exponentiations proportional to the instance size and check a single pairing product equation, which only  has  3  pairings.  

The  construction  can  be  instantiated  with  any  type  of  pairings including Type III pairings, which are the most efficient pairings. The argument has perfect completeness and perfect zero-knowledge. For soundness ?? 

In the common reference string model.

Setup: 
\begin{itemize}
\item random elements $\alpha,\beta,\gamma, \delta, s \in \mathbb{F}_{scalar}$ 
\item Common reference string $CRS_{QAP}$, specific to the $QAP$ and the choice of statement and witness $CRS_{QAP}= (CRS_{\mathbb{G}_1},CRS_{\mathbb{G}_2})$, with $n=deg(t)$: 
$$
CRS_{\mathbb{G}_{1}}=\left\{ \begin{array}{c}
[\alpha]g,[\beta]g,[\delta]g,\left\{ [s^{k}]g\right\} _{k=0}^{n-1},\left\{ [\frac{\beta a_{k}(s)+\alpha b_{k}(s)+c_{k}(s)}{\gamma}]g\right\} _{k\in I}\\
\left\{ [\frac{\beta a_{k}(s)+\alpha b_{k}(s)+c_{k}(s)}{\delta}]g\right\} _{k\in W},\left\{ [\frac{s^{k}t(s)}{\delta}]g\right\} _{k=0}^{n-2}
\end{array}\right\} 
$$
$$
CRS_{\mathbb{G}_{2}}=\left\{ [\beta]h ,[\gamma]h,[\delta]h,\left\{[s^k]h\right\} _{k=0}^{n-1}\right\} 
$$
\item Toxic waste: Must delete random elements after $CRS_{QAP}$ generation.
\end{itemize} 

\begin{example}[Generalized factorization snark]
\label{main_example_2_5}
In this example we want to compile our main example in Groth16. Input is the R1CS from example \ref{main_example_2_4}. We choose the following parameters

\begin{tabular}{ccccc}
\\
curve = BLS6-6 & $\mathbb{G}_1=$ BLS6-6(13) & $g_1 = (13,15) $
& $\mathbb{G}_2=$ & $g_2=$
\end{tabular} 

Setup phase: Recall the quadratic arithmetic program of example XXX. 

For our example we choose the following elements $\alpha=6$, $\beta=5$, $\gamma=4$, $\delta=3$, $s=2$ from $\mathbb{F}_{13}$
$$
CRS_{\mathbb{G}_{1}}=\left\{ \begin{array}{c}
[6](13,15),[5](13,15),[3](13,15),\left\{ [s^{k}](13,15)\right\} _{k=0}^{1},\left\{ [\frac{5 a_{k}(2)+6 b_{k}(2)+c_{k}(2)}{4}](13,15)\right\} _{k\in S}\\
\left\{ [\frac{5 a_{k}(2)+6 b_{k}(2)+c_{k}(2)}{3}](13,15)\right\} _{k\in W},\left\{ [\frac{s^{k}t(2)}{3}](13,15)\right\} _{k=0}^{0}
\end{array}\right\}
$$
Since we have instance indices $I=\{1, in_1,in_2\}$ and witness indices $W=\{in_3,mid_1,out_1\}$ we have 
The instance parts.
\begin{multline*}
\left[\frac{5 a_{c}(2)+6 b_{c}(2)+c_{c}(2)}{4}\right](13,15) = 
\left[\frac{5\cdot 0 +6\cdot 0 + 0 }{4}\right](13,15) =
\left[0\right](13,15) = \mathcal{O}
\end{multline*}
\begin{multline*}
\left[\frac{5 a_{in_3}(2)+6 b_{in_3}(2)+c_{in_3}(2)}{4}\right](13,15) =
\left[(5\cdot 0+6\cdot(7\cdot 2 +4)+0)\cdot 10\right](13,15) =\\
\left[(6\cdot 5 )\cdot 10\right](13,15) =
\left[1\right](13,15) =
(13,15)
\end{multline*}
\begin{multline*}
\left[\frac{5 a_{out}(2)+6 b_{out}(2)+c_{out}(2)}{4}\right](13,15) = 
\left[(5\cdot 0 +6\cdot 0 + (7\cdot 2 + 4))\cdot 10 \right](13,15) =\\
\left[5\cdot 10 \right](13,15) =
\left[11\right](13,15) = 
(33,9)
\end{multline*}

Witness part:
\begin{multline*}
\left[\frac{5 a_{in_1}(2)+6 b_{in_1}(2)+c_{in_1}(2)}{3}\right](13,15) = 
\left[(5\cdot (6\cdot 2 +10) +6\cdot 0 +0 )\cdot 9\right](13,15) = \\
\left[(5\cdot 9)\cdot 9\right](13,15) =
\left[2\right](13,15) = (33,34)
\end{multline*}
\begin{multline*}
\left[\frac{5 a_{in_2}(2)+6 b_{in_2}(2)+c_{in_2}(2)}{3}\right](13,15) = 
\left[(5\cdot 0 +6\cdot (6\cdot 2 + 10) + 0 )\cdot 9\right](13,15) = \\
\left[(6\cdot 9)\cdot 9\right](13,15) =
\left[5\right](13,15) =
(26,34)
\end{multline*}
\begin{multline*}
\left[\frac{5 a_{mid_1}(2)+6 b_{mid_1}(2)+c_{mid_1}(2)}{3}\right](13,15) = 
\left[(5\cdot (7\cdot 2 + 4) +6\cdot 0 + 0 )\cdot 9\right](13,15) = \\
\left[(5\cdot 5)\cdot 9\right](13,15) =
\left[4\right](13,15) =
(35,28)
\end{multline*}
For $\left\{\left[\frac{s^{k}t(2)}{3}\right](13,15)\right\} _{k=0}^{0}$ we get
\begin{multline*}
\left[\frac{2^{0}t(2)}{3}\right](13,15)=
[t(2)\cdot 9](13,15)= 
[(2^2+2+9)\cdot 9](13,15)= 
[5](13,15) =
(26,34)
\end{multline*}
All together, the key is:
$$
CRS_{\mathbb{G}_{1}}=\left\{ \begin{array}{c}
(27,34),(26,34),(38,15),\left\{(13,15),(33,34)\right\},
\left\{\mathcal{O}, (13,15), (33,9)\right\}\\
\left\{(33,34),(26,34),(35,28)\right\},
\left\{(26,34)\right\}
\end{array}\right\}
$$


\end{example}

