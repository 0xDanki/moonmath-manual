\chapter{Elliptic Curves}\label{chap:elliptic-curves}
\label{chap:elliptic_curves}
%references http://infosec.pusan.ac.kr/wp-content/uploads/2019/09/Pairings-For-Beginners.pdf
Generally speaking, elliptic curves are geometric objects in projective planes (see \secname{} \ref{sec:planes}) over a given field, made up of points that satisfy certain equations. One of their key features from the point of view of cryptography is that, if the underlying field is of positive characteristic, elliptic curves are finite, cyclic groups (\secname{} \ref{sec:finite-groups}). Further, it is believed that, in this case, the \concept{discrete logarithm problem} (\secname{} \ref{def:DL-secure}) on many elliptic curve groups is hard, given that the underlying characteristic is large enough.\footnote{An in-depth introduction to elliptic curves is given, for example, in \cite{silverman-1994}. An introduction from a cryptographic point of view is given in \cite{hoffstein-2008}.}

A special class of elliptic curves are so-called pairing-friendly curves, which have a notation of a group pairing (\secname{} \ref{pairing-map}) attached to them, which has cryptographically advantageous properties. 

In this chapter, we introduce elliptic curves as they are used in pairing-based approaches to the construction of SNARKs. The elliptic curves we consider are all defined over prime fields or prime field extensions, meaning that we rely heavily on the concepts and notations from \chaptname{} \ref{chap:algebra}. 

\section{\concept{short Weierstrass} Curves}
\label{sec:short_weierstrass_curve}
In this section, we introduce \term{\concept{short Weierstrass} curves}, which are the most general types of curves over finite fields of characteristics greater than $3$ (see \chaptname{} \ref{def:characteristic}), and start with their so-called \term{affine representation}.\sme{do we still need the explanation of projective planes from the previous chapter? or can they be moved to this one?}\footnote{Introducing elliptic curves in their affine representation is probably not the most common and conceptually cleanest way, but we believe that such an introduction makes elliptic curves more understandable to beginners, since an elliptic curve in the affine representation is just a set of pairs of numbers, so it is more accessible to readers unfamiliar with projective coordinates. However, the affine representation has the disadvantage that a special ``point at infinity'' that is not a point on the curve, is necessary to describe the curve's group structure.}

We then introduce the elliptic curve group law, and describe elliptic curve scalar multiplication, which is an instantiation of the exponential map of general cyclic groups \ref{scalarmultiplication}. After that, we look at the projective representation of elliptic curves, which has the advantage that no special symbol is necessary to represent the point at infinity.\footnote{As this representation is conceptually more straightforward, this is how elliptic curves are usually introduced in math classes. We believe a the major drawback from a beginner's point of view is that in the projective representation, points are elements from projective planes, which are classes of numbers.}

We finish this section with an explicit equivalence that transforms the affine representations into projective representations and vice versa.

\subsection{Affine \concept{short Weierstrass} form} Probably the least abstract and most straight-forward way to introduce elliptic curves for non-mathematicians and beginners is the so-called \term{affine representation} of a \term{\concept{short Weierstrass} curve}. To see what this is, let $\F$ be a finite field of characteristic $q$ with $q>3$, and let $a,b\in \F$ be two field elements such that the so-called \term{discriminant} $4a^3+ 27b^2$ is not equal to zero. Then a \term{\concept{short Weierstrass} elliptic curve} $E_{a,b}(\F)$ over $\F$ in its affine representation is the set of all pairs of field elements $(x,y)\in \F\times \F$ that satisfy the \concept{short Weierstrass} cubic equation $y^2=x^3+a\cdot x+b$, together with a distinguished symbol $\Oinf$, called the \term{point at infinity}:

\begin{equation}
\label{def_short_weierstrass_curve}
E_{a,b}(\F) = \{(x,y)\in \F\times \F\;|\; y^2=x^3+a\cdot x+b \} \bigcup \{\Oinf\}
\end{equation}
The term ``curve'' is used here because, if an elliptic curve is defined over a characteristic zero field, like the field $\Q$ of rational numbers, the set of all points $(x,y)\in \Q\times \Q$ that satisfy $y^2 = x^3 +a\cdot x +b$ looks like a curve. We should note, however, that visualizing elliptic curves over finite fields as ``curves'' has its limitations, and we will therefore not dwell on the geometric picture too much, but focus on the computational properties instead. To understand the visual difference, consider the following two elliptic curves: 

\medskip

% let Sage draw some elliptic curve in R^2, but show only the picture
\begin{sagesilent}
E1 = EllipticCurve([-2,1])
C1 = E1.plot()
F = GF(9973)
E2 = EllipticCurve(F, [-2,1])
C2 = E2.plot()
\end{sagesilent}
\begin{minipage}{0.48\textwidth}
\sageplot[scale=.48]{C1} 
\end{minipage}
%
\begin{minipage}{0.48\textwidth}
\label{plot:elliptic_curve}
\sageplot[scale=.5]{C2}
\end{minipage}

Both elliptic curves are defined by the same \concept{short Weierstrass} equation $y^2 = x^3-2x+1$, but the first curve is defined over the rational numbers $\Q$, that is, the pair $(x,y)$ contains rational numbers, while the second one is defined over the prime field $\F_{9973}$, which means that both coordinates $x$ and $y$ are from the prime field $\F_{9973}$. Every blue dot represents a pair $(x,y)$ that is a solution to $y^2 = x^3-2x+1$. As we can see, the second curve hardly looks like a geometric structure one would naturally call a curve. This shows that our geometric intuitions from $\Q$ are obfuscated curves over finite fields.

The equation $4a^3+ 27b^2\neq 0$ ensures that the curve is  \term{non-singular}, which loosely means that the curve has no cusps or self-intersections in the geometric sense, if seen as an actual curve. As we will see in \ref{sec:affine_group_law}, cusps and self-intersections would make the group law potentially amibugous.

Throughout this book, we have encouraged you to do as many computations in a pen-and-paper fashion as possible, as this is helps getting a deeper understanding of the details. However, when dealing with elliptic curves, computations can quickly become cumbersome and tedious, and we might get lost in the details. Fortunately, Sage is very helpful in dealing with elliptic curves. The following snippet shows a way to define elliptic curves and how to work with them in Sage:

\begin{sagecommandline}
sage: F5 = GF(5) # define the base field
sage: a = F5(2) # parameter a
sage: b = F5(4) # parameter b
sage: # check discriminant
sage: F5(6)*(F5(4)*a^3+F5(27)*b^2) != F5(0)
sage: # Short Weierstrass curve over field F5
sage: E = EllipticCurve(F5,[a,b]) # y^2 == x^3 + ax +b 
sage: # point on a curve
sage: P = E(0,2) # 2^2 == 0^3 + 2*0 + 4
sage: P.xy() # affine coordinates
sage: INF = E(0) # point at infinity
sage: try: 	# point at infinity has no affine coordinates
....:     INF.xy()
....: except ZeroDivisionError:
....:     pass
sage: P = E.plot() # create a plotted version 
\end{sagecommandline}
The following three examples give a more practical understanding of what an elliptic curve is and how we can compute it. We advise you to read these examples carefully, and ideally also do the computations yourself. We will repeatedly build on these examples in this chapter, and use \examplename{} \ref{TJJ13}  throughout the entire book.
\begin{example}\label{E1F5} Consider the prime field $\F_5$ from \examplename{} \ref{primfield_z_5}. To define an elliptic curve over $\F_5$, we have to choose two numbers $a$ and $b$ from that field. Assuming we choose $a=1$ and $b=1$, then $\kongru{4a^3+ 27b^2}{1}{5}$. This means that the corresponding elliptic curve $E_{1,1}(\F_5)$ is given by the set of all pairs $(x,y)$ from $\F_5$ that satisfy the equation $y^2=x^3+x+1$, along with the special symbol $\Oinf$, which represents the ``point at infinity''. 

To get a better understanding of this curve, observe that, if we arbitrarily choose to test the pair $(x,y)=(1,1)$, we see that $1^2 \neq 1^3+1 + 1$, and hence $(1,1)$ is not a point on the curve $E_{1,1}(\F_5)$. On the other hand, if we choose to test the pair $(x,y)=(2,1)$, we see that $1^2 = 2^3 + 2 + 1$, and hence the pair $(2,1)$ is a point on the curve $E_{1,1}(\F_5)$ (Remember that all computations are done in modulo $5$ arithmetics.)

Since the set $\F_5\times \F_5$ of all pairs $(x,y)$ from $\F_5$ only contains  $5\cdot 5=25$ pairs, we can compute the curve by just inserting every possible pair $(x,y)$ into the \concept{short Weierstrass} equation $y^2 = x^3 + x +1$. If the equation holds, the pair is a curve point. If not, that means that the point is not on the curve. Combining the result of this computation with the point at infinity gives the curve as follows:
$$
E_{1,1}(\F_5) = \{\Oinf, (0,1),(2,1),(3,1),(4,2),(4,3),(0,4),(2,4),(3,4)\}
$$
This means that the elliptic curve is a set of $9$ elements, $8$ of which are pairs of elements from $\F_5$, and one is special symbol $\Oinf$ (the point at infinity). Visualizing $E_{1,1}(\F_5)$ gives the following plot:\sme{should $\Oinf$ also be shown as a point?}
\begin{sagesilent}
F5 = GF(5)
E1 = EllipticCurve(F5, [1,1])
C1 = E1.plot()
\end{sagesilent}
\begin{center} 
\sageplot[scale=.5]{C1}
\end{center}
% sage: AffinePoints = [P.xy() for P in E1.points() if P.order > 1]
\end{example}
In the development of SNARKs, it is sometimes necessary to do elliptic curve cryptography ``in a circuit", which basically means that the elliptic curve needs to be implemented in a certain SNARK-friendly way. We will look at what this means in chapter \ref{chap:circuit-compilers}. To be able to do this efficiently, it is desirable to have curves with special properties. The following example is a pen-and-paper version of such a curve, called Tiny-\comms{jubjub}. We design this curve especially to resemble a well-known cryptographically secure curve, called \curvename{Baby-jubjub}, the latter of which is extensively used in real-world SNARKs.\sme{insert reference}\footnote{In the literature, the \term{Baby-jubjub curve} is commonly introduced as a so-called \concept{twisted Edwards} curve, which we will cover in \ref{sec:edwards}. However, as we will see in \ref{sec:edwards}, every \concept{twisted Edwards} curve is equivalent to a \concept{short Weierstrass} curve and hence we start with an introduction of Tiny-Jubjub in its \concept{short Weierstrass} incarnation.} 
The interested reader is advised to study this example carefully, as we will use it and build on it in various places throughout the book. 

\begin{example}[The \curvename{Tiny-jubjub} curve]\label{TJJ13} Consider the prime field $\F_{13}$ from \exercisename{} \ref{prime_field_F13}. If we choose $a=8$ and $b=8$, then $\kongru{4a^3+ 27b^2}{6}{13}$, and the corresponding elliptic curve is given by all pairs $(x,y)$ from $\F_{13}$ such that $y^2=x^3+8x+8$ holds. We call this curve the \term{\curvename{Tiny-jubjub}} curve (in its affine \concept{short Weierstrass} representation), or \TJJ{} for short.

Since the set $\F_{13}\times \F_{13}$ of all pairs $(x,y)$ from $\F_{13}$  only contains  $13\cdot 13=169$ pairs, we can compute the curve by just inserting every possible pair $(x,y)$ into the \concept{short Weierstrass} equation $y^2 = x^3 +8x +8$.  We get the following result:
\begin{multline}\label{eq:TJJ13-weierstrass}
%\begin{split}
\mathit{TJJ\_13} = \{\Oinf, (1, 2), (1, 11), (4, 0), (5, 2), (5, 11), (6, 5), (6, 8), (7,2), (7, 11), \\ (8, 5), (8, 8), (9, 4), (9, 9), (10, 3), (10,10), (11, 6), (11, 7), (12, 5), (12, 8)\}
%\end{split}
\end{multline}
As we can see, the curve consists of $20$ points; $19$ pairs of elements from $\F_{13}$ and the point at infinity. To get a visual impression of the \TJJ{} curve, we might plot all of its points (except the point at infinity): 
\begin{sagesilent}
F13 = GF(13)
TJJ_13 = EllipticCurve(F13, [8,8])
CTJJ_13 = TJJ_13.plot()
\end{sagesilent}
\begin{center} 
\sageplot[scale=.5]{CTJJ_13}
\end{center}
As we will see in what follows, this curve is rather special, as it is possible to represent it in two alternative forms called the \term{Montgomery} and the \term{\concept{twisted Edwards} form} (See sections \ref{sec:montgomery} and \ref{sec:edwards}, respectively).
\end{example}
Now that we have seen two pen-and-paper friendly elliptic curves, let us look at a curve that is used in actual cryptography. Cryptographically secure elliptic curves are not \hilight{qualitatively} different from the curves we looked at so far, but the prime number modulus of their prime field is much larger. Typical examples use prime numbers that have binary representations in the magnitude of more than double the size of the desired \uterm{security level}. If, for example, a security of $128$ bits is desired, a prime modulus of binary size $\geq 256$ is chosen. The following example provides such a curve. 

\begin{example}[Bitcoin's \curvename{secp256k1} curve]\label{secp256k1}
To give an example of a real-world, cryptographically secure curve, let us look at curve \curvename{secp256k1}, which is famous for being used in the public key cryptography of \href{https://bitcoin.org/}{Bitcoin}. The prime field $\F_p$ of \curvename{secp256k1} is defined by the following prime number:
$$
p = \scriptstyle 115792089237316195423570985008687907853269984665640564039457584007908834671663
$$
 
 The binary representation of this number needs $256$ bits, which implies that the prime field $\F_p$  contains approximately $2^{256}$ many elements, which is considered quite large. To get a better impression of how large the base field is: the number $2^{256}$ is approximately in the same order of magnitude as the estimated number of atoms in the observable universe. 

The curve \curvename{secp256k1} is defined by the parameters $a,b\in \F_p$ with $a=0$ and $b=7$. Since $\Zmod{4\cdot 0^3 + 27\cdot 7^2}{p}=1323$, those parameters indeed define an elliptic curve given as follows:
$$
\mathit{secp256k1} = \{(x,y)\in \F_p\times \F_p \;|\; y^2 = x^3 +7\;\} 
$$
Clearly, the \curvename{secp256k1} curve is too large to be useful in pen-and-paper computations, since it can be shown that  the number of its elements is a prime number $r$ that also has a binary representation of $256$ bits:
$$
r = \scriptstyle 11579208923731619542357098500868790785283756427907490438260516
3141518161494337
$$
Cryptographically secure elliptic curves are therefore not useful in pen-and-paper computations, but fortunately, Sage handles large curves efficiently:
\begin{sagecommandline}
sage: p = 115792089237316195423570985008687907853269984665640564039457584007908834671663
sage: # Hexadecimal representation
sage: p.str(16)
sage: p.is_prime()
sage: p.nbits()
sage: Fp = GF(p)
sage: secp256k1 = EllipticCurve(Fp,[0,7])
sage: r = secp256k1.order() # number of elements
sage: r.str(16)
sage: r.is_prime()
sage: r.nbits()
\end{sagecommandline}
%\seqsplit{115792089237316195423570985008687907853269984665640564039457584007908834671663}
\end{example}
\begin{example}[Ethereums's \curvename{alt\_bn128} curve]\label{BN128}
To give an example of a real-world, cryptographically secure curve, that we will use in our \lgname{circom} implementations, let us look at curve \curvename{alt\_bn128} as defined in \href{https://github.com/ethereum/EIPs/blob/master/EIPS/eip-197.md}{EIP-197}. This curve is used in zk-SNARK verification on the Ethereum blockchain. The prime field $\F_p$ of \curvename{alt\_bn128} is defined by the following prime number:
$$
p = \scriptstyle 21888242871839275222246405745257275088696311157297823662689037894645226208583
$$
The binary representation of this number needs $254$ bits, which implies that the prime field $\F_p$  contains approximately $2^{254}$ many elements. We write $\F_{bn128}:=F_p$ for this prime field. 

\curvename{alt\_bn128} is a Short Weierstrass curve, defined by the parameters $a,b\in \F_p$ with $a=0$ and $b=3$. Since $\Zmod{4\cdot 0^3 + 27\cdot 3^2}{p}=243$, those parameters indeed define an elliptic curve given as follows:
$$
\mathit{alt\_bn128} = \{(x,y)\in \F_p\times \F_p \;|\; y^2 = x^3 +3\;\} 
$$
The number of points on the elliptic curve \curvename{alt\_bn128} is a prime number $r$ that also has a binary representation of $254$ bits:
$$
r = \scriptstyle 21888242871839275222246405745257275088548364400416034343698204186575808495617
$$
In order to use this curve in our circom examples, we implement this Short Weierstrass elliptic curve in Sage:
\begin{sagecommandline}
sage: p = 21888242871839275222246405745257275088696311157297823662689037894645226208583
sage: p.is_prime()
sage: p.nbits()
sage: Fbn128 = GF(p)
sage: bn128 = EllipticCurve(Fbn128,[0,3])
sage: r = bn128.order() # number of elements
sage: r.is_prime()
sage: r.nbits()
\end{sagecommandline}
\end{example}

\begin{exercise} Consider the curve $E_1(\F)$ from \examplename{} \ref{E1F5} and compute the set of all curve points $(x,y)\in E_1(\F)$.
\end{exercise}
\begin{exercise} Consider the curve $TJJ\_13$ from \examplename{} \ref{TJJ13} and compute the set of all curve points $(x,y)\in TJJ\_13$.
\end{exercise}
\begin{exercise}
Look up the definition of curve \curvename{BLS12-381}, implement it in Sage, and compute the number of all curve points.
\end{exercise}
\subsubsection{Isomorphic affine \concept{short Weierstrass} curves}
\label{sec:isomorphic_curves} As explained previously in this chapter, elliptic curves are defined by pairs of parameters $(a, b)\in \F\times \F$ for some field $\F$. An important question in classifying elliptic curves is to decide which pairs of parameters $(a,b)$ and $(a',b')$ instantiate equivalent curves in the sense that there is a 1:1 correspondence between the set of curve points.

To be more precise, let $\F$ be a field, and let $(a,b)$ and $(a',b')$ be two pairs of parameters such that there is an invertible field element $c\in \F^*$ such that $a' = a\cdot c^4$ and $b' = b\cdot c^6$ hold. Then the elliptic curves $E_{a,b}(\F)$ and $E_{a',b'}(\F)$ are called \term{isomorphic}, and there is a map that maps the curve points of $E_{a,b}(\F)$ onto the curve points of $E_{a',b'}(\F)$:
\begin{equation}
\label{eq:curve_isomorphism}
I: E_{a,b}(\F) \to E_{a',b'}(\F):\; \begin{cases}(x,y)\\\Oinf\end{cases}
\mapsto \begin{cases}(c^2\cdot x,c^2\cdot y)\\\Oinf\end{cases}
\end{equation}
This map is a 1:1 correspondence, and its inverse map is given by mapping the point at infinity onto the point at infinity, and mapping each curve point $(x,y)$ onto the curve point $(c^{-2}x,c^{-2}y)$.
\begin{example}
\label{ex:isomorphic_E1F5}
 Consider the \concept{short Weierstrass} elliptic curve $E_{1,1}(\F_5)$ from \examplename{} \ref{E1F5} and the following elliptic curve:
\begin{equation}
E_{1,4}(\F_5):= \{(x,y)\in \F_5\times \F_5 \; | \; y^3 = x^3 + x + 4 \}
\end{equation}

If we insert all pairs of elements $(x,y)\in \F_5\times \F_5$ into the \concept{short Weierstrass} equation $y^3 = x^3 + x + 4$ of $E_{1,4}(\F_5)$, we get the following set of points:
\begin{equation}
E_{1,4}(\F_5) = \{
\Oinf, (0,2), (0,3), (1,1), (1,4), (2,2), (2,3), (3,2), (3,3)\}
\end{equation}

As we can see, both curves are of the same order. Since $2$ is an invertible element from $\F_5$ with $1 = 2^4\cdot 1$ and $4=2^6\cdot 1$, $E_{1,4}(\F)$ and $E_{1,1}(\F)$ are isomorphic: the map $I: E_{1,1}(\F_5)\to E_{1,4}(\F_5): (x,y)\mapsto (4x,4y)$ from \ref{eq:curve_isomorphism} defines a 1:1 correspondence. For example, the point $(4,3)\in E_{1,1}(\F)$ is mapped onto the point $I(4,3)=(4\cdot 4, 4\cdot 3) = (1,2)\in E_{1,4}(\F)$.
\end{example}
  
\begin{exercise}
Let $\F$ be a finite field, let $(a,b)$ and $(a',b')$ be two pairs of parameters, and let $c\in \F^*$ be an invertible field element such that $a' = a\cdot c^4$ and $b' = b\cdot c^6$ hold. Show that the function $I$ from \eqref{eq:curve_isomorphism} maps curve points onto curve points.
\end{exercise}

\begin{exercise}
\label{ex:isomorphic_TJJ13} Consider the \curvename{Tiny-jubjub} curve from \examplename{} \ref{TJJ13} and the elliptic curve $E_{5,12}(\F_{13})$ defined as follows:
\begin{equation}
E_{5,12}(\F_{13}) = \{ (x,y)\in \F_{13}\times \F_{13}\;|\; y^2 = x^3 + 5x +12\}
\end{equation}
Show that $TJJ\_13$ and $E_{5,12}(\F_{13})$ are isomorphic. Then compute the set of all points from $E_{5,12}(\F_{13})$, construct $I$ and map all points of $TJJ\_13$ onto $E_{5,12}(\F_{13})$.
\end{exercise}

\subsubsection{Affine compressed representation}
\label{sec:affine_point_compression}
As we have seen in \examplename{} \ref{secp256k1}, cryptographically secure elliptic curves are defined over large prime fields, where elements of those fields typically need more than $255$ bits of storage on a computer. Since elliptic curve points consist of pairs of those field elements, they need double that amount of storage.

However, we can reduce the amount of space needed to represent a curve point by using a technique called \term{point compression}. To understand this, note that, for each given $x\in\F$, there are only $2$ possible $y$s $\in \F$ such that the pair $(x,y)$ is a point on an affine \concept{short Weierstrass} curve, since $x$ and $y$ have to satisfy the equation $y^2 = x^3 + a\cdot x + b$. From this, it follows that $y$ can be computed from $x$, since it is an element from the set of square roots $\sqrt{x^3 + a\cdot x +b}$ (see \ref{ded:square_root}), which contains exactly two elements for $x^3 + a\cdot x +b\neq 0$ and exactly one element for $x^3 + a\cdot x +b=0$. 

This implies that we can represent a curve point in \term{compressed form} by simply storing the $x$ coordinate together with a single bit called the \term{sign bit}, the latter of which deterministically decides which of the two roots to choose. One convention could be to always choose the root closer to $0$ when the sign bit is $0$, and the root closer to the order of $\F$ when the sign bit is $1$. In case the $y$ coordinate is zero, both sign bits give the same result.

\begin{example}[\curvename{Tiny-jubjub}] To understand the concept of compressed curve points a bit better, consider the \TJJ{} curve from \examplename{} \ref{TJJ13} again. Since this curve is defined over the prime field $\F_{13}$, and numbers between $0$ and $13$ need approximately $4$ bits to be represented, each \TJJ{} point on this curve needs $8$ bits of storage in uncompressed form. The following set represents the uncompressed form of the points on this curve:
\begin{multline}
\mathit{TJJ\_13} = \{\Oinf, (1, 2), (1, 11), (4, 0), (5, 2), (5, 11), (6, 5), (6, 8), (7,2), (7, 11), \\ (8, 5), (8, 8), (9, 4), (9, 9), (10, 3), (10,
10), (11, 6), (11, 7), (12, 5), (12, 8)\}
\end{multline}
Using the technique of point compression, we can reduce the bits needed to represent the points on this curve to  $5$ per point. To achieve this, we can replace the $y$ coordinate in each $(x,y)$ pair by a sign bit indicating whether or not $y$ is closer to $0$ or to $13$. As a result, $y$ values in the range $[0,\ldots,6]$ will have the sign bit $0$, while $y$-values in the range $[7,\ldots,12]$ will have the sign bit $1$. Applying this to the points in \TJJ{} gives the compressed representation as follows:
\begin{multline}
\mathit{TJJ\_13} = \{\Oinf, (1, 0), (1, 1), (4, 0), (5, 0), (5, 1), (6, 0), (6, 1), (7,0), (7, 1), \\ (8, 0), (8, 1), (9, 0), (9, 1), (10, 0), (10,1), (11, 0), (11, 1), (12, 0), (12, 1)\}
\end{multline} 
Note that the numbers $7,\ldots, 12$ are the negatives (additive inverses) of the numbers $1,\ldots, 6$ in modular $13$ arithmetics, and that $-0=0$.

To recover the uncompressed counterpart of, say, the compressed point $(5,1)$, we insert the $x$ coordinate $5$ into the \concept{short Weierstrass} equation and get $y^2 = 5^3 + 8\cdot 5 +8 = 4$. As expected, $4$ is a quadratic residue in $\F_{13}$ with roots $\sqrt{4}= \{2,11\}$. Since the sign bit of the point is $1$, we have to choose the root closer to the modulus $13$, which is $11$. The uncompressed point is therefore $(5,11)$. 
\end{example}
\begin{comment}
Looking at the previous examples, the compression rate does not look very impressive. However, looking at the real-life example of the \curvename{secp256k1} curve shows that compression is has significant practical advantages.
\begin{example}
Consider the \curvename{secp256k1} curve from \examplename{} \ref{secp256k1} again. The following code uses Sage to generate a random affine curve point, then applies our compression method to it:
\begin{sagecommandline}
sage: P = secp256k1.random_point().xy()
sage: P
sage: # uncompressed affine point size
sage: ZZ(P[0]).nbits()+ZZ(P[1]).nbits()
sage: # compute the compression
sage: if P[1] > Fp(-1)/Fp(2):
....:     PARITY = 1
....: else:
....:     PARITY = 0
sage: PCOMPRESSED = [P[0],PARITY]
sage: PCOMPRESSED
sage: # compressed affine point size
sage: ZZ(PCOMPRESSED[0]).nbits()+ZZ(PCOMPRESSED[1]).nbits()
\end{sagecommandline}
\end{example}\sme{add explanation of how this shows what we claim}
\end{comment}

\subsection{\concept{Affine group law}}
\label{sec:affine_group_law}
%group law
% http://wwwmayr.informatik.tu-muenchen.de/konferenzen/Jass07/courses/1/Lukyanenko/Lukyanenk_Paper.pdf
One of the key properties of an elliptic curve is that it is possible to define a group law on the set of its points such that the point at infinity serves as the neutral element, and inverses are reflections on the $x$-axis. The origin of this law can be understood in a geometric picture and is known as the \term{chord-and-tangent rule}. In the affine representation of a \concept{short Weierstrass} curve, the rule can be described in the following way, using the symbol $\oplus$ for the group law:

\begin{definition}[\deftitle{Chord-and-tangent rule: geometric definition}]\label{def:chord-and-tangent}

\defvsep{}

\begin{itemize}
\label{def:chord-and-tangent}
\item (Point at infinity) We define the point at infinity $\Oinf$ as the neutral element of addition, that is, we define $P\oplus\Oinf = P$ for all points $P\in E(\F)$.
\item (Chord Rule) Let $P$ and $Q$ be two distinct points on an elliptic curve, neither of them the point at infinity: $P, Q\in E(\F)\textbackslash \{\Oinf\}$ and $P\neq Q$\\
The sum of $P$ and $Q$ is defined as follows:\\
Consider the line $l$ which intersects the curve in $P$ and $Q$. If $l$ intersects the elliptic curve at a third point $R'$, define the sum of $P$ and $Q$ as the reflection of $R'$ at the $x$-axis: $R=P\oplus Q$. If the line $l$ does not intersect the curve at a third point, define the sum to be the point at infinity $\Oinf$. Calling such a line a \term{chord}, it can be shown that no chord will intersect the curve in more than three points. This implies that addition is not ambiguous.
\item (Tangent Rule) Let $P$ be a point on an elliptic curve, which is not the point at infinity: $P \in E(\F)\textbackslash \{\Oinf\}$\\
 The sum of $P$ with itself (the doubling of $P$) is defined as follows:\\
Consider the line which is tangential to the elliptic curve at $P$, in the sense that it ``just touches'' the curve at that point. If this line intersects the elliptic curve at a second point $R'$, the sum $P\oplus P$ is the reflection of $R'$ at the $x$-axis. If it does not intersect the curve at a third point, define the sum to be the point at infinity $\Oinf$. Calling such a line a \term{tangent}, it can be shown that no such tangent will intersect the curve in more than two points. This implies that doubling is not ambiguous.
\end{itemize}
\end{definition}

It can be shown that the points of an elliptic curve form a commutative group with respect to the previously stated chord-and-tangent rule such that $\Oinf$ acts the neutral element, and the inverse of any element $P\in E(\F)$ is the reflection of $P$ on the $x$-axis. 

The chord-and-tangent rule defines the group law of an elliptic curve geometrically, and we just stated it informally as an intuition above. In order to apply those rules on a computer, we have to translate it into algebraic equations. To do so, first observe that, for any two given curve points $(x_1,y_1), (x_2,y_2)\in E(\F)$, the identity $x_1=x_2$ implies $y_2=\pm y_1$ as explained in \secname{} \ref{sec:affine_point_compression}. This shows that the following rules are a complete description of the elliptic curve group $(E(\F),\oplus)$:

\begin{definition}[\deftitle{Chord-and-tangent rule: algebraic definition}]\label{def:chord-tangent-algebra}
\defvsep{}
\begin{itemize}
\item (The neutral element) The point at infinity $\Oinf$ is the neutral element.
\item (The inverse element) The inverse of $\mathcal{O}$ is $\mathcal{O}$. For any other curve point $(x,y) \in E(\F)\textbackslash \{\mathcal{O}\}$, the inverse is given by $(x,-y)$.
\item (The group law) For any two curve points $P, Q \in E(\F)$, the group law is defined by one of the following cases:
\begin{enumerate}
\item (Neutral element) If $Q=\Oinf$, then the group law is defined as $P\oplus Q=P$.
\item (Inverse elements)  If $P=(x,y)$ and $Q=(x,-y)$, the group law is defined as $P\oplus Q=\Oinf$.
\item (Tangent Rule) If $P=(x,y)$ with $y\neq 0$, the group law $P\oplus P=(x',y')$ is defined as follows:
$$
\begin{array}{llr}
x' = \left(\frac{3x^2+a}{2y}\right)^2 -2x &,&
y' = \left(\frac{3x^2+a}{2y}\right)\left(x-x'\right) - y
\end{array} 
$$
\item (Chord Rule) If $P=(x_1,y_1)$ and $Q=(x_2,y_2)$ such that $x_1 \neq x_2$, the group law $R=P\oplus Q$ with $R=(x_3,y_3)$ is defined as follows:
$$
\begin{array}{llr}
x_3 = \left(\frac{y_2-y_1}{x_2-x_1}\right)^2 -x_1-x_2 &, &
y_3 = \left(\frac{y_2-y_1}{x_2-x_1} \right)\left(x_1-x_3\right) - y_1
\end{array} 
$$
\end{enumerate}
\end{itemize}
\end{definition}
\begin{notation}
Let $\F$ be a field and $E(\F)$ an elliptic curve over $\F$. We write $\oplus$ for the group law on $E(\F)$, $(E(\F),\oplus)$ for the commutative group of elliptic curve points, and use the additive notation (\notationname{} \ref{def:additive_notation}) on this group. If $P$ is a point on a \concept{short Weierstrass} curve with $P=(x,0)$ then $P$ is called \term{self-inverse}.
\end{notation}
As we can see, it is very efficient to compute inverses on elliptic curves. However, computing the addition of elliptic curve points in the affine representation needs to consider many cases, and involves extensive finite field divisions. As we will see in \ref{sec:projective_group_law}, the addition law is simplified in projective coordinates.

Let us look at some practical examples of how the group law on an elliptic curve is computed.

\begin{example}\label{ex:01+42}
Consider the elliptic curve $E_{1,1}(\F_5)$ from \examplename{} \ref{E1F5} again. As we have seen, the curve conists of the following $9$ elements:
\begin{equation}
E_{1,1}(\F_5) = \{\Oinf, (0,1),(2,1),(3,1),(4,2),(4,3),(0,4),(2,4),(3,4)\}
\end{equation}
We know that this set defines a group, so we can perform addition on any two elements from $E_{1,1}(\F_5)$ to get a third element of this group. 

To give an example, consider the elements $(0,1)$ and $(4,2)$. Neither of these elements is the neutral element $\Oinf$, and since the $x$ coordinate of $(0,1)$ is different from the $x$ coordinate of $(4,2)$, we know that we have to use the chord rule from definition \ref{def:chord-tangent-algebra} to compute the sum $(0,1)\oplus (4,2)$:
%\begin{tabular}{lr}
\begin{align*}
x_3  & = \left(\frac{y_2-y_1}{x_2-x_1}\right)^2 -x_1-x_2 & \text{\# insert points}\\
     & = \left(\frac{2-1}{4-0}\right)^2 -0-4  & \text{\# simplify in } \F_5\\
     & = \left(\frac{1}{4}\right)^2 +1
       = 4^2 +1
       = 1 +1 
       = 2
\\
\\
y_3  & = \left(\frac{y_2-y_1}{x_2-x_1} \right)\left(x_1-x_3\right) - y_1  & \text{\# insert points}\\     
     & = \left(\frac{2-1}{4-1} \right)\left(0-2\right) - 1   & \text{\# simplify in } \F_5\\    
     & = \left(\frac{1}{4} \right)\cdot 3 + 4   
       = 4\cdot 3 + 4
       = 2 + 4
       = 1          
\end{align*} 
%\end{tabular}
So, in the elliptic curve $E_{1,1}(\F_5)$, we get $(0,1)\oplus (4,2) =(2,1)$, and, indeed, the pair $(2,1)$ is an element of $E_{1,1}(\F_5)$ as expected. On the other hand, $(0,1)\oplus (0,4) =\Oinf$, since both points have equal $x$ coordinates and inverse $y$ coordinates, rendering them inverses of each other. Adding the point $(4,2)$ to itself, we have to use the tangent rule from definition \ref{def:chord-tangent-algebra}:
\begin{align*}
x'  & = \left(\frac{3x^2+a}{2y}\right)^2 -2x   & \text{\# insert points}\\
    & = \left(\frac{3\cdot 4^2+1}{2\cdot 2}\right)^2 -2\cdot 4 & \text{\# simplify in } \F_5 \\
    & = \left(\frac{3\cdot 1+1}{4}\right)^2 +3\cdot 4
      = \left(\frac{4}{4}\right)^2 +2
      = 1 +2 
      = 3
\\
\\
y'  & = \left(\frac{3x^2+a}{2y}\right)^2\left(x-x'\right) - y  & \text{\# insert points} \\
    & = \left(\frac{3\cdot 4^2+1}{2\cdot 2}\right)^2\left(4-3\right) - 2 & \text{\# simplify in } \F_5\\
    & = 1\cdot 1 + 3
      = 4
\end{align*}
So, in the elliptic curve $E_{1,1}(\F_5)$, we get the doubling  of $(4,2)$, that is, $(4,2)\oplus (4,2) =(3,4)$, and, indeed, the pair $(3,4)$ is an element of $E_{1,1}(\F_5)$ as expected. The group $E_{1,1}(\F_5)$ has no self-inverse points other than the neutral element $\Oinf$, since no point has $0$ as its $y$ coordinate. We can use Sage to double-check the computations. 
\begin{sagecommandline}
sage: F5 = GF(5)
sage: E1 = EllipticCurve(F5,[1,1])
sage: INF = E1(0) # point at infinity
sage: P1 = E1(0,1)
sage: P2 = E1(4,2)
sage: P3 = E1(0,4)
sage: R1 = E1(2,1)
sage: R2 = E1(3,4)
sage: R1 == P1+P2
sage: INF == P1+P3
sage: R2 == P2+P2
sage: R2 == 2*P2
sage: P3 == P3 + INF
\end{sagecommandline}
\end{example}
\begin{example}[\curvename{Tiny-jubjub}]\label{ex:TJJ13-self-inverse} Consider the \TJJ{} curve from \examplename{} \ref{TJJ13} again, and recall that its group of points is given as follows:
\begin{multline}
\mathit{TJJ\_13} = \{\Oinf, (1, 2), (1, 11), (4, 0), (5, 2), (5, 11), (6, 5), (6, 8), (7,2), (7, 11), \\ (8, 5), (8, 8), (9, 4), (9, 9), (10, 3), (10,
10), (11, 6), (11, 7), (12, 5), (12, 8)\}
\end{multline}
In contrast to the group from the previous example, this group contains a self-inverse point, which is different from the neutral element, defined by $(4,0)$. To see what this means, observe that we cannot add $(4,0)$ to itself using the tangent rule from \defname{} \ref{def:chord-tangent-algebra}, as the $y$ coordinate is zero. Instead, we have to use the rule for additive inverses, since $0=-0$. We get $(4,0)\oplus (4,0)=\Oinf$ in \TJJ{}, which shows that the point $(4,0)$ is  the inverse of itself, because adding it to itself results in the neutral element. We check our calculation with Sage:

\begin{sagecommandline}
sage: F13 = GF(13)
sage: TJJ = EllipticCurve(F13,[8,8])
sage: P = TJJ(4,0)
sage: INF = TJJ(0) # Point at infinity
sage: INF == P+P
sage: INF == 2*P
\end{sagecommandline}
\end{example}
\begin{example}
Consider the \curvename{secp256k1} curve from \examplename{} \ref{secp256k1} again. The following code uses Sage to generate two random affine curve points and then add these points together: 
\begin{sagecommandline}
sage: P = secp256k1.random_point()
sage: Q = secp256k1.random_point()
sage: R = P + Q
sage: P.xy()
sage: Q.xy()
sage: R.xy()
\end{sagecommandline}
\end{example}
\begin{exercise}
Consider the commutative group $(\mathit{TJJ\_13},\oplus)$ of the \curvename{Tiny-jubjub} curve from \examplename{} \ref{TJJ13}. 
\begin{enumerate}
\item Compute the inverse of $(10,10)$, $\Oinf$, $(4,0)$ and $(1,2)$.
\item Solve the equation $x \oplus (9,4) = (5,2) $ for some $x\in \mathit{TJJ\_13}$.
\end{enumerate}
\end{exercise}

\subsubsection{Scalar multiplication}
As we have seen in the previous section, elliptic curves $E(\F)$ have the structure of a commutative group associated to them. It can be shown that this group is finite and cyclic whenever the underlying field $\F$ is finite. As we know from \eqref{scalarmultiplication}, this implies that there is a notation of scalar multiplication associated to any elliptic curve over finite fields.

To understand this scalar multiplication, recall from \defname{} \ref{cyclic-groups} that every finite cyclic group of order $n$ has a generator $g$ and an associated exponential map $g^{(\cdot)}: \Z_n \to \G$, where $g^x$ is the $x$-fold product of $g$ with itself.  

Elliptic curve scalar multiplication is the exponential map written in additive notation. To be more precise, let $\F$ be a finite field, $E(\F)$ an elliptic curve of order $n$, and $P$ a generator of $E(\F)$. Then the \term{elliptic curve scalar multiplication} with base $P$ is defined as follows (where $[0]P = \Oinf$ and $[m]P = P+P+\ldots + P$ is the $m$-fold sum of $P$ with itself ):
$$
[\cdot]P: \Z_n \to E(\F)\;;\; m \mapsto [m]P
$$
Therefore, elliptic curve scalar multiplication is an instantiation of the general exponential map using additive instead of multiplicative notation.

\subsubsection{Logarithmic Ordering}
\label{def:logarithmic_ordering}
As explained in \eqref{logarithm_map}, the inverse of the exponential map exists, and it is usually called the \term{elliptic curve discrete logarithm map}. However, we don't know of any efficient way to actually compute this map, which is one reason why some elliptic curves are believed to be DL-secure (see \defname{} \ref{def:DL-secure}).

One useful property of the exponential map in regard to the examples in this book is that it can be used to greatly simplify pen-and-paper computations. As we have seen in \examplename{} \ref{ex:01+42}, computing the elliptic curve addition law takes quite a bit of effort when done without a computer. However, when $g$ is a generator of a small pen-and-paper elliptic curve group of order $n$, we can use the exponential map to write the elements of the group in the following way, which we call its \term{logarithmic order} with respect to the generator $g$:
\begin{equation}\label{def:logarithmic_order}
\G = \{[1]g\to [2]g \to [3]g\to\cdots\to [n-1]g\to \Oinf\}
\end{equation} 
For small pen-and-paper groups, the logarithmic order greatly simplifies complicated elliptic curve addition into the much simpler case of modular $n$ arithmetic. In order to add two curve points $P$ and $Q$, we only have to look up their discrete log relations with the generator $P=[l]g$ and $Q=[m]g$, and compute the group law as $P\oplus Q = [l+m]g$, where $l+m$ is addition in modular $n$ arithmetics. 

The reader should keep in mind though, that many elliptic curves are believed to be DL-secure (\defname{} \ref{def:DL-secure}), which implies that, for those curves, the logarithmic order can not be computed efficiently.  

In the following example, we look at some implications of the fact that elliptic curves are finite cyclic groups and apply the logarithmic order.
\begin{example}\label{ex:G1G2-subgroups} Consider the elliptic curve group $E_{1,1}(\F_5)$ from \examplename{} \ref{E1F5}. Since it is a finite cyclic group of order $9$, and the prime factorization of $9$ is $3\cdot 3$, we can use the fundamental theorem of finite cyclic groups (\defname{} \ref{def:fundamental_theorem_groups}) to reason about all its subgroups. In fact, since the only  factors of $9$ are $1$, $3$ and $9$, we know that $E_{1,1}(\F_5)$ has the following subgroups:
\begin{itemize}
\item $E_{1,1}(\F_5)[9]$ is a subgroup of order $9$. By definition, any group is a subgroup of itself.
\item $E_{1,1}(\F_5)[3] = \{(2,1),(2,4),\Oinf\}$ is a subgroup of order $3$. This is the subgroup associated to the prime factor $3$.
\item $E_{1,1}(\F_5)[1] = \{\Oinf\}$ is a subgroup of order $1$. This is the trivial subgroup.
\end{itemize}
Moreover, since $E_{1,1}(\F_5)$ and all its subgroups are cyclic, we know from \defname{} \ref{cyclic-groups} that they must have generators. For example, the curve point $(2,1)$ is a generator of the order $3$ subgroup $E_{1,1}(\F_5)[3]$, since every element of $E_{1,1}(\F_5)[3]$ can be generated by repeatedly adding $(2,1)$ to itself: 
\begin{align*}
[1](2,1) & = (2,1) \\
[2](2,1) & = (2,4) \\
[3](2,1) & = \Oinf
\end{align*}
Since $(2,1)$ is a generator, we know from \eqref{exponentialmap} that it gives rise to an exponential map from the finite field $\F_3$ onto $\G_2$ defined by scalar multiplication:
\begin{equation}
[\cdot](2,1): \F_3 \to E_{1,1}(\F_5)[3]\; : \; x\mapsto [x](2,1) 
\end{equation}
To give an example of a generator that generates the entire group $E_{1,1}(\F_5)$, consider the point $(0,1)$. Applying the tangent rule repeatedly, we compute as follows:
\begin{equation}
\begin{array}{lccr}
{}\begin{array}{lcl}
{}[0](0,1) &=& \Oinf \\
{}[2](0,1) &=& (4, 2) \\ 
{}[4](0,1) &=& (3, 4) \\ 
{}[6](0,1) &=& (2, 4) \\ 
{}[8](0,1) &=& (0, 4) \\ 
\end{array} & & &
{}\begin{array}{lcl}
{}[1](0,1) &=& (0, 1) \\
{}[3](0,1) &=& (2, 1) \\
{}[5](0,1) &=& (3, 1) \\
{}[7](0,1) &=& (4, 3) \\
{}[9](0,1) &=& \Oinf
\end{array}
\end{array}
\end{equation}
Again, since $(0,1)$ is a generator, we know from \eqref{exponentialmap} that it gives rise to an exponential map. However, since the group order is not a prime number, the exponential map does not map from a field, but from the ring $\Z_9$ of modular $9$ arithmetics:
\begin{equation}
[\cdot](0,1): \Z_9 \to E_{1,1}(\F_5)\; : \; x\mapsto [x](0,1) 
\end{equation}
Using the generator $(0,1)$ and its associated exponential map, we can write $E(\F_1)$ in logarithmic order with respect to $(0,1)$ as explained in \defname{} \ref{def:logarithmic_ordering}. We get the following:
\begin{equation}
E_{1,1}(\F_5) = \{(0, 1)\to (4, 2)\to (2, 1)\to (3, 4)\to (3, 1)\to (2, 4)\to (4, 3)\to (0, 4)\to \Oinf \}
\end{equation}
This indicates that the first element is a generator, and the $n$-th element is the scalar product of $n$ and the generator. To see how logarithmic orders like this simplify the computations in small elliptic curve groups, consider \examplename{} \ref{ex:01+42} again. In that example, we use the chord-and-tangent rule to compute $(0,1)\oplus (4,2)$. Now, in the logarithmic order of $E_1(\F)$, we can compute that sum much easier, since we can directly see that $(0,1)=[1](0,1)$ and $(4,2)=[2](0,1)$. We can then deduce $(0,1)\oplus (4,2)= (2,1)$ as follows:

\begin{equation}
(0,1)\oplus (4,2)=[1](0,1)\oplus [2](0,1)= [3](0,1)=(2,1)
\end{equation}
To give another example, we can immediately see that $(3,4)\oplus (4,3) = (4,2)$, without doing any expensive elliptic curve addition, since we know that $(3,4)= [4](0,1)$ and  $(4,3)= [7](0,1)$ from the logarithmic ordering of $E_{1,1}(\F_5)$. Since $4+7 = 2$ in $\Z_9$, the result must be $[2](0,1)=(4,2)$.

Finally, we can use $E_{1,1}(\F_5)$ as an example to understand the concept of cofactor clearing from \defname{} {\ref{def:cofactor_clearing}}. Since the order of $E_{1,1}(\F_5)$ is $9$, we only have a single factor, which happens to be the cofactor as well. Cofactor clearing then implies that we can map any element from $E_{1,1}(\F_5)$ onto its prime factor group $E_{1,1}(\F_5)[3]$ by scalar multiplication with $3$. For example, taking the element $(3,4)$, which is not in $E_{1,1}(\F_5)[3]$, and multiplying it with $3$, we get $[3](3,4)= (2,1)$, which is an element of $E_{1,1}(\F_5)[3]$ as expected.
\end{example}



\begin{example}\label{ex:TJJ13-cofactor-clearing} Consider the \curvename{Tiny-jubjub} curve \TJJ{} from \examplename{} \ref{TJJ13} again. In this example, we look at the subgroups of the \curvename{Tiny-jubjub} curve, define generators, and compute the logarithmic order for pen-and-paper computations. Then we take another look at the principle of cofactor clearing.

Since the order of \TJJ{} is $20$, and the prime factorization of $20$ is $2^2\cdot 5$, we know that the \TJJ{} contains a ``large'' prime-order subgroup of size $5$ and a small prime oder subgroup of size $2$. 

To compute those groups, we can apply the technique of cofactor clearing \eqref{def:cofactor_clearing}\sme{check reference} in a try-and-repeat loop. We start the loop by arbitrarily choosing an element $P\in \mathit{TJJ\_13}$, then multiplying that element with the cofactor of the group that we want to compute. If the result is $\Oinf$, we try a different element and repeat the process until the result is different from the point at infinity $\Oinf$.  

To compute a generator for the small prime-order subgroup $(\mathit{TJJ\_13})[2]$, first observe that the cofactor is $10$, since $20=2\cdot 10$. We then arbitrarily choose the curve point $(5,11)\in \mathit{TJJ\_13}$ and compute $[10](5,11)=\Oinf$. Since the result is the point at infinity, we have to try another curve point, say $(9,4)$. We get $[10](9,4)=(4,0)$ and we can deduce that $(4,0)$ is a generator of $(\mathit{TJJ\_13})[2]$. Logarithmic order then gives the following order:
\begin{equation}
(\mathit{TJJ\_13})[2] = \{(4,0)\to \Oinf\}
\end{equation}
This is expected, since we know from \examplename{} \ref{ex:TJJ13-self-inverse} that $(4,0)$ is self-inverse, with $(4,0)\oplus (4,0)=\Oinf$. We double-check the computations using Sage: 
\begin{sagecommandline}
sage: F13 = GF(13)
sage: TJJ = EllipticCurve(F13,[8,8])
sage: P = TJJ(5,11)
sage: INF = TJJ(0)
sage: 10*P == INF
sage: Q = TJJ(9,4)
sage: R = TJJ(4,0)
sage: 10*Q == R
\end{sagecommandline}
We can apply the same reasoning to the ``large'' prime-order subgroup $(\mathit{TJJ\_13})[5]$, which contains $5$ elements. To compute a generator for this group, first observe that the associated cofactor is $4$, since $20=5\cdot 4$. We choose the curve point $(9,4)\in \mathit{TJJ\_13}$ again, and compute $[4](9,4)=(7,11)$. Since the result is not the point at infinity, we know that $(7,11)$ is a generator of $(\mathit{TJJ\_13})[5]$. Using the generator $(7,11)$, we compute the exponential map $[\cdot](7,11): \F_5 \to \mathit{TJJ\_13}[5]$ and get the following:
\begin{align*}
[0](7,11) &= \Oinf\\
[1](7,11) &= (7,11)\\
[2](7,11) &= (8,5)\\
[3](7,11) &= (8,8)\\
[4](7,11) &= (7,2)
\end{align*}
We can use this computation to write the large-order prime group $(\mathit{TJJ\_13})[5]$ of the \curvename{Tiny-jubjub} curve in logarithmic order, which we will use quite frequently in what follows. We get the following:
\begin{equation}\label{eq:TJJ13-logarithmic-order}
(\mathit{TJJ\_13})[5] = \{(7,11)\to(8,5)\to(8,8)\to(7,2)\to \Oinf\}
\end{equation}
From this, we can immediately see, for example, that  $(8,8)\oplus (7,2)= (8,5)$, since 
$3+4=2$ in $\F_5$.
\end{example}
Based on the previous two examples, you  might get the impression that elliptic curve computation can be largely replaced by modular arithmetics. This however, is not true in general, but only an artifact of small groups, where it is possible to write the entire group in a logarithmic order.
\begin{exercise} Consider \examplename{} \ref{ex:G1G2-subgroups} and compute the set 
$\{[1](0,1), [2](0,1),\ldots,[8](0,1,[9](0,1)\}$ using the tangent rule only.
\end{exercise}
\begin{exercise} Consider \examplename{} \ref{ex:TJJ13-cofactor-clearing} and compute the scalarmultiplications $[10](5,11)$ as well as $[10](9,4)$ and $[4](9,4)$ with pen and paper using the algorithm from exercise \ref{alg_double-and-add}.
\end{exercise}
\subsection{Projective \concept{short Weierstrass} form}
% https://www.cosic.esat.kuleuven.be/bcrypt/lecture%20slides/wouter.pdf
As we have seen in the previous section, describing elliptic curves as pairs of points that satisfy a certain equation is relatively straight-forward. However, in order to define a group structure on the set of points, we had to add a special point at infinity to act as the neutral element. 

Recalling the definition of projective planes \ref{sec:planes},\sme{S: move that section here?} we know that points at infinity are handled as ordinary points in projective geometry. Therefore, it makes  sense to look at the definition of a \concept{short Weierstrass} curve in projective geometry.

To see what a \concept{short Weierstrass} curve in projective coordinates is, let $\F$ be a finite field of order $q$ and characteristic $>3$, let $a,b\in \F$ be two field elements such that $\Zmod{4a^3+ 27b^2}{q}\neq 0$ and let $\F\mathrm{P}^2$ be the projective plane over $\F$ as introduced in \secname{} \ref{sec:planes}. Then a \term{projective \concept{short Weierstrass} elliptic curve} over $\F$ is the set of all points $[X:Y:Z]\in \F\mathrm{P}^2$ from the projective plane that satisfy the cubic equation $Y^2\cdot Z = X^3+a\cdot X\cdot Z^2 + b\cdot Z^3$:

\begin{equation}
\label{def:projective_cubic_equation}
E(\F\mathrm{P}^2) = \{[X:Y:Z]\in \F\mathrm{P}^2\;|\; Y^2\cdot Z = X^3+a\cdot X\cdot Z^2 + b\cdot Z^3 \}
\end{equation}

To understand how the point at infinity is unified in this definition, recall from \secname{} \ref{sec:planes} that, in projective geometry, points at infinity are given by projective coordinates $[X:Y:0]$. Inserting representatives $(x_1,y_1,0)\in [X:Y:0]$ from those coordinates into the defining cubic equation \ref{def:projective_cubic_equation} results in the following identity:
\begin{align*}
y_1^2\cdot 0 & = x_1^3+a\cdot x_1\cdot 0^2 + b\cdot 0^3 & \Leftrightarrow \\
0 & = x_1^3
\end{align*} 

This implies $X=0$, and shows that the only projective point at infinity that is also a point on a projective \concept{short Weierstrass} curve is the class $[0,1,0] = \{(0,y,0)\;|\; y\in \F\}$. The point $[0:1:0]$ is the projective representation of the point at infinity $\mathcal{O}$ in the affine representation. The projective representation of a \concept{short Weierstrass} curve, therefore, has the advantage that it does not need a special symbol to represent the point at infinity from the affine definition.

\begin{example}
\label{ex:E1F5-projective}
 To get an intuition of how an elliptic curve in projective geometry looks, consider curve $E_{1,1}(\F_5)$ from \examplename{} \ref{E1F5}. We know that, in its affine representation, the set of points on the affine \concept{short Weierstrass} curve is given as follows:

\begin{equation}\label{eq:E1F5-affine}
E_{1,1}(\F_5) = \{\Oinf, (0,1),(2,1),(3,1),(4,2),(4,3),(0,4),(2,4),(3,4)\}
\end{equation}

This is defined as the set of all pairs $(x,y)\in \F_5\times \F_5$ such that the affine \concept{short Weierstrass} equation $y^2 = x^3 + ax +b$ with $a=1$ and $b=1$ is satisfied.

To find the set of elements of $E_{1,1}(\F_5)$ in the projective representation of a \concept{short Weierstrass} curve with the same parameters $a=1$ and $b=1$, we have to compute the set of projective points $[X:Y:Z]$ from the projective plane $\F_5\mathrm{P}^2$ that satisfies the following homogenous cubic equation for any representative $(x_1,y_1,z_1)\in [X:Y:Z]$:
\begin{equation}\label{eq:homogenous-cubic}
y_1^2z_1 = x_1^3 + 1\cdot x_1 z_1^2 + 1\cdot z_1^3
\end{equation}
We know from \secname{} \ref{sec:planes} that the projective plane $\F_5\mathrm{P}^2$ contains $5^2+5+1= 31$ elements, so we can insert all elements into equation \eqref{eq:homogenous-cubic} and see if both sides match.

For example, consider the projective point $[0:4:1]$. We know from \eqref{def:projective_coordinate} that this point in the projective plane represents the following line in $\F_5^3\backslash\{(0,0,0)\}$:
\begin{equation}
\label{ex:projective_coordinate_1}
[0:4:1] = \{(0,4,1),(0,3,2),(0,2,3),(0,1,4)\}
\end{equation} 

To check whether or not $[0:4:1]$ satisfies \eqref{eq:homogenous-cubic}, we can insert any representative, in other words, any element from  \eqref{ex:projective_coordinate_1}. Each element satisfies the equation if and only if all other elements satisfy the equation. As an arbitrary choice, we insert $(0,3,2)$ and get the following result:
$$
3^2\cdot 2 = 0^3 + 1\cdot 0\cdot 2^2 + 1\cdot 2^3 \Leftrightarrow
3 = 3
$$
This tells us that the affine point $[0:4:1]$ is indeed a solution to the equation \eqref{eq:homogenous-cubic}, but we could just as well have inserted any other representative of the element. For example, inserting $(0,1,4)$ also satisfies \eqref{eq:homogenous-cubic}: 
$$
1^2\cdot 4 = 0^3 + 1\cdot 0\cdot 4^2 + 1\cdot 4^3 \Leftrightarrow
4=4
$$
To find the projective representation of $E_{1,1}(\F_5)$, we first observe that the projective line at infinity $[1:0:0]$ is not a curve point on any projective \concept{short Weierstrass} curve, since it cannot satisfy the defining equation in \eqref{def:projective_cubic_equation} for any parameter $a$ and $b$. Therefore, we can exclude it from our consideration. 

Moreover, a point at infinity $[X:Y:0]$ can only satisfy the equation in \eqref{def:projective_cubic_equation} for any $a$ and $b$, if $X=0$, which implies that the only point at infinity relevant for \concept{short Weierstrass} elliptic curves is $[0:1:0]$, since $[0:k:0]= [0:1:0]$ for all $k\in\F^*$. Therefore, we can exclude all points at infinity except the point $[0:1:0]$.

All points that remain are the affine points $[X:Y:1]$. Inserting all of them into \eqref{eq:homogenous-cubic}, we get the set of all projective curve points as follows:

\begin{multline}
\label{E1F5_projective_points_set}
E_1(\F_5\mathrm{P}^2)=\{[0:1:0], [0:1:1], [2:1:1], [3:1:1], \\ [4:2:1], [4:3:1], [0:4:1], [2:4:1], [3:4:1]\}
\end{multline}

If we compare this with the affine representation, we see that there is a 1:1 correspondence between the points in the affine representation in \eqref{eq:E1F5-affine} and the affine points in projective geometry, and that the projective point $[0:1:0]$ represents the additional point $\Oinf$ in the affine representation.
\end{example} 

\begin{exercise}
Consider \examplename{} \ref{ex:E1F5-projective} and compute the set \eqref{E1F5_projective_points_set} by inserting all points from the projective plane $\F_5\mathrm{P}^2$ into the defining projective \concept{short Weierstrass} equation.
\end{exercise}

\begin{exercise}
Compute the projective representation of the \curvename{Tiny-jubjub} curve (\examplename{} \ref{TJJ13}) and the logarithmic order of its large prime-order subgroup with respect to the generator $[7:11:1]$ in projective coordinates.
\end{exercise}

\subsubsection{Projective Group law}
\label{sec:projective_group_law}
As we saw in section \ref{sec:affine_group_law}, one of the key properties of an elliptic curve is that it comes with a definition of a group law on the set of its points, described geometrically by the chord-and-tangent rule (\defname{} \ref{def:chord-and-tangent}). This rule was fairly intuitive, with the exception of the distinguished point at infinity, which appeared whenever the chord or the tangent did not have a third intersection point with the curve.

One of the key features of projective coordinates is that, in projective space, it is guaranteed that any chord will always intersect the curve in three points, and any tangent will intersect it in two points. So, the geometric picture simplifies, as we don't need to consider external symbols and associated cases. The price to pay for this mathematical simplification is that projective coordinates might be less intuitive for beginners.

It can be shown that the points of an elliptic curve in projective space form a commutative group with respect to the tangent-and-chord rule such that the projective point $[0:1:0]$ is the neutral element, and the additive inverse of a point $[X:Y:Z]$ is given by $[X:-Y:Z]$. The addition law is usually described by  \algname{} \ref{alg:projective_group_law}, minimizing the number of necessary additions and multiplications in the base field. % https://www.hyperelliptic.org/EFD/precomp.pdf

\begin{algorithm}\caption{Projective \concept{short Weierstrass} Addition Law}
\label{alg:projective_group_law}
% https://en.wikibooks.org/wiki/Cryptography/Prime_Curve/Standard_Projective_Coordinates
\begin{algorithmic}[0]
\Require $[X_1:Y_1:Z_1],[X_2:Y_2:Z_2] \in E(\F\mathbb{P}^2)$
\Procedure{Add-Rule}{$[X_1:Y_1:Z_1],[X_2:Y_2:Z_2]$}
\If{$[X_1:Y_1:Z_1] == [0:1:0]$}
  \State $[X_3:Y_3:Z_3] \gets [X_2:Y_2:Z_2]$
\ElsIf{$[X_2:Y_2:Z_2] == [0:1:0]$}
  \State $[X_3:Y_3:Z_3] \gets [X_1:Y_1:Z_1]$
\Else
  \State $U_1 \gets Y_2\cdot Z_1$
  \State $U_2 \gets Y_1\cdot Z_2$
  \State $V_1 \gets X_2\cdot Z_1$
  \State $V_2 \gets X_1\cdot Z_2$
  \If{$V_1 == V_2$}
    \If{$U_1 \neq U_2$}
      $[X_3:Y_3:Z_3] \gets [0:1:0]$
    \Else
      \If{$Y_1 == 0$}
        $[X_3:Y_3:Z_3] \gets [0:1:0]$
      \Else
        \State $W \gets a\cdot Z_1^2 + 3\cdot X_1^2$
        \State $S \gets Y_1\cdot Z_1$
        \State $B \gets X_1\cdot Y_1\cdot S$
        \State $H \gets W^2 - 8\cdot B$
        \State $X' \gets 2\cdot H\cdot S$
        \State $Y' \gets W\cdot (4\cdot B - H) - 8\cdot Y_1^2\cdot S^2$
        \State $Z' \gets 8\cdot S^3$
        \State $[X_3:Y_3:Z_3] \gets [X':Y':Z']$
      \EndIf
    \EndIf
  \Else
    \State $U = U_1 - U_2$
    \State $V = V_1 - V_2$
    \State $W = Z_1\cdot Z_2$
    \State $A = U^2\cdot W - V^3 - 2\cdot V^2\cdot V_2$
    \State $X' = V\cdot A$
    \State $Y' = U\cdot(V^2\cdot V_2 - A) - V^3\cdot U_2$
    \State $Z' = V^3\cdot W$
    \State $[X_3:Y_3:Z_3]\gets [X':Y':Z']$
  \EndIf
\EndIf
\State \textbf{return} $[X_3:Y_3:Z_3]$
\EndProcedure
\Ensure $ [X_3:Y_3:Z_3] == [X_1:Y_1:Z_1] \oplus [X_2:Y_2:Z_2]$
\end{algorithmic}
\end{algorithm}
%\begin{example}[Polynomial evaluation on secret points]
%Since scalar multiplication is assumed to be a one way function, it can be used to encrypt computations. For example it can be used to proof identities of bounded degree polynomials (with some probability), without actually revealing the polynomials. To see what this means, consider the moon-jubjub curve $TJJ(\F_{13})$ from XXX\sme{add reference} and the set $\F_{13}[x]_{\leq 2}$ of all polynomials with coefficients in $\F_{13}$ and maximum degree $2$.

%Now assume that there are two parties $A$ and $B$ such that $A$ choose polynomial $P_A$ and $B$ chooses polynomial $P_B$ from $\F_{13}[x]_{\leq 2}$. The task is to check (with some probability) weather or not $P_A$ equals $P_B$ without actually revealing any information about the polynomials. 

%This task can be solved, by evaluating the polynomials at a secret point in the exponent of a (DFHM-PROPERTY) group and then compare the results.

%So we assume that there is some trusted third party $C$ that chooses a publicly known generator of a large prime-order subgroup of $TJJ(\F_{13})$, say  a secrete point $s\in\F_{13}$, say $s=2$. $C$ then c
%\end{example}
\begin{exercise} Consider \examplename{} \ref{ex:E1F5-projective} again. Compute the following expression for projective points on $E_1(\F_5\mathrm{P}^2)$ using \algname{} \ref{alg:projective_group_law}:
\begin{itemize}
\item $[0:1:0]\oplus [4:3:1]$
\item $[0:3:0]\oplus [3:1:2]$
\item $-[0:4:1]\oplus [3:4:1]$
\item $[4:3:1]\oplus [4:2:1]$
\end{itemize}
and then solve the equation $[X:Y:Z] \oplus [0:1:1]= [2:4:1]$ for some point $[X:Y:Z]$ from the projective \concept{short Weierstrass} curve $E_1(\F_5\mathrm{P}^2)$.
\end{exercise}

\begin{exercise}
Compare the affine addition law for \concept{short Weierstrass} curves with the projective addition rule. Which branch in the projective rule corresponds to which case in the affine law? 
\end{exercise}

\subsubsection{Coordinate Transformations} As we can see by comparing the examples 
\ref{ex:E1F5-projective} and \ref{ex:E1F5-projective},\sme{same example twice} there is a close relation between the affine and the projective representation of a \concept{short Weierstrass} curve. This is not a coincidence. In fact, from a mathematical point of view, projective and affine \concept{short Weierstrass} curves describe the same thing, as there is a one-to-one correspondence (an isomorphism) between both representations for any parameters $a$ and $b$. 

To specify the correspondence, let $E(\F)$ and $E(\F\mathrm{P}^2)$ be an affine and a projective \concept{short Weierstrass} curve defined for the same parameters $a$ and $b$. Then, the function in \eqref{eq:weierstrass-isomorphism-map} maps points from the affine representation to points from the projective representation of a \concept{short Weierstrass} curve. In other words, if the pair of field elements $(x,y)$ satisfies the affine \concept{short Weierstrass} equation $y^2= x^3 + ax + b$, then all homogeneous coordinates $(x_1,y_1,z_1)\in [x:y:1]$ satisfy the projective \concept{short Weierstrass} equation $y_1^2\cdot z_1= x_1^3 + ay_1\cdot z_1^2 + b\cdot z_1^3$. 

\begin{equation}\label{eq:weierstrass-isomorphism-map}
I : E(\F) \to E(\F\mathrm{P}^2)\;:\;
\begin{array}{lcl}
(x,y)       &\mapsto & [x:y:1]\\
\mathcal{O} &\mapsto & [0:1:0]
\end{array}
\end{equation}
This map is a $1:1$ correspondence, which means that it maps exactly one point from the affine representation onto one point from the projective representation. It is therefore possible to invert this map in order to map points from the projective representation to points from the affine representation of a \concept{short Weierstrass} curve. The inverse is given by the following map:
\begin{equation}
I^{-1} : E(\F\mathbb{P}^2)\to E(\F) \;:\; [X:Y:Z] \mapsto \begin{cases}
(\frac{X}{Z},\frac{Y}{Z}) & \text{ if } Z\neq 0\\
\mathcal{O} & \text{ if } Z=0
\end{cases}
\end{equation}
Note that the only projective point $[X:Y:Z]$ with $Z = 0$ that satisfies the equation in \ref{def:projective_cubic_equation} is given by the class $[0:1:0]$. A key feature of $I$ and its inverse is that both maps respect the group structure, which means that the neutral element is mapped to the neutral element $I(\mathcal{O})=[0:1:0]$, and that  $I((x_1,y_1)\oplus (x_2,y_2))$ is equal to $I(x_1,y_1)\oplus I(x_2,y_2)$. The same holds true for the inverse map $I^{-1}$.

Maps with these properties are called \term{group isomorphisms}, and, from a mathematical point of view, the existence of function $I$ implies that the affine and the projective definition of \concept{short Weierstrass} elliptic curves are equivalent, and represent the same mathematical thing in just two different views. Implementations can therefore choose freely between these two representations. 


\section{Montgomery Curves}\label{sec:montgomery}
% https://eprint.iacr.org/2017/212.pdf
Affine and the projective \concept{short Weierstrass} forms are the most general ways to describe elliptic curves over fields of characteristics larger than $3$. However, in certain situations, it might be advantageous to consider more specialized representations of elliptic curves, in order to get faster algorithms for the group law or the scalar multiplication, for example. 

As we will see in this section, so-called \term{Montgomery curves} are a subset of all elliptic curves that can be written in the \term{Montgomery form}. Those curves allow for constant time algorithms for (specializations of) the elliptic curve scalar multiplication. 

To see what a Montgomery curve in its affine representation is, let $\F$ be a prime field of order $p>3$, and let $A,B\in \F$ be two field elements such that $B\neq 0$ and $A^2 \neq \Zmod{4}{p}$.  A \term{Montgomery elliptic curve} $M(\F)$ over $\F$ in its affine representation is the set of all pairs of field elements $(x,y)\in \F\times \F$ that satisfy the \term{Montgomery cubic equation} $B\cdot y^2 = x^3 + A\cdot x^2 + x$, together with a distinguished symbol $\Oinf$, called the \term{point at infinity}.

\begin{equation}
\label{eq:montgomery-curve}
M(\F) = \{(x,y)\in \F\times \F\;|\; B\cdot y^2 = x^3 + A\cdot x^2 + x  \} \bigcup \{\Oinf\}
\end{equation}

Despite the fact that Montgomery curves look different from \concept{short Weierstrass} curves, they are just a special way to describe certain \concept{short Weierstrass} curves. In fact, every curve in affine Montgomery form can be transformed into an elliptic curve in \concept{short Weierstrass} form. To see that, assume that a curve is given in Montgomery form $B y^2 = x^3 + A x^2 + x$. The associated \concept{short Weierstrass} form is then defined as follows:

\begin{equation}\label{eq:montgomery-to-weierstrass}
y^2 = x^3 + \frac{3-A^2}{3\cdot B^2}\cdot x + \frac{2\cdot A^3-(\Zmod{9}{p})\cdot A}{(\Zmod{27}{p})\cdot B^3}
\end{equation}

On the other hand, not every elliptic curve $E(\F)$ over a prime field $\F$ of characteristic $p>3$ given in \concept{short Weierstrass} form $y^2 = x^3 + a x + b$ can be converted into Montgomery form. For a \concept{short Weierstrass} curve to be a Montgomery curve, the following conditions need to hold:

\begin{definition}\label{def:montgomery}
\defvsep{}

\begin{itemize}
\item The number of points on $E(\F)$ is divisible by $4$.
\item The polynomial $z^3 + a z + b \in \F[z]$ has at least one root $z_0\in\F$.
\item $3z_0^2 + a$ is a quadratic residue in $\F^*$.
\end{itemize}
\end{definition}

When these conditions are satisfied, then for $s=({\sqrt{3z_0^{2}+a}})^{-1}$, a Montgomery curve is defined by the following equation:
\begin{equation}\label{eq:montgomery-form}
sy^{2}=x^{3}+(3z_0 s)x^{2}+x
\end{equation}

In the following example, we will look at the \curvename{Tiny-jubjub} curve again, and show that it is actually a Montgomery curve.
\begin{example}\label{TJJ13-montgomery}
Consider the prime field $\F_{13}$ and the \curvename{Tiny-jubjub} curve \TJJ{} from \examplename{} \ref{TJJ13}. To see that it is a Montgomery curve, we have to check the requirements from \defname{} \ref{def:montgomery}: 

Since the order  of \TJJ{} is $20$, which is divisible by $4$, the first requirement is met.

As for the second criterion, since $a=8$ and $b=8$, we have to check that the polynomial $P(z) = z^3 + 8z + 8$ has a root in $\F_{13}$. To see this, we simply evaluate $P$ at all numbers $z\in \F_{13}$, and find that $P(4)=0$, so a root is given by $z_0=4$. 

In the last step, we have to check that $3\cdot z_0^2 + a$ has a root in $\F_{13}^*$. We compute as follows:
\begin{align*}
3z_0^2 + a & = 3\cdot 4^2 + 8 \\
           & = 3 \cdot 3 + 8 \\
           & = 9 + 8 \\
           & = 4
\end{align*}

To see that $4$ is a quadratic residue in $\F_{13}$, we use Euler's criterion (\ref{eq: Euler_criterion}) to compute the Legendre symbol of $4$. We get the following:

$$
\left(\frac{4}{13}\right) = 4^{\frac{13-1}{2}} = 4^6 = 1
$$ 
This means that $4$ does have a root in $\F_{13}$. In fact, computing a root of $4$ in $\F_{13}$ is easy, since the integer root $2$ of $4$ is also one of its roots in $\F_{13}$. The other root is given by $13-4=9$.

Since all requirements are met, we have shown that \TJJ{} is indeed a Montgomery curve, and we can use \eqref{eq:montgomery-form} to compute its associated Montgomery form as follows:
\begin{align*}
s & = \left(\sqrt{3\cdot z_0^2 +8}\right)^{-1} \\
  & = 2^{-1} & \text{\# Fermat's little theorem} \\
  & = 2^{13-2} & \text{\# }\Zmod{2048}{13} = 7\\
  & = 7
\end{align*}

The defining equation for the Montgomery form of the \curvename{Tiny-jubjub} curve is  given by as follows:
\begin{align*}
sy^{2} & =x^{3}+(3z_0 s)x^{2}+x  & \Rightarrow\\
7\cdot y^{2} & =x^{3}+(3\cdot 4 \cdot 7)x^{2}+x &\Leftrightarrow\\
7\cdot y^{2} & =x^{3}+6x^{2}+x
\end{align*}
So, we get the defining parameters as $B= 7$ and $A=6$, and we can write the \curvename{Tiny-jubjub} curve in its affine Montgomery representation as follows:
\begin{equation}\label{eq:TJJ13-montgomery-representation}
\mathit{TJJ\_13} = \{(x,y)\in \F_{13}\times \F_{13}\;|\; 7\cdot y^{2} =x^{3}+6x^{2}+x \}\bigcup \{\Oinf\}
\end{equation}

Now that we have the abstract definition of the \curvename{Tiny-jubjub} curve in Montgomery form, we can compute the set of points by inserting all pairs $(x,y)\in\F_{13}\times \F_{13}$, similarly to how we computed the curve points in its \concept{short Weierstrass} representation \eqref{eq:TJJ13-weierstrass}\sme{check reference}. We get the following:
\begin{multline}
\label{tjj_13_montgomery_points_set}
\mathit{M\_TJJ\_13} = \{\Oinf, (0, 0),(1, 4),(1, 9),(2, 4),(2, 9),(3, 5),(3, 8),(4, 4),(4, 9),\\ (5, 1),(5, 12),(7, 1),(7, 12),(8, 1),(8, 12),(9, 2),(9, 11),(10, 3),(10, 10)\}
\end{multline}

We can check the results with Sage:

\begin{sagecommandline}
sage: F13 = GF(13)
sage: L_MTJJ = []
....: for x in F13:
....:     for y in F13:
....:         if F13(7)*y^2 == x^3 + F13(6)*x^2 +x:
....:             L_MTJJ.append((x,y))
sage: MTJJ = Set(L_MTJJ)
sage: # does not compute the point at infinity
\end{sagecommandline}
\end{example}
\begin{exercise}
Consider \examplename{} \ref{TJJ13-montgomery} and compute the set in \eqref{tjj_13_montgomery_points_set} by inserting every pair of field elements $(x,y)\in \F_{13}\times \F_{13}$ into the defining Montgomery equation.
\end{exercise}
\begin{exercise}
Consider the elliptic curve $E_1(\F)$ from \examplename{} \ref{E1F5} and show that $E_1(\F)$ is not a Montgomery curve.
\end{exercise}
\begin{exercise}
Consider the elliptic curve \curvename{secp256k1} from \examplename{} \ref{secp256k1} and show that \curvename{secp256k1} is not a Montgomery curve.
\end{exercise}

\subsection{Affine Montgomery coordinate transformation} Comparing the Montgomery representation in \eqref{eq:TJJ13-montgomery-representation} with the \concept{short Weierstrass} representation of the same curve in \eqref{eq:TJJ13-weierstrass}, we see that there is a 1:1 correspondence between the curve points in these two examples. This is no accident. In fact, if $M_{A,B}$ is a Montgomery curve, and $E_{a,b}$ a \concept{short Weierstrass} curve with $a = \frac{3-A^2}{3B^2}$ and $b= \frac{2A^2 -9A}{27B^3}$, then the following function maps all points in Montgomery representation onto the points in \concept{short Weierstrass} representation:
\begin{equation}
I: M_{A,B} \to E_{a,b}\; : \; (x,y) \mapsto \left(\frac{3x + A}{3B}, \frac{y}{B}\right)
\end{equation}
The point at infinity of the Montgomery form is mapped to the point at infinity of the \concept{short Weierstrass} form. This map is a 1:1 correspondence (an isomorphism), and its inverse map is given by the following equation (where $z_0$ is a root of the polynomial $z^3 + a z + b \in \F[z]$ and $s=({\sqrt{3z_0^{2}+a}})^{-1}$).
\begin{equation}
I^{-1}: E_{a,b} \to M_{A,B}\; : \; (x,y) \mapsto \left(s\cdot(x-z_0), s\cdot y\right)
\end{equation}
The point at infinity of the \concept{short Weierstrass} form is mapped to the point at infinity of the Montgomery form. Using this map, it is therefore possible for implementations of Montgomery curves to freely transit between the \concept{short Weierstrass} and the Montgomery form. 
 
\begin{example} Consider the \curvename{Tiny-jubjub} curve again. In \ref{eq:TJJ13-weierstrass} we defined its \concept{short Weierstrass} representation and in \examplename{} \ref{eq:TJJ13-montgomery-representation}, we derived its Montgomery representation. 

To see how the coordinate transformation $I$ works in this example, let's map points from the Montgomery representation onto points from the \concept{short Weierstrass} representation. Inserting, for example, the point $(0,0)$ from the Montgomery representation \eqref{eq:TJJ13-montgomery-representation} into $I$ gives the following:
\begin{align*}
I(0,0) & = \left(\frac{3\cdot 0 + A}{3B}, \frac{0}{B}\right) \\
          & = \left(\frac{3\cdot 0 + 6}{3\cdot 7}, \frac{0}{7}\right) \\
          & = \left(\frac{6}{8}, 0\right) \\
          & = \left(4, 0\right) \\
\end{align*}

As we can see, the Montgomery point $(0,0)$ maps to the self-inverse point $(4,0)$ of the \concept{short Weierstrass} representation. On the other hand, we can use our computations of $s=7$ and $z_0=4$ from \examplename{} \ref{TJJ13-montgomery} to compute the inverse map $I^{-1}$, which maps points on the \concept{short Weierstrass} representation to points on the Mongomery form. Inserting, for example, $(4,0)$ we get the following:
\begin{align*}
I^{-1}(4,0) & = \left(s\cdot(4-z_0), s\cdot 0\right)\\
               & = \left(7\cdot(4-4), 0\right)\\
               & = (0,0)
\end{align*}

As expected, the inverse map maps the \concept{short Weierstrass} point back to where it originated in the Montgomery form. We can use Sage to check that our computation of $I$ is correct:
\begin{sagecommandline}
sage: # Compute I of Montgomery form:
sage: L_I_MTJJ = []
sage: for (x,y) in L_MTJJ: # LMTJJ as defined previously                                   
....:     v = (F13(3)*x + F13(6))/(F13(3)*F13(7))
....:     w = y/F13(7)
....:     L_I_MTJJ.append((v,w))
sage: I_MTJJ = Set(L_I_MTJJ)
sage: # Computation \concept{short Weierstrass} form
sage: C_WTJJ = EllipticCurve(F13,[8,8]) 
sage: L_WTJJ = [P.xy() for P in C_WTJJ.points() if P.order() > 1]
sage: WTJJ = Set(L_WTJJ)
sage: # check I(Montgomery) == Weierstrass
sage: WTJJ == I_MTJJ
sage: # check the inverse map I^(-1)
sage: L_IINV_WTJJ = []
sage: for (v,w) in L_WTJJ:
....:     x = F13(7)*(v-F13(4))
....:     y = F13(7)*w
....:     L_IINV_WTJJ.append((x,y))
sage: IINV_WTJJ = Set(L_IINV_WTJJ)
sage: MTJJ == IINV_WTJJ
\end{sagecommandline}
\end{example}

\subsection{Montgomery group law} We have seen that Montgomery curves are special cases of \concept{short Weierstrass} curves. As such, they have a group structure defined on the set of their points, which can also be derived from the chord-and-tangent rule. In accordance with \concept{short Weierstrass} curves, it can be shown that the identity $x_1=x_2$ implies $y_2=\pm y_1$, meaning that the following rules are a complete description of the elliptic curve group law:

\begin{definition}[\deftitle{Montgomery group law}]\label{def:montgomery-group-law}
\defvsep{}

\begin{itemize}
\item (The neutral element) The point at infinity $\Oinf$ is the neutral element.
\item (The inverse element) The inverse of $\mathcal{O}$ is $\mathcal{O}$. For any other curve point $(x,y) \in M(\F_q)\textbackslash \{\mathcal{O}\}$, the inverse is given by $(x,-y)$.
\item (The group law) For any two curve points $P, Q \in M(\F_q)$, the group law is defined by one of the following cases:
\begin{enumerate}
\item (Neutral element) If $Q=\Oinf$, then the sum is defined as $P\oplus Q=P$.
\item (Inverse elements)  If $P=(x,y)$ and $Q=(x,-y)$, the group law is defined as $P\oplus Q=\Oinf$.
\item (Tangent rule) If $P=(x,y)$ with $y\neq 0$, the group law $P\oplus P=(x',y')$ is defined as follows:
$$
\begin{array}{llr}
x' = (\frac{3x_1^2 + 2A x_1 +1}{2By_1})^2\cdot B - (x_1 + x_2) - A &,&
y' = \frac{3x_1^2 + 2A x_1 +1}{2By_1}(x_1-x') - y_1
\end{array} 
$$
\item (Chord rule) If $P=(x_1,y_1)$ and $Q=(x_2,y_2)$ such that $x_1 \neq x_2$, the group law $R=P\oplus Q$ with $R=(x_3,y_3)$ is defined as follows:
$$
\begin{array}{llr}
x' = (\frac{y_2-y_1}{x_2-x_1})^2B - (x_1 + x_2) - A &, &
y' = \frac{y_2-y_1}{x_2-x_1}(x_1-x') - y_1
\end{array} 
$$
\end{enumerate}
\end{itemize}
\end{definition}
\begin{exercise}
Consider the commutative group $(\mathit{M\_TJJ\_13},\oplus)$ of the \curvename{Tiny-jubjub} curve in its Montgomery form from \examplename{} \eqref{tjj_13_montgomery_points_set}. 
\begin{enumerate}
\item Compute the inverse of $(1,9)$, $\Oinf$, $(7,12)$ and $(4,9)$.
\item Solve the equation $x \oplus (3,8) = (10,3) $ for some $x\in \mathit{M\_TJJ\_13}$.
\end{enumerate}
Choose some element $x\in \mathit{M\_TJJ\_13}$ and test if $x$ is a generator of $\mathit{M\_TJJ\_13}$. If $x$ is not a generator, repeat until you find some generator $x$. Write $\mathit{M\_TJJ\_13}$ in logarithmic order with respect to $x$.
\end{exercise}
\begin{exercise}
Consider the curve \curvename{alt\_bn128} from example \ref{BN128}. Show that this curve is not a Montgomery curve.
\end{exercise}
%\subsubsection{Projective Montgomery Form}
%As with more general curves in \concept{short Weierstrass} form, we can look at Montgomery curves in projective space. To see how such a curve looks in projective coordinates is, let $\F_{q}$ be a finite field and $A,B\in \F_q$ two field elements such that $B\neq 0$ and $A^2\neq 4$. Then a \term{Montgomery elliptic curve} $E/\F_q$ over $\F_q$ in its projective representation is the set
%\begin{equation}
%\label{def_short_weierstrass_curve}
%E/\F_q\mathbb{P}^2 = \{[X:Y:Z]\in \F_q\mathbb{P}^2\;|\; B\cdot Y^2 \cdot Z = X^3 + A\cdot X^2\cdot Z + X\cdot Z^2  \}
%\end{equation}
%of all points $[X:Y:Z]\in \F_q\mathbb{P}^2$ from the projective plane that satisfy the \term{homogenous} cubic equation $B\cdot Y^2 \cdot Z = X^3 + A\cdot X^2\cdot Z + X\cdot Z^2$.

% TODO: https://maths-people.anu.edu.au/~brent/pd/Subramanya-thesis.pdf
% x-coordinate only aka differential arithmetics on Montgomery curves
\section{Twisted Edwards Curves}\label{sec:edwards}
As we have seen in \defname{} \ref{def:chord-tangent-algebra} and \defname{} \ref{def:montgomery-group-law}, both \concept{short Weierstrass} and Montgomery curves have somewhat complicated group laws, as many cases have to be distinguished. This can add complexity to a programming implementation, because each case translates to another branches in a computer program. However, in the context of SNARK development, computational models for bounded computations are used in which program branches are undesirably costly (more on this in \secname{} \ref{sec:R1CS} and \ref{sec:circuits}) .  To make elliptic curves ``SNARK-friendly'', it is therefore advantageous to look for curves with a group law that requires no branches and utilizes as few field operations as possible.

So-called \term{SNARK-friendly \concept{twisted Edwards} curves} are particularly useful here, as these curves have a compact and easily implementable group law that works for all points including the point at infinity. Implementing this law needs no branching. 

% https://eprint.iacr.org/2008/013.pdf
To see what a \term{\concept{twisted Edwards} curve} in its affine form looks like, let $\F$ be a finite field of characteristic $>3$, and let $a,d\in \F\backslash\{0\}$ be two non-zero field elements such that $a\neq d$.  A \term{\concept{twisted Edwards} elliptic curve} in its affine representation is  the set of all pairs $(x,y)$ from $\F\times \F$ that satisfy the \concept{twisted Edwards} equation $a\cdot x^2+y^2= 1+d\cdot x^2y^2$:
\begin{equation}\label{eq:twisted-edwards}
E(\F)=\{(x,y)\in\F\times\F\;|\; a\cdot x^2+y^2= 1+d\cdot x^2y^2\}
\end{equation} 
A \concept{twisted Edwards} curve is called a \term{SNARK-friendly \concept{twisted Edwards} curve} if the parameter $a$ is a quadratic residue and the parameter $d$ is a quadratic non-residue.

As we can see from the definition, affine \concept{twisted Edwards} curves look somewhat different from \concept{short Weierstrass} curves, as their affine representation does not need a special symbol to represent the point at infinity. In fact, the pair $(0,1)$ is always a point on any \concept{twisted Edwards} curve, and it takes the role of the point at infinity.

Despite their different appearances, \concept{twisted Edwards} curves are equivalent to Montgomery curves in the sense that, for every \concept{twisted Edwards} curve, there is a Montgomery curve, and a way to map the points of one curve onto the other (and vice versa) in a 1:1 correspondence. 

To see that, assume that a curve in \concept{twisted Edwards} form is given. The associated Montgomery curve is then defined by the Montgomery equation:
\begin{equation}
\frac{4}{a-d} y^2 = x^3 + \frac{2(a+d)}{a-d}\cdot x^2 + x 
\end{equation}

On the other hand, a Montgomery curve $By^{2}=x^{3}+Ax^{2}+x$ such that $B\neq 0$ and $A^2\neq 4$ gives rise to a \concept{twisted Edwards} curve defined by the following equation:
\begin{equation}\label{eq:montgomery-to-twisted-edwards}
(\frac{A+2}{B})x^2+y^2= 1+(\frac{A-2}{B})x^2y^2
\end{equation}

\begin{example}Consider the \curvename{Tiny-jubjub} curve from \examplename{} \ref{TJJ13} again. We know from \examplename{} \ref{TJJ13-montgomery} that it is a Montgomery curve, and, since Montgomery curves are equivalent to \concept{twisted Edwards} curves, we want to write this curve in \concept{twisted Edwards} form. We use equation \eqref{eq:montgomery-to-twisted-edwards}, and compute the parameters $a$ and $d$ as follows:
\begin{align*}
a & = \frac{A+2}{B} & \text{\# insert A=6 and B=7}\\
  & = \frac{8}{7} = 3 & \text{\# } 7^{-1}= 2 \\
  \\
d & = \frac{A-2}{B} \\
  & = \frac{4}{7} = 8 
\end{align*}

Thus, we get the defining parameters  $a= 3$ and $d=8$. 

Since our goal is to use this curve later in implementations of pen-and-paper SNARKs, let us show that \curvename{Tiny-jubjub} is also a SNARK-friendly \concept{twisted Edwards} curve. To see that, we  have to show that $a$ is a quadratic residue and $d$ is a quadratic non-residue. We therefore compute the Legendre symbols of $a$ and $d$ using Euler's criterion. We get the following:
\begin{align*}
\left(\frac{3}{13}\right) &= 3^{\frac{13-1}{2}} \\
                          & = 3^6 
                            = 1\\
                          \\
\left(\frac{8}{13}\right) &= 8^{\frac{13-1}{2}} \\
                          & = 8^6 
                            = 12
                            = -1                     
\end{align*}

This proves that \curvename{Tiny-jubjub} is SNARK-friendly. We can write the \curvename{Tiny-jubjub} curve in its affine \concept{twisted Edwards} representation as follows:
\begin{equation}\label{TJJ13-twisted-edwards}
\mathit{TJJ\_13} = \{(x,y)\in \F_{13}\times \F_{13}\;|\; 3\cdot x^{2} + y^2 =1+ 8\cdot x^{2}\cdot y^2 \}
\end{equation}

Now that we have the abstract definition of our \curvename{Tiny-jubjub} curve in \concept{twisted Edwards} form, we can compute the set of points by inserting all pairs $(x,y)\in\F_{13}\times \F_{13}$, similarly to how we computed the curve points in its \concept{short Weierstrass} or Edwards representation. We get the following:
\begin{equation}
\begin{split}
\mathit{TE\_TJJ\_13} = \{(0, 1),(0, 12),(1, 2),(1, 11),(2, 6),(2, 7),(3, 0),(5, 5),(5, 8),(6, 4),\\
(6, 9),(7, 4),(7, 9),(8, 5),(8, 8),(10, 0),(11, 6),(11, 7),(12, 2),(12, 11)\}
\end{split}
\end{equation}

We double-check our results with Sage:

\begin{sagecommandline}
sage: F13 = GF(13)
sage: L_ETJJ = []
....: for x in F13:
....:     for y in F13:
....:         if F13(3)*x^2 + y^2 == 1+ F13(8)*x^2*y^2:
....:             L_ETJJ.append((x,y))
sage: ETJJ = Set(L_ETJJ)
\end{sagecommandline}
\end{example}
\subsection{Twisted Edwards group law}
\label{sec:twisted_ed_group_law} As we have seen, \concept{twisted Edwards} curves are equivalent to Montgomery curves, and, as such, they also have a group law. However, in contrast to Montgomery and \concept{short Weierstrass} curves, the group law of SNARK-friendly \concept{twisted Edwards} curves can be described by a single computation that works in all cases, even if we add the neutral element, the inverse, or if we have to double a point. 

To see what the \concept{twisted Edwards} group law looks like, let $(x_1, y_1)$, $(x_2, y_2)$ be two points on an Edwards curve $E(\F)$. The sum of those points is then given by the following equation:

\begin{equation}\label{twisted-edwards-group-law}
(x_1, y_1) \oplus (x_2, y_2) =\left(\frac{x_1y_2+y_1x_2}{1 +dx_1x_2y_1y_2},\frac{y_1y_2-ax_1x_2}{1-dx_1x_2y_1y_2}\right)
\end{equation}

In order to see what the neutral element of the group law is, first observe that the point $(0,1)$ is a solution to the \concept{twisted Edwards} equation $a\cdot x^{2} + y^2 =1+ d\cdot x^{2}\cdot y^2$ for any parameters $a$ an $d$, and hence $(0,1)$ is a point on any \concept{twisted Edwards} curve. It can be shown that $(0,1)$ serves as the neutral element, and that the inverse of a point $(x_1, y_1)$ is given by the point $(-x_1, y1)$.
\begin{example}
\label{example:TETJJ13}
 Let's look at the \curvename{Tiny-jubjub} curve in Edwards form from \eqref{TJJ13-twisted-edwards} again. As we have seen, this curve is given by as follows:
\begin{multline}
\mathit{TE\_TJJ\_13} = \{(0, 1),(0, 12),(1, 2),(1, 11),(2, 6),(2, 7),(3, 0),(5, 5),(5, 8),(6, 4),\\
(6, 9),(7, 4),(7, 9),(8, 5),(8, 8),(10, 0),(11, 6),(11, 7),(12, 2),(12, 11)\}
\end{multline}
To get an understanding of the \concept{twisted Edwards} addition law, let's first add the neutral element $(0,1)$ to itself. We apply the group law from \eqref{twisted-edwards-group-law} and get the following:
\begin{align*}
(0, 1) \oplus (0, 1) &= \left(\frac{0\cdot 1+1 \cdot 0}{1 +8\cdot0\cdot 0\cdot 1\cdot 1},\frac{1\cdot 1-3\cdot 0\cdot 0}{1-8\cdot 0\cdot 0\cdot 1\cdot 1}\right)\\
                     & = (0,1)
\end{align*}
So, as expected, the neutral element added to itself results in the neutral element. 

Now let's add the neutral element to some other curve point. We get the following:
\begin{align*}
(0, 1) \oplus (8, 5) &= \left(\frac{0\cdot 5+1 \cdot 8}{1 +8\cdot0\cdot 8\cdot 1\cdot 5},\frac{1\cdot 5 - 3\cdot 0\cdot 8}{1-8\cdot 0\cdot 8\cdot 1\cdot 5}\right)\\
                     & = (8,5)
\end{align*}

Again, as expected, adding the neutral element to any element will result in that element. 

Given any curve point $(x,y)$, we know that its inverse is given by $(-x,y)$. To see how adding  a point to its inverse works, we compute as follows:
\begin{align*}
(5, 5) \oplus (8, 5) &= \left(\frac{5\cdot 5+5 \cdot 8}{1 +8\cdot 5\cdot 8\cdot 5\cdot 5},\frac{5\cdot 5 - 3\cdot 5\cdot 8}{1-8\cdot 5\cdot 8\cdot 5\cdot 5}\right)\\
                     &= \left(\frac{12+1}{1 +5},\frac{12 - 3}{1-5}\right)\\
                     &= \left(\frac{0}{6},\frac{12 + 10}{1+8}\right)\\
                     &= \left(0,\frac{9}{9}\right)\\
                     &=  (0,1)
\end{align*}

Adding a curve point to its inverse gives the neutral element, as expected. 


As we have seen from these examples, the \concept{twisted Edwards} addition law handles edge cases particularly well and in a unified way.
\end{example}
\begin{exercise} Consider the commutative group $(TE\_TJJ\_13,\oplus)$ from \examplename{} \ref{example:TETJJ13}.
\begin{enumerate}
\item Compute the inverse of $(1,11)$, $(0,1)$, $(3,0)$ and $(5,8)$.
\item Solve the equation $x \oplus (5,8) = (1,11) $ for some $x\in \mathit{TE\_TJJ\_13}$.
\end{enumerate}
Choose some element $x\in \mathit{TE\_TJJ\_13}$, and test if $x$ is a generator of $\mathit{TE\_TJJ\_13}$. If $x$ is not a generator, repeat until you find some generator $x$. Write $\mathit{TE\_TJJ\_13}$ in logarithmic order with respect to $x$.
\end{exercise}
%\begin{example}[Non \concept{twisted Edwards} curves have order 4 points]
%In this example, we will show that every Edwards curve has a point of order $4$. To see that let $E$ be an arbitrary Edwards curve ($a=1$). Then the point $(1,0)$ is on that curve, since $1^2+0^2= 1+d 1^2 0^2$. We compute 
%\begin{align*}
%[4](1,0) = \\
%[2]([2](1,0))=\\
%[2]\left(\frac{1 0 +1 0}{1 +d 1 1 0 0},\frac{00-1\cdot 1}{1-d1 1 00}\right)=\\
%[2](0,-1)=\\
%\left(\frac{0(-1)+(-1)0}{1 +d 0 0 (-1)(-1)},\frac{(-1)(-1)-00}{1-d00(-1)(-1)}\right) =\\
%(0,1)
%\end{align*} 

%Now having seen that every Edwards curve has point of order $4$, we can deduce that the order of every Edwards curve must contain $4$ as a factor. This is restrictive and in fact Edwards curves are rare. 
%\end{example}
\section{Elliptic Curve Pairings}
\label{sec:elliptic_curve_pairings}
As introduced in \eqref{pairing-map}, some groups come with the notion of a pairing map. In this section, we discuss \term{pairings on elliptic curves}, which form the basis of several zk-SNARKs and other zero-knowledge proof schemes, essentially because they allow computations ``in the exponent'' (see \examplename{} \ref{ex:in-the-exponent}) to be split into different parts computable by different parties.\footnote{A more detailed introduction to elliptic curve pairings can be found, for example, in \chaptname{} 6, \secname{} 6.8 and 6.9 in \cite{hoffstein-2008}.}

We start out by defining some important subgroups of the so-called full torsion group of an elliptic curve. We then introduce the Weil pairing of an elliptic curve, and describe Miller's algorithm, which makes these pairings efficiently computable.

\subsection{Embedding Degrees} As we will see in what follows, every elliptic curve gives rise to a pairing map. However, we will also see in \examplename{} \ref{ex:pairings_on_secp256k1} that not every such pairing can be efficiently computed. In order to distinguish curves with efficiently computable pairings from the rest, we need to start with an introduction to the so-called \term{embedding degree} of a curve. 

To see what the embedding degree of an elliptic curve is, let $\F$ be a finite field of order $|\F|=q$, $E(\F)$ an elliptic curve over $\F$, and let $r$ be a prime factor of the order $n$ of $E(\F)$. The embedding degree of $E(\F)$ with respect to $r$ is  the smallest integer $k$ such that the following equation holds:
\begin{equation}
\label{def:embedding-degree}
r\,|\, q^k-1
\end{equation}
Fermat's little theorem \eqref{fermats-little-theorem} implies that there always exists an embedding degree $k(r)$ for every elliptic curve and that any factor $r$ of the curve's order $n$, since $k=r-1$ is always a solution to the congruency $\kongru{q^{k}}{1}{r}$. This implies that the remainder of the integer division of $q^{r-1}-1$ by $r$ is $0$.

\begin{notation} Let $\F$ be a finite field of order $q$, and be $E(\F)$ an elliptic curve over $\F$ such that $r$ is a prime factor of the order of $E(\F)$. We write $k(r)$ for the embedding degree of $E(\F)$ with respect to $r$.
\end{notation}

\begin{example} To get a better intuition of the embedding degree, let's consider the elliptic curve $E_{1,1}(\F_5)$ from \examplename{} \ref{E1F5}. We know that the order of $E_{1,1}(\F_5)$ is $9$, and, since the only prime factor of $9$ is $3$, we compute the embedding degree of $E_{1,1}(\F_5)$ with respect to $3$. 

To find the embedding degree, we have to find the smallest integer $k$ such that $3$ divides $q^k-1= 5^k-1$. We try and increment until we find a proper $k$. 

\begin{align*}
k=1 &\text{ : } 5^1-1 = 4 & \text{ not divisible by } 3\\ 
k=2 &\text{ : } 5^2-1 = 24 & \text{ divisible by } 3
\end{align*} 

This shows that the embedding degree of the elliptic curve $E_{1,1}(\F_5)$ relative to the the prime factor $3$ of the order of $E_{1,1}(\F_5)$ is $2$ .
\end{example}

\begin{example}\label{ex:TJJ13-embedding-degree} Let us consider the \curvename{Tiny-jubjub} curve \TJJ{} from \examplename{} \ref{TJJ13}. We know that the order of \TJJ{} is $20$, and that the order therefore has two prime factors, a large prime factor $5$ and a small prime factor $2$. 

We start by computing the embedding degree of \TJJ{} with respect to the large prime factor $5$. To find this embedding degree, we have to find the smallest integer $k$ such that $5$ divides $q^k-1= 13^k-1$. We try and increment until we find a proper $k$. 
\begin{align*}
k=1 &\text{: } 13^1-1 = 12 & \text{ not divisible by } 5\\ 
k=2 &\text{: } 13^2-1 = 168 & \text{ not divisible by } 5\\ 
k=3 &\text{: } 13^3-1 = 2196 & \text{ not divisible by } 5\\ 
k=4 &\text{: } 13^4-1 = 28560 & \text{ divisible by } 5
\end{align*} 
Now we know that the embedding degree of \TJJ{} relative to the the prime factor $5$ is $k(5)=4$.

In real-world applications, like in the case of pairing-friendly elliptic curves such as \curvename{BLS\_12-381}\sme{add reference}, usually only the embedding degree of the large prime factor is relevant. In the case of our \curvename{Tiny-jubjub} curve, this is represented by $5$. It should be noted, however, that every prime factor of a curve's order has its own embedding degree despite the fact that this is mostly irrelevant in applications.

To find the embedding degree of the small prime factor $2$, we have to find the smallest integer $k$ such that $2$ divides $q^k-1= 13^k-1$. We try and increment until we find a proper $k$. 
\begin{align*}
k=1 &\text{: } 13^1-1 = 12 & \text{ divisible by } 2
\end{align*} 

Now we know that the embedding degree of \TJJ{} relative to the the prime factor $2$ is $1$. As we have seen, different prime factors can have different embedding degrees in general.

We check our computations with Sage:

\begin{sagecommandline}
sage: p = ZZ(13)
sage: # large prime factor
sage: r = ZZ(5)
sage: k = ZZ(1)
sage: while k < r:  # Fermat's little theorem
....:     if (p^k-1)%r == 0:
....:         break
....:     k=k+1
sage: k
sage: # small prime factor
sage: r = ZZ(2)
sage: k = ZZ(1)
sage: while k < r:  # Fermat's little theorem
....:     if (p^k-1)%r == 0:
....:         break
....:     k=k+1
sage: k
\end{sagecommandline}
\end{example}

\begin{example}
\label{ex:pairings_on_secp256k1}
 To give an example of a cryptographically secure real-world elliptic curve that does not have a small embedding degree, let's look at curve \curvename{secp256k1} again. We know from \examplename{} \ref{secp256k1} that the order of this curve is a prime number, which means that we only have a single embedding degree.

To test potential embedding degrees $k$, say, in the range $1\leq k < 1000$, we can use Sage and compute as follows:
\begin{sagecommandline}
sage: p = ZZ(115792089237316195423570985008687907853269984665640564039457584007908834671663)
sage: r = ZZ(115792089237316195423570985008687907852837564279074904382605163141518161494337)
sage: k = ZZ(1)
sage: while k < 1000:
....:     if (p^k-1)%r == 0:
....:         break
....:     k=k+1
sage: k
\end{sagecommandline}
We see that \curvename{secp256k1} has no embedding degree $k<1000$, which means that  \curvename{secp256k1} is a curve that has no small embedding degree. This property will be of importance later on.\sme{add reference}
\end{example}
\begin{example}
\label{ex:embedding_degre_BN128}
To give an example of a cryptographically secure real-world elliptic curve that does have a small embedding degree, let's look at curve \curvename{alt\_bn128} again. We know from \examplename{} \ref{BN128} that the order of this curve is a prime number, which means that we only have a single embedding degree.

To compute the embedding degrees $k$, we can use Sage and loop through small embedding degrees until we find as match. We compute as follows:
\begin{sagecommandline}
sage: p = ZZ(21888242871839275222246405745257275088696311157297823662689037894645226208583)
sage: r = ZZ(21888242871839275222246405745257275088548364400416034343698204186575808495617)
sage: k = ZZ(1)
sage: # degree is supposed to be small
sage: while k < 50: 
....:     if (p^k-1)%r == 0:
....:         break
....:     k=k+1
sage: k
\end{sagecommandline}
\end{example}



\subsubsection{Elliptic Curves over extension fields}
\label{sec:curve-extensions}
\sme{TODO:Rewrite intro together with Sven} Suppose that $p$ is a prime number, and $\F_p$ its associated prime field. We know from equation \eqref{eq:prime-extension-field} that the fields $\F_{p^m}$ are extensions of $\F_p$ in the sense that $\F_{p}$ is a subfield of $\F_{p^m}$. This implies that we can extend the affine plane that an elliptic curve is defined on by changing the base field to any extension field. To be more precise, let 
$E(\F)=\{ (x,y)\in \F \times \F \;|\; y^2 = x^3 +a\cdot x +b \}$ be an affine \concept{short Weierstrass} curve, with parameters $a$ and $b$ taken from $\F$. If $\F'$ is an extension field of $\F$, then we extend the domain of the curve by defining $E(\F')$ as follows:

\begin{equation}\label{elliptic-curve-extension}
E(\F')=\{ (x,y)\in \F' \times \F' \;|\; y^2 = x^3 +a\cdot x +b \}
\end{equation}   

While we did not change the defining parameters, we consider curve points from the affine plane over the extension field now. Since $\F\subset \F'$, it can be shown that the original elliptic curve $E(\F)$ is a sub-curve of the extension curve $E(\F')$.

\begin{example}\label{ex:EF52} Consider the prime field $\F_5$ from \examplename{} \ref{prime-field-F5} together with the elliptic curve $E_{1,1}(\F_5)$ and its definition from \examplename{} \ref{E1F5} and the construction the extension field $\F_{5^2}$ relative to the polynomial $t^2+2 \in \F_5[t]$ from exercise \ref{exercise:finite_fieldF5_2}. In this example we extend the definition of $E_{1,1}(\F_5)$ to an elliptic curve over $\F_{5^2}$ and compute its set of points:
$$
E_1(\F_{5^2}) = \{ (x,y)\in \F_{5^2}\times \F_{5^2} \;|\; y^2 = x^3 + x +1 \}
$$
Since $\F_{5^2}$ contains $25$ points, in order to compute the set $E_1(\F_{5^2})$, we have to try $25\cdot 25 = 625$ pairs, which is probably a bit tedious. Instead, we use Sage to compute the curve for us. To do, we choose the representation of $\F_{5^2}$ from \ref{exercise:finite_fieldF5_2}. We get:
\begin{sagecommandline}
sage: F5= GF(5)
sage: F5t.<t> = F5[] 
sage: P_MOD_2 = F5t(t^2+2)
sage: P_MOD_2.is_irreducible()
sage: F5_2.<t> = GF(5^2, name='t', modulus=P_MOD_2)
sage: E1F5_2 = EllipticCurve(F5_2,[1,1])
sage: E1F5_2.order()
\end{sagecommandline}
The curve $E_1(\F_{5^2})$ consist of $27$ points, in contrast to curve $E_1(\F_{5})$, which consists of $9$ points. Writing those points down gives the following:
\begin{multline*}
E_1(\F_{5^2}) = \{\Oinf, (0, 4), (0, 1), (3, 4), (3, 1), (4, 3), (4, 2), (2, 4), (2, 1),\\ 
(4t + 3, 3t + 4), (4t + 3, 2t + 1),  (3t + 2, t), (3t + 2, 4t),\\ 
(2t + 2, t), (2t + 2, 4t), (2t + 1, 4t + 4), (2t + 1, t + 1),\\ 
(2t + 3, 3), (2t + 3, 2), (t + 3, 2t + 4), (t + 3, 3t + 1),\\ 
(3t + 1, t + 4), (3t + 1, 4t + 1), (3t + 3, 3), (3t + 3, 2), (1, 4t),  (1, t)
\}
\end{multline*}
As we can see, the set of points from the elliptic curve $E_{1,1}(\F_5)$ is a subset of the sets of points from the elliptic curve $E(\F_{5^2})$. This was expected since the prime field $\F_5$ is a subfield of the finite field $\F_{5^2}$.
\end{example}
\begin{exercise} Consider the \concept{short Weierstrass} elliptic curve $E(\F_{5^2})$ from \examplename{} \ref{ex:EF52}, compute the expression $(4t+3,2t+1)\oplus(3t+3,2)$ using pen and paper and double-check the computation using sage. Then solve the equation $x\oplus (3t+3,3)=(3,4)$ for some $x\in E(\F_{5^2})$. After that compute the scalar multiplication $[5](2t + 1, 4t + 4)$ using the double-and-add algorithm from exercise \ref{alg_double-and-add}.
\end{exercise}
\begin{exercise}
\label{exercise:TJJ134}
 Consider the \curvename{Tiny-jubjub} curve from \examplename{} \ref{TJJ13}. Show that the polynomial $t^4+2\in \F_{13}[t]$ is irreducible. Then write a sage program to implement the finite field extension $\F_{13^4}$, implement the curve extension $TJJ\_13(\F_{13^4})$ and compute the number of curve points.
\end{exercise}
\begin{exercise}
\label{exercise:BN128-extension}
Consider the \curvename{alt\_bn128} curve and its associated base field $\F_{bn128}$ from \examplename{} \ref{BN128}. As we know from example \ref{ex:embedding_degre_BN128} this curve has an embedding degree of $12$. Use Sage to find an irreducible polynomial $P\in \F_{bn128}[t]$ and write a sage program to implement the finite field extension $\F_{p^{12}}$, where $p$ is the modulus of $\F_{bn128}$ and to implement the curve extension $alt\_bn128(\F_{p^12})$ and compute the number of curve points.
\end{exercise}
\subsection{Full torsion groups}
\label{sec:full-torsion} As we will see in what follows, cryptographically interesting pairings are defined on so-called torsion subgroups of elliptic curves. To define \term{torsion groups} of an elliptic curve, let $\F$ be a finite field, $E(\F)$ an elliptic curve of order $n$ and $r$ a factor of $n$. Then the \term{$r$-torsion group} of the elliptic curve $E(\F)$ is defined as the set
\begin{equation}
\label{def:torsion_group}
E(\F)[r]:= \{P\in E(\F)\;|\; [r]P = \mathcal{O}\}
\end{equation} 
The fundamental theorem of finite cyclic groups \ref{def:fundamental_theorem_groups} states that every factor $r$ of a cyclic group's order uniquely defines a subgroup of the size of that factor and those subgroup are important examples of $r$-torsion groups. We have seen examples of those subgroups in \ref{ex:G1G2-subgroups} and \ref{ex:TJJ13-cofactor-clearing}.

When we consider elliptic curve extensions as defined in \ref{elliptic-curve-extension}, we could ask what happens to the $r$-torsion groups in the extension. One might intuitively think that their extension just parallels the extension of the curve. For example, when $E(\F_p)$ is a curve over prime field $\F_p$, with some $r$-torsion group $E(\F_p)[r]$ and when we extend the curve to $E(\F_{p^m})$, then there might be a bigger $r$-torsion group $E(\F_{p^m})[r]$ such that $E(\F_p)[r]$ is a subgroup of $E(\F_{p^m})[r]$. This might make intuitive sense, as $E(\F_p)$ is a subset of $E(\F_{p^m})$. 

However, the actual situation is a bit more surprising than that. To see that, let $\F_p$ be a prime field and let $E(\F_p)$ be an elliptic curve of order $n$, such that $r$ is a factor of $n$, with embedding degree $k(r)$ and $r$-torsion group $E(\F_p)[r]$. Then the $r$-torsion group $E(\F_{p^m})[r]$ of a curve extension is equal to $E(\F_p)[r]$, only as long as the power $m$ is less than the embedding degree $k(r)$ of $E(\F_p)$. 

For the prime power $p^{k(r)}$, the $r$-torsion group $E(\F_{p^{k(r)}})[r]$ might then be larger than $E(\F_p)[r]$ and it contains $E(\F_p)[r]$ as a subgroup. We call it the \term{full $r$-torsion group} of that elliptic curve and write is as follows
\begin{equation}
\label{def:full_torsion_group}
E[r] := E(\F_{p^{k(r)}})[r]
\end{equation}
The $r$-torsion groups $E(\F_{p^m})[r]$ of any curve extensions for $m>k(r)$ are all equal to $E[r]$. In this sense $E[r]$ is already the largest $r$-torsion group, which justifies the name. The full $r$-torsion group contains $r^2$ many elements and consists of $r+1$ subgroups, one of which is $E(\F_{p})[r]$. The following diagram summarizes the situation:
\begin{equation}
\label{def:full_torsion_group_tower}
\begin{array}{lccclclclcl}
E(\F_{p}) & \subset & \cdots  & \subset & E(\F_{p^{k(r)-1}}) & \subset & E(\F_{p^{k(r)}}) & \subset & E(\F_{p^{k(r)+1}}) & \subset & \ldots\\
E(\F_{p})[r] & = & \cdots  & = & E(\F_{p^{k(r)-1}})[r] & \subset & E(\F_{p^{k(r)}})[r] & = & 
E(\F_{p^{k(r)+1}})[r] & = & \ldots
\end{array}
\end{equation}

So, when we consider nested elliptic curve extensions as in \ref{def:full_torsion_group_tower}, ordered by the prime power $m$, then the $r$-torsion group stays constant for every level $m$ that is smaller than the embedding degree $k(r)$, while it suddenly blossoms into a larger group on level $k(r)$ with $r+1$ subgroups, and then all $r$-torsion groups on higher levels $m\geq k(r)$ stay the same. In other words, once the extension field is big enough to find one more curve point $P$ with $[r]P=\mathcal{O}$ that is not an element of the curve over the base field, then we actually find all of the points in the full torsion group.

\begin{example}
\label{example:E1_full_torsion}
 Consider curve $E_{1,1}(\F_5)$ again. We know from \ref{ex:G1G2-subgroups} that it contains a $3$-torsion group and that the embedding degree of $3$ is $k(3)=2$. From this we can deduce that we can find the full $3$-torsion group $E_1[3]$ in the curve extension $E_1(\F_{5^2})$, the latter of which we computed in \examplename{} \ref{ex:EF52}. 

Since that curve is small, in order to find the full $3$-torsion, we can loop through all elements of $E_1(\F_{5^2})$ and check the defining equation $[3]P= \Oinf$. Invoking Sage and using our implementation of $E_1(\F_{5^2})$ in sage from \ref{ex:EF52}, we compute as follows:
\begin{sagecommandline}
sage: INF = E1F5_2(0) # Point at infinity
sage: L_E1_3 = []
sage: for p in E1F5_2:
....:     if 3*p == INF:
....:         L_E1_3.append(p)
sage: E1_3 = Set(L_E1_3) # Full 3-torsion set
\end{sagecommandline}
$$
E_1[3] = \{\Oinf,(2,1),(2,4),(1,t),(1,4t), (2t + 1,t + 1), (2t + 1, 4t + 4), (3t + 1,t + 4), (3t + 1, 4t + 1) \}
$$
As we can see the group $E_1[3]$ contains $9=3^3$ many elements and the $3$-torsion group $E_{1,1}(\F_5)[3]$ of the curve over the prime field is a subset of the full torsion group. 
\end{example}
\begin{example}\label{ex:TJJ13-full-torsion} Consider the \curvename{Tiny-jubjub} curve from \examplename{} \ref{TJJ13}. We know from \examplename{} \ref{ex:TJJ13-embedding-degree} that it contains a $5$-torsion group and that the embedding degree of $5$ is $4$. This implies that we can find the full $5$-torsion group $\mathit{TJJ\_13}[5]$ in the curve extension $\mathit{TJJ\_13}(\F_{13^4})$. 

To compute the full torsion, first observe that, since $\F_{13^4}$ contains $28561$ elements, computing $\mathit{TJJ\_13}(\F_{13^4})$ means checking $28561^2=815730721$ elements. From each of these curve points $P$, we then have to check the equation $[5]P=\Oinf$. Doing this for $815730721$ is a bit too slow even on a computer.

Fortunately, Sage has a funcion that computes all points $P$, such that $[m]P=Q$ for given integer $m$ and curve point $Q$. Using the curve extension from exercise \ref{exercise:TJJ134}, the following Sage code  provides a way to compute the full torsion group:
\begin{sagecommandline}
sage: # define the extension field
sage: F13= GF(13) # prime field
sage: F13t.<t> = F13[] # polynomials over t
sage: P_MOD_4 = F13t(t^4+2) # degree 4 irreducible polynomial
sage: P_MOD_4.is_irreducible()
sage: F13_4.<t> = GF(13^4, name='t', modulus=P_MOD_4)
sage: TJJF13_4 = EllipticCurve(F13_4,[8,8]) # TJJ extension
sage: # compute the full 5-torsion
sage: INF = TJJF13_4(0) # point at infinity
sage: L_TJJF13_4_5 = INF.division_points(5) # [5]P == INF
sage: TJJF13_4_5 = Set(L_TJJF13_4_5)
sage: TJJF13_4_5.cardinality()	# number of elements
\end{sagecommandline}
As expected, we get a group that contains $5^2=25$ elements. To see that the embedding degree $4$ is actually the smallest prime power to find the full $5$-torsion group, let's compute the $5$-torsion group over of the \curvename{Tiny-jubjub} curve of the extension field $\F_{13^3}$. We get the following:
\begin{sagecommandline}
sage: # define the extension field
sage: P_MOD_3 = F13t(t^3+2) # degree 3 irreducible polynomial
sage: P_MOD_3.is_irreducible()
sage: F13_3.<t> = GF(13^3, name='t', modulus=P_MOD_3)
sage: TJJF13_3 = EllipticCurve(F13_3,[8,8]) # TJJ extension
sage: # compute the 5-torsion
sage: INF = TJJF13_3(0)
sage: L_TJJF13_3_5 = INF.division_points(5) # [5]P == INF
sage: TJJF13_3_5 = Set(L_TJJF13_3_5) # $5$-torsion
sage: TJJF13_3_5.cardinality()	# number of elements
\end{sagecommandline}

As we can see, the $5$-torsion group of \curvename{Tiny-jubjub} over $\F_{13^3}$ is equal to the $5$-torsion group of \curvename{Tiny-jubjub} over $\F_{13}$ itself. 
\end{example}

\begin{example} 
\label{example:secp256k1}
% https://www.sikoba.com/docs/SKOR_SV_Pairing_Based_Crypto.pdf
Let's look at the curve \curvename{secp256k1}. We know from \examplename{} \ref{secp256k1} that the curve is of some prime order $r$. Because of this, the only torsion group to consider is the curve itself, so the curve group is the $r$-torsion. 

In order to find the full $r$-torsion of \curvename{secp256k1}, we need to compute the embedding degree $k$. And as we have seen in \ref{ex:pairings_on_secp256k1} it is at least not small. However, we know from Fermat's little theorem \ref{fermats-little-theorem} that a finite embedding degree must exist. It can be shown that it is given by the following 256-bit number:
$$
k = \scriptstyle 192986815395526992372618308347813175472927379845817397100860523586360249056 
$$
This means that the embedding degree is very large, which implies that the field extension $\F_{p^k}$ is very large too. To understand how big $\F_{p^k}$ is, recall that an element of $\F_{p^m}$ can be represented as a string $<x_0,\ldots,x_m>$ of $m$ elements, each containing a number from the prime field $\F_p$. Now, in the case of \curvename{secp256k1}, such a representation has $k$-many entries, each of them $256$ bits in size. So, without any optimizations, representing such an element would need $k\cdot 256$ bits, which is too much to be representable in the observable universe. It follows that it is not only infeasible to compute the full $r$-torsion group of \curvename{secp256k1}, but moreover to even write down single elements of that group in general. 
\end{example}
\begin{exercise} Consider the full $5$-torsion group $TJJ\_13[5]$ from \examplename{} \ref{ex:TJJ13-full-torsion}. Write down the set of all elements from this group and identify the subset of all elements from $TJJ\_13(\F_{13})[5]$ as well as $TJJ\_13(\F_{13^2})[5]$. Then compute the $5$-torsion group $TJJ\_13(\F_{13^{8}})[5]$ .
\end{exercise}
\begin{exercise} Consider the curve \curvename{secp256k1} from \examplename{} \ref{secp256k1} and its full $r$-torsion group as introduced in \examplename{} \ref{example:secp256k1}. Write down a single element from the curves full torsion group that is not the point at infinity.
\end{exercise}
\begin{exercise}
\label{exercise:BN128-full-torsion}
 Consider the curve \curvename{alt\_bn128} from \examplename{} \ref{BN128} and its curve extension from exercise \ref{exercise:BN128-extension}. Write a Sage program that computes a generator from the curves full torsion group.
\end{exercise}

\subsection{Pairing groups}
\label{sec:pairing_groups}
 As we have stated above, any full $r$-torsion group contains $r+1$ cyclic subgroups, two of which are of particular interest in pairing-based elliptic curve cryptography. To characterize these groups, we need to consider the so-called \term{Frobenius endomorphism} of an elliptic curve $E(\F)$ over some finite field $\F$ of characteristic $p$:
\begin{equation}\label{eq:frobenius-enomorphism}
\pi : E(\F) \to E(\F): \;\; 
\begin{array}{lcl}
(x,y)       &\mapsto & (x^p,y^p)\\
\Oinf &\mapsto & \Oinf
\end{array} 
\end{equation}
It can be shown that $\pi$ maps curve points to curve points. The first thing to note is that, in case  $\F$ is a prime field, the Frobenius endomorphism acts as the identity map, since $(x^p,y^p) = (x,y)$ on prime fields due to Fermat's little theorem \ref{fermats-little-theorem}. This means that the Frobenius map is more interesting on elliptic curves over prime field extensions.

With the Frobenius map at hand, we can characterize two important subgroups of the full $r$-torsion group $E[r]$ of an elliptic curve. The first subgroup is the group of elements from the full $r$-torsion group, on which the Frobenius map acts trivially. Since in pairing-based cryptography, this group is usually written as $\G_1$, assuming that the prime factor $r$ in the definition is implicitly given, we define $\G_1$ as follows:

\begin{equation}
\label{def:pairing_group_G1}
\G_1[r] := \{(x,y)\in E[r]\;|\; \pi(x,y) = (x,y)\;\}
\end{equation}

It can be shown that $\G_1$ is precisely the $r$-torsion group $E(\F_p)[r]$ of the unextended elliptic curve defined over the prime field. There is another subgroup of the full $r$-torsion group that can be characterized by the Frobenius map and in the context of pairing-based cryptography, this subgroup is often called $\G_2$. This group is defined as follows:

\begin{equation}
\label{def:pairing_group_G2}
\G_2[r]:= \{(x,y)\in E[r]\;|\; \pi(x,y) = [p](x,y)\;\}
\end{equation}

\begin{notation} If $E(\F)$ is an elliptic curve and $r$ is the largest prime factor of the curves order, we call $\G_1[r]$ and $\G_2[r]$ \term{pairing groups}. If the prime factor $r$ is clear from the context, we sometimes simply write $\G_1$ and $\G_2$ to mean $\G_1[r]$ and $\G_2[r]$, respectively. 
\end{notation}

It should be noted that other definitions of $\G_2$ exists in the literature, too. However, in the context of pairing-based cryptography, this is a common choice as it is particularly useful because we can define efficient hash functions that map into $\G_2$, which is not possible for all subgroups of the full $r$-torsion.

\begin{example} Consider the curve $E_{1,1}(\F_5)$ from \examplename{} \ref{E1F5} again. As we have seen, this curve has the embedding degree $k=2$, and a full $3$-torsion group is given as follows:
\begin{multline}
E_1[3] = \{\Oinf,(2,1),(2,4), (1,t), (1,4t), (2t + 1,t + 1),\\ (2t + 1, 4t + 4),
(3t + 1,t + 4), (3t + 1, 4t + 1) \}
\end{multline}

According to the general theory, $E_1[3]$ contains $4$ subgroups, and we can characterize the subgroups $\G_1$ and $\G_2$ using the Frobenius endomorphism. Unfortunately, at the time of writing, Sage does not have a predefined Frobenius endomorphism for elliptic curves, so we have to use the Frobenius endomorphism of the underlying field as a temporary workaround. Using our implementation of $E_1[3]$ in sage from \examplename{} \ref{example:E1_full_torsion}, we compute $\G_1$ as follows:
\begin{sagecommandline}
sage: L_G1 = []
sage: for P in E1_3: 
....:     PiP = E1F5_2([a.frobenius() for a in P]) # pi(P)
....:     if P == PiP:
....:         L_G1.append(P)
sage: G1 = Set(L_G1)
\end{sagecommandline}
As expected, the group $\G_1=\{\Oinf, (2,4), (2,1)\}$ is identical to the $3$-torsion group of the (unextended) curve over the prime field $E_{1,1}(\F_5)$. 

In order to compute the group $\G_2$ for the curve $E_{1,1}(\F_5)$, we can use almost the same algorithm as we used for the computation of $\G_1$. Since $p=5$ we get the following:
\begin{sagecommandline}
sage: L_G2 = []
sage: for P in E1_3: 
....:     PiP = E1F5_2([a.frobenius() for a in P]) # pi(P)
....:     pP = 5*P # [5]P
....:     if pP == PiP:
....:         L_G2.append(P)
sage: G2 = Set(L_G2)
\end{sagecommandline}

Thus, we have computed the pairing group $\G_2$ of the full $3$-torsion group of curve $E_{1,1}(\F_5)$ as the set $\G_2 = \{\Oinf, (1,t), (1,4t)\}$. 
\end{example}

\begin{example}
\label{example:TJJ_pairing_groups}
Consider the \curvename{Tiny-jubjub} curve \TJJ{} from \examplename{} \ref{TJJ13}. In \examplename{} \ref{ex:TJJ13-full-torsion} we computed its full $5$ torsion, which is a group that has $6$ subgroups. We compute $\G_1$ using Sage as follows:
\begin{sagecommandline}
sage: L_TJJ_G1 = []
sage: for P in TJJF13_4_5: 
....:     PiP = TJJF13_4([a.frobenius() for a in P]) # pi(P)
....:     if P == PiP:
....:         L_TJJ_G1.append(P)
sage: TJJ_G1 = Set(L_TJJ_G1)
\end{sagecommandline}
We get $\G_1= \{\Oinf, (7,2), (8,8), (8,5), (7,11)\}$ and as expected, $\G_1$ is identical to the $5$-torsion group of the (unextended) curve over the prime field $TJJ_13$ as computed in \examplename{} \ref{eq:TJJ13-logarithmic-order}.

In order to compute the group $\G_2$ for the tiny jubjub curve, we can use almost the same algorithm as we used for the computation of $\G_1$. Since $p=13$ we get the following:
\begin{sagecommandline}
sage: L_TJJ_G2 = []
sage: for P in TJJF13_4_5: 
....:     PiP = TJJF13_4([a.frobenius() for a in P]) # pi(P)
....:     pP = 13*P # [13]P
....:     if pP == PiP:	# pi(P) ==[13]P
....:         L_TJJ_G2.append(P)
sage: TJJ_G2 = Set(L_TJJ_G2)
\end{sagecommandline}
$\G_2 = \{\Oinf, (9t^2 + 7,t^3 + 11t), (9t^2 + 7, 12t^3 + 2t), (4t^2 + 7,5t^3 + 10t),(4t^2 + 7,8t^3 + 3t)\}$
\end{example}

\begin{example}
\label{example:secp256k1_pairing_groups}
Consider Bitcoin's curve \curvename{secp256k1} again. Since the group $\G_1$ is identical to the torsion group of the unextended curve, and since \curvename{secp256k1} has prime order, we know that, in this case, $\G_1$ is identical to \curvename{secp256k1} itself. However it is infeasible to compute elements from $\G_2$, since according to \examplename{} \ref{example:secp256k1} we can not store avarage curve points from the extension curve $secp256k1(\F_{p^k})$ on any computer, let alone compute their images under the Frobenious map.
\end{example}
\begin{exercise}
Consider the small prime factor $2$ of the \curvename{Tiny-jubjub} curve. Compute the full $2$-torsion group of $TJJ\_13$ and then compute the groups $\G_1[2]$ and $\G_2[2]$.  
\end{exercise}
\begin{exercise}
\label{exercise:BN128-pairing-groups}
 Consider the curve \curvename{alt\_bn128} from \examplename{} \ref{BN128} and its curve extension from exercise \ref{exercise:BN128-extension}. Write a Sage program that computes a generator for each of the torsion group $\G_1[p]$ and $\G_2[p]$.
\end{exercise}


\subsection{The Weil pairing}
\label{sec:weil-pairing} Recall the definition of a non-degenerate group pairing from \ref{pairing-map}. In this part, we consider a pairing function defined on
the subgroups $\G_1[r]$ and $\G_2[r]$ of the full $r$-torsion $E[r]$ of a \concept{short Weierstrass} elliptic curve. To be more precise, let $E(\F_p)$ be an elliptic curve of embedding degree $k$ such that $r$ is a prime factor of its order. Then the \term{Weil pairing} is defined as the following bilinear, non-degenerate map:

\begin{equation}\label{eq:weil-pairing}
e(\cdot,\cdot) : \G_1[r] \times \G_2[r] \to \F^*_{p^k}\; ;\; 
(P,Q)\mapsto (-1)^r \cdot \frac{f_{r,P}(Q)}{f_{r,Q}(P)}
\end{equation} 

The extension field elements $f_{r,P}(Q), f_{r,Q}(P)\in \F_{p^k}$ in the definition of the Weil pairing are computed by \term{Miller's algorithm} below.

\begin{algorithm}\caption{Miller's algorithm for \concept{short Weierstrass} curves $y^2 = x^3 +ax +b$}\label{alg:millersalgo}
% https://www.math.u-bordeaux.fr/~damienrobert/csi2018/pairings.pdf
\begin{algorithmic}[0]
\Require $r>3$, $P \in E[r]$, $Q\in E[r]$ and
\State $b_0,\ldots, b_t\in \{0,1\}$ with $r= b_0\cdot 2^0 + b_1\cdot 2^1 + \ldots + b_t\cdot 2^t$ and $b_t=1$
\Procedure{Miller's Algorithm}{$P,Q$}
\If{$P = \Oinf$ or $Q = \Oinf$ or $P = Q$}
	\State \textbf{return} $f_{r,P}(Q) \gets (-1)^r$
\EndIf
\State $(x_T,y_T) \gets (x_P,y_P)$
\State $f_1\gets 1$
\State $f_2\gets 1$
\For{$j\gets t-1,\ldots, 0$}
	\State $m \gets \frac{3\cdot x_T^2+a}{2\cdot y_T}$	
    \State $f_1 \gets f_1^2\cdot (y_Q - y_T - m\cdot(x_Q-x_T))$
	\State $f_2 \gets f_2^2\cdot (x_Q + 2x_T -m^2)$
	\State $x_{2T} \gets m^2 - 2 x_T$
	\State $y_{2T} \gets -y_T - m\cdot (x_{2T}-x_T)$
	\State $(x_T,y_T)\gets (x_{2T},y_{2T})$ 
	\If{$b_j = 1$}
		\State $m \gets \frac{y_T -y_P}{x_T - x_P}$
		\State $f_1 \gets f_1\cdot (y_Q -y_T -m\cdot (x_Q - x_T))$
		\State $f_2 \gets f_2\cdot (x_Q + (x_P+x_T) - m^2)$
		\State $x_{T+P} \gets m^2 -x_T -x_P$
		\State $y_{T+P}\gets -y_T - m\cdot (x_{T+P}-x_T)$
		\State $(x_T,y_T)\gets (x_{T+P},y_{T+P})$
	\EndIf
\EndFor
\State $f_1 \gets f_1\cdot (x_Q - x_T)$
\State \textbf{return} $f_{r,P}(Q) \gets \frac{f_1}{f_2}$
\EndProcedure
\end{algorithmic}
\end{algorithm}

Understanding the details of how and why this algorithm works requires the concept of \term{divisors}, which is outside of the scope this book. The interested reader might look at \chaptname{} 6, \secname{} 6.8.3 in \cite{hoffstein-2008}, or at \href{https://static1.squarespace.com/static/5fdbb09f31d71c1227082339/t/5ff394720493bd28278889c6/1609798774687/PairingsForBeginners.pdf}{Craig Costello’s great tutorial on elliptic curve pairings}.\tbds{add this to references}  As we can see, the algorithm is more efficient on prime numbers $r$, that have a low Hamming weight \ref{def:binary_representation_integer}.

We call an elliptic curve $E(\F_p)$ \term{pairing-friendly} if there is a prime factor of the groups order such that the Weil pairing is efficiently computable with respect to that prime factor. In real-world applications of pairing-friendly elliptic curves, the embedding degree is usually a small number like $2$, $4$, $6$ or $12$, and the number $r$ is the largest prime factor of the curve's order. 

\begin{example}Consider curve $E_{1,1}(\F_5)$ from \examplename{} \ref{E1F5}. Since the only prime factor of the group's order is $3$, we cannot compute the Weil pairing on this group using our definition of Miller's algorithm. In fact, since $\G_1$ is of order $3$, executing the algorithm will lead to a ``division by zero''.
\end{example}

\begin{example} Consider the \curvename{Tiny-jubjub} curve $\mathit{TJJ\_13}(\F_{13})$ from \examplename{} \ref{TJJ13} and its associated pairing groups from \examplename{} \ref{example:TJJ_pairing_groups}:
\begin{align*}
\G_1[5] & = \{\Oinf, (7,2), (8,8), (8,5), (7,11)\}\\
\G_2[5] & = \{\Oinf, (9t^2 + 7,t^3 + 11t), (9t^2 + 7, 12t^3 + 2t), 
(4t^2 + 7, 5t^3 + 10t), (4t^2 + 7, 8t^33 + 3t)\}
\end{align*}

Since we know from \examplename{} \ref{ex:TJJ13-embedding-degree} that the embedding degree of $5$ id $4$, we can instantiate the general definition of the Weil pairing for this example as follows:
$$
e(\cdot,\cdot): \G_1[5] \times \G_2[5] \to \F_{13^4}
$$ 

The first if-statement in Miller's algorithm, implies that $e(\Oinf,Q)=1$ as well as $e(P,\Oinf)=1$ for all arguments $P\in\G_1[5]$ and $Q\in \G_2[5]$. In order to compute a non-trivial Weil pairing, we choose the argument $P=(7,2)\in \G_1$ and $Q=(9t^2 + 7, 12t^3 + 2t)\in\G_2$. Invoking sage we get the following computation of the Weil pairing:  
\begin{sagecommandline}
sage: F13 = GF(13)
sage: F13t.<t> = F13[]
sage: P_MOD_4 = F13t(t^4+2)
sage: F13_4.<t> = GF(13^4, name='t', modulus=P_MOD_4)
sage: TJJF13_4 = EllipticCurve(F13_4,[8,8])
sage: P=TJJF13_4([7,2])
sage: Q=TJJF13_4([9*t^2+7,12*t^3+2*t])
sage: P.weil_pairing(Q,5)
\end{sagecommandline}
\end{example}
\begin{example}
Consider Bitcoin's curve \curvename{secp256k1} again. As we have seen in \examplename{} \ref{example:secp256k1_pairing_groups}, it is infeasible to compute elements from the pairing group $\G_2$ and as we know from \examplename{} \ref{example:secp256k1} it is moreover infeasible to do calculations in the extension field $\F_{p^k}$. It follows that the Weil pairing is not efficiently computable and that \curvename{secp256k1} is not pairing friendly. 
\end{example}
\begin{exercise}
\label{exercise:BN128-pairing}
Consider the curve \curvename{alt\_bn128} from \examplename{} \ref{BN128} and the generators $g_1$ and $g_2$ of $\G_1[p]$ and $\G_2[p]$ from exercise \ref{exercise:BN128-pairing-groups}. Write a Sage program that computes the Weil pairing $e(g_1,g_2)$.
\end{exercise}

% http://www.pdmi.ras.ru/~lowdimma/BSD/Silverman-Arithmetic_of_EC.pdf
% p. 396ff

\section{Hashing to Curves} Elliptic curve cryptography frequently requires the ability to hash data onto elliptic curves. If the order of the curve is not a prime number, hashing to prime order subgroups is of importance, too and in the context of pairing-friendly curves, it is sometimes necessary to hash specifically onto the pairing group $\G_1$ or $\G_2$ as introduced in \ref{sec:pairing_groups}.

As we have seen in section \ref{sec:hashing-to-groups}, some general methods are known for hashing into finite cyclic groups and since elliptic curves over finite fields are finite and cyclic groups, those methods can be utilized in this case, too. However, in what follows we want to describe some methods specific to elliptic curves that are frequently used in real-world applications. 

\subsection{Try-and-increment hash functions}
One of the most straight-forward ways of hashing onto an elliptic curve point in a secure way is to use a cryptographic hash function together with one of the hashing into modular aithmetics methods as described in section \ref{hash-to-modular-arithmetics}.

Both constructions can be combined in such a way that the image provides an element of the base field of the elliptic curve together with a single auxiliary bit. The base field element can then be interpreted as the $x$-coordinate of a potential curve point, and the auxiliary bit can be used to determine one of the two possible $y$ coordinates of that curve point as explained in \ref{sec:affine_point_compression}.

Such an approach would be deterministic and easy to implement, and it would conserve the cryptographic properties of the original hash function. However, not all $x$ coordinates generated in such a way will result in quadratic residues when inserted into the defining equation. It follows that not all field elements give rise to actual curve points. 

In fact,
% https://www.cs.umd.edu/users/gasarch/TOPICS/res/burgess.pdf
on a prime field, only half of the field elements are quadratic residues. Hence, assuming an even distribution of the hash values in the field, this method would fail to generate a curve point in about half of the attempts. 

One way to account for this problem is the following so-called \term{try-and-increment} method. Instead of simply hashing a binary string $s$ to the field, this method use a try-and-increment hash to the base field as described in \ref{def:try_and_increment_hash} in combination with a single auxiliary bit derived from the underlying cryptographic hash function.

If any try of hashing to the field does not result in a field element or a valid curve point, the counter is incremented, and the hashing is repeated. This is done until a valid curve point is found (see the algorithm below).

\begin{algorithm}\label{alg:hash-to-e}\caption{Hash-to-$E(\F_p)$}
\begin{algorithmic}[0]
\Require $p \in \Z$ with $|p|=k$ and $s\in\{0,1\}^*$
\Require Curve equation $y^2 = x^3 + ax +b$ over $\F_p$
\Procedure{Try-and-Increment}{$r,k,s$}
\State $c \gets 0$	\Comment{Try-and-Increment counter}
\Repeat
\State $s' \gets s||Bits(c)$
\State $x \gets H(s')_0\cdot 2^0 + H(s')_1\cdot 2^1 + \ldots + H(s')_{k}\cdot 2^{k}$ \Comment{potential $x$}
\State $y^2 \gets z^3 + a\cdot z + b$ \Comment{potential $y^2$}
\State $c\gets c+1$
\Until{$x<p$ and $\Zmod{(y^2)^{\frac{p-1}{2}}}{r}=1$ } \Comment{Check $x$ in field and $y^2$ has root}
\If {$H(s')_{k+1} == 0$} \Comment{auxiliary bit decides root}
\State $y \gets y'\in \sqrt{y^2}$ with $0\leq y' \leq (p-1)/2$
\Else 
\State $y \gets y'\in \sqrt{y^2}$ with $(p-1)/2 < y' < p$
\EndIf
\State \textbf{return} $(x,y)$
\EndProcedure
\Ensure $(x,y)\in E(\F_r)$
\end{algorithmic}
\end{algorithm}

The try-and-increment method is relatively easy to implement, and it maintains the cryptographic properties of the original hash function. It should be noted that if the curve is not of prime order, the image of the try-and-increment hash will be a general curve point that might not be an element from the large prime-order subgroup. To map onto the large prime order subgroup it is therefore necessary to apply the technique of cofactor clearing as explained in \ref{def:cofactor_clearing}.

\begin{example} Consider the \curvename{Tiny-jubjub} curve from \examplename{} \ref{TJJ13}. We want to construct a try-and-increment hash function that maps a binary string $s$ of arbitrary length onto the large prime-order subgroup of size $5$ from \examplename{} \ref{eq:TJJ13-logarithmic-order}. 

Since the curve $TJJ\_13$ is defined over the field $\F_{13}$, and the binary representation of $13$ is $Bits(13)=<1,1,0,1>$, one way to implement a try-and-increment function is to apply SHA256 from Sage's hashlib library on the concatenation $s||c$ for some binary counter string $c$, and use the first $4$ bits of the image to try to hash into $\F_{13}$. In case we are able to hash to a value $x$ such that $x^3 +8\cdot x + 8$ is a quadratic residue in $\F_{13}$, we use the fifth bit to decide which of the two possible roots of $x^3 + 8\cdot x + 8$ we will choose as the $y$ coordinate. The result is a curve point different from the point at infinity. To project it onto the large prime order subgroup $TJJ\_13[5]$, we multiply it with the cofactor $4$. If the result is not the point at infinity, it is the result of the hash.

To make this concrete, let $s=<1,1,1,0,0,1,0,0,0,0>$ be our binary string that we want to hash onto $TJJ_13[5]$. We use a binary counter string starting at zero, that is, we choose $c=<0>$. Invoking Sage, we define the try-hash function as follows:
\begin{sagecommandline}
sage: import hashlib
sage: def try_hash(s,c):
....:     s_1 = s+c # string concatenation
....:     hasher = hashlib.sha256(s_1.encode('utf-8')) # compute SHA256
....:     digest = hasher.hexdigest()
....:     z = ZZ(digest, 16) # cast into integer
....:     z_bin = z.digits(base=2, padto=256) # cast to 256 bits
....:     x = z_bin[0]*2^0 + z_bin[1]*2^1 + z_bin[2]*2^2+z_bin[3]*2^3
....:     return (x,z_bin[4])
sage: try_hash('1110010000','0')
\end{sagecommandline}

As we can see, our first attempt to hash into $\F_{13}$ was not successful, as $15$ is not an element in $\F_{13}$, so we increment the binary counter by $1$ and try again: 
\begin{sagecommandline}
sage: try_hash('1110010000','1')
\end{sagecommandline}

With this try, we found a hash into $\F_{13}$. However, this point is not guaranteed to define a curve point. To see that, we insert $x=3$ into the right side of the \concept{short Weierstrass} equation of the \curvename{Tiny-jubjub} curve, and compute $3^3 + 8\cdot 3 + 8 = 7$. However, $7$ is not a quadratic residue in $\F_{13}$, since $7^{\frac{13-1}{2}}=7^6=12=-1$. This means that the field element $7$ is a not suitable as the $x$-coordinate of any curve point. We therefore have to increment the counter another time: 
\begin{sagecommandline}
sage: try_hash('1110010000','10')
\end{sagecommandline}
Since $12^3 + 8\cdot 12 + 8 = 12$, and we have $\sqrt{12} = \{5, 8\}$, we finally found the valid $x$-coordinate $x=12$ for a curve point hash. Now, since the auxiliary bit of this hash is $1$, we choose the larger root $y=8$ as the $y$ coordinate and get the following hash which is a valid curve point on the \curvename{Tiny-jubjub} curve:
$$
H_{TJJ\_13}(<1,1,1,0,0,0,0,0>) = (12,8)
$$

In order to project this onto the ``large'' prime-order subgroup, we have to do cofactor clearing, that is, we have to multiply the point with the cofactor $4$. Using sage we get
\begin{sagecommandline}
sage: P = TJJ_13(12,8)
sage: (4*P).xy()
\end{sagecommandline}

This implies that hashing the binary string $<1,1,1,0,0,0,0,0>$ onto the large prime order subgroup $TJJ\_13[5]$ gives the hash value $(8,8)$ as a result. 
$$
H_{TJJ\_13[5]}(<1,1,1,0,0,0,0,0>) = (8,8)
$$
\end{example}
\begin{exercise}
Use our definition of the $try\_hash$ algorithm to implement a hash function $H_{TJJ\_13[5]} : \{0,1\}^*\to TJJ\_13(\F_{13})[5]$ that maps binary strings of arbitrary length onto the $5$-torsion group of $TJJ13(\F_{13})$. 
\end{exercise}
\begin{exercise}
Implement a cryptographic hash function $H_{secp256k1} : \{0,1\}^*\to secp256k1$ that maps binary strings of arbitrary length onto the elliptic curve \curvename{secp256k1}. 
\end{exercise}
\section{Constructing elliptic curves} Cryptographically secure elliptic curves like \curvename{secp256k1} \ref{secp256k1} have been known for quite some time. Given the latest advancements in cryptography, however, it is often necessary to design and instantiate elliptic curves from scratch that satisfy certain very specific properties. 

For example, in the context of SNARK development, it became necessary to design elliptic curves that can be efficiently implemented inside of a so-called \uterm{algebraic circuit} in order to enable primitives like elliptic curve \uterm{signature schemes} in a zero-knowledge proof. Such a curve is given by the Baby-jubjub curve as defined in \cite{whitehat-21}, and we have paralleled its definition by introducing the \curvename{Tiny-jubjub} curve from \examplename{} \ref{TJJ13}. As we have seen, those curves are instances of so-called \concept{twisted Edwards} curves, and as such have easy to implement addition laws that work without branching. However, we introduced the \curvename{Tiny-jubjub} curve out of thin air, as we just gave the curve parameters without explaining how we came up with them.

Another requirement in the context of many so-called \term{pairing-based zero-knowledge proofing systems} is the existence of a suitable, pairing-friendly curve with a specified security level and a low embedding degree as defined in \ref{def:embedding-degree}. Famous examples are the BLS\_12 and the NMT curves.\sme{add references}

The major goal of this section is to explain the most important method of designing elliptic curves with predefined properties from scratch, called the \term{\concept{complex multiplication method}} (cf. \chaptname{} 6 of \cite{silverman-1994}). We will apply this method in section \ref{BLS6} to synthesize a particular BLS6 curve, which is one of the most insecure curves, that is particular well suited to serve  as the main curve to build our pen-and-paper SNARKs on. As we will see, this curve has a ``large'' prime factor subgroup of order $13$, which implies that we can use our \curvename{Tiny-jubjub} curve to implement certain elliptic curve cryptographic primitives in circuits over that BLS6 curve. 
 
Before we introduce the \concept{complex multiplication method}, we have to explain a few properties of elliptic curves that are of key importance in understanding that method. 

\subsection{The Trace of Frobenius} To understand the \concept{complex multiplication method} of elliptic curves, we have to define the so-called \term{trace} of an elliptic curve first.

We know that elliptic curves are cyclic groups of finite order. Therefore, an interesting question is whether it is possible to estimate the number of elements that this curve contains. Since an affine \concept{short Weierstrass} curve consists of pairs $(x,y)$ of elements from a finite field $\F_q$ plus the point at infinity, and the field $\F_q$ contains $q$ elements, the number of curve points cannot be arbitrarily large, since it can contain at most $q^2+1$ many elements. 

There is however, a more precise estimation, usually called the \term{Hasse bound}. To understand it, let $E(\F_q)$ be an affine \concept{short Weierstrass} curve over a finite field $\F_q$ of order $q$, and let $|E(\F_q)|$ be the order of the curve. Then there is an integer $t\in \Z$, called the \term{trace of Frobenius} of the curve, such that $|t| \leq 2\sqrt{q}$ and the following equation holds:
\begin{equation}\label{hasse-bound}
|E(\F)| = q +1 -t
\end{equation}

A positive trace, therefore, implies that the curve contains no more points than the underlying field, whereas a non-negative trace means that the curve contains more points. However, the estimation $|t| \leq 2\sqrt{q}$ implies that the difference is not very large in either direction, and the number of elements in an elliptic curve is always approximately in the same order of magnitude as the size of the curve's base field.

\begin{example}\label{ex:E1F5-frobenius} Consider the elliptic curve $E_{1,1}(\F_5)$ from \examplename{} \ref{E1F5}. We know that it contains $9$ curve points. Since the order of $\F_5$ is $5$, we compute the trace of $E_{1,1}(\F_5)$ to be $t=-3$, since the Hasse bound is given by the following equation:
$$
9 = 5 + 1 - (-3)
$$
Indeed, we have $|t| \leq 2\sqrt{q}$, since $\sqrt{5}> 2$ and 
$|-3|= 3 \leq 4 = 2\cdot 2< 2\cdot \sqrt{5}$.
\end{example}

\begin{example}\label{ex:TJJ13-frobenius} To compute the trace of the \curvename{Tiny-jubjub} curve, recall from \examplename{} \ref{TJJ13} that the order of \TJJ{} is $20$. Since the order of $\F_{13}$ is $13$, we can therefore use the Hasse bound and compute the trace as $t=-6$:
\begin{equation}
20 = 13 + 1 - (-6)
\end{equation}

Again, we have $|t| \leq 2\sqrt{q}$, since $\sqrt{13}> 3$ and 
$|-6|= 6 = 2\cdot 3< 2\cdot \sqrt{13}$.
\end{example}

\begin{example}\label{ex:secp256k1-trace}To compute the trace of \curvename{secp256k1}, recall from \examplename{} \ref{secp256k1} that this curve is defined over a prime field with $p$ elements, and that the order of that group is given by $r$:  
\begin{align*}
p &= \scriptstyle 115792089237316195423570985008687907853269984665640564039457584007908834671663\\
r &= \scriptstyle 115792089237316195423570985008687907852837564279074904382605163141518161494337
\end{align*}

Using the Hasse bound $r = p + 1 -t$, we therefore compute $t= p+1 -r$, which gives the trace of curve \curvename{secp256k1} as follows:
$$
t = \scriptstyle 432420386565659656852420866390673177327
$$

As we can see, \curvename{secp256k1} contains less elements than its underlying field. However,  the difference is tiny, since the order of \curvename{secp256k1} is in the same order of magnitude as the order of the underlying field. Compared to $p$ and $r$, the integer $t$ is tiny.

\begin{sagecommandline}
sage: p = 115792089237316195423570985008687907853269984665640564039457584007908834671663
sage: r = 115792089237316195423570985008687907852837564279074904382605163141518161494337
sage: t = p + 1 -r
sage: t.nbits()
sage: abs(RR(t)) <= 2*sqrt(RR(p))
\end{sagecommandline}
\end{example} 
\begin{exercise}
\label{exercise:BN128-trace}
Consider the curve \curvename{alt\_bn128} from \examplename{} \ref{BN128}. Write a Sage program that computes the trace of Frobenius for \curvename{alt\_bn128}. Does the curve contain more or less elements than its base field $\F_{bn128}$?
\end{exercise}

\subsection{The $j$-invariant}
\label{sec:j-inv} As we have seen in \ref{sec:isomorphic_curves} , two elliptic curves $E_1(\F)$ defined by $y^2 = x^3 + ax +b$ and $E_2(\F)$ defined by $y^2 + a'x + b'$ are strictly isomorphic if and only if there is a quadratic residue $d\in \F$ such that $a' = a d^2$ and $b' = b d^3$. 

There is, however, a more general way to classify elliptic curves over finite fields $\F_q$, based on the so-called \term{$j$-invariant} of an elliptic curve with $j(E(\F_q))\in\F_q$, as defined below:
\begin{equation}\label{eq:j-invariant1}
j(E(\F_q)) = \Zmod{1728\cdot \frac{4\cdot a^3}{4\cdot a^3+ 27\cdot b^2}}{q}
\end{equation}

A detailed description of the $j$-invariant is beyond the scope of this book. For our present purposes, it is sufficient to note that the $j$-invariant is an important tool to classify elliptic curves and it is needed in the \concept{complex multiplication method} to decide on an actual curve instantiation that implements abstractly chosen properties.

\begin{example} Consider the elliptic curve $E_{1,1}(\F_5)$ from \examplename{} \ref{E1F5}\sme{check reference}. We compute its $j$-invariant as follows:
\begin{align*}
j(E_{1,1}(\F_5)) &= \Zmod{1728\cdot \frac{4\cdot 1^3}{4\cdot 1^3+ 27\cdot 1^2}}{5}\\
             &= 3 \frac{4}{4+ 2}\\
             &= 3\cdot 4\\
             & = 2
\end{align*}
\end{example}
\begin{example} Consider the elliptic curve \TJJ{} from \examplename{} \ref{TJJ13}\sme{check reference}. We compute its $j$-invariant as follows:
\begin{align*}
j(TJJ\_13) &= \Zmod{1728\cdot \frac{4\cdot 8^3}{4\cdot 8^3+ 27\cdot 8^2}}{13}\\
             &= 12\cdot \frac{4\cdot 5}{4\cdot 5+ 1\cdot 12}\\
             &= 12\cdot \frac{7}{7+ 12}\\
             &= 12\cdot 7\cdot 6^{-1}\\
             &= 2\cdot 7\\
             &= 1 
\end{align*}
\end{example}
\begin{example}Consider \curvename{secp256k1} from \examplename{} \ref{secp256k1}. We compute its $j$-invariant using Sage: 
\begin{sagecommandline}
sage: p = 115792089237316195423570985008687907853269984665640564039457584007908834671663
sage: F = GF(p)
sage: j = F(1728)*((F(4)*F(0)^3)/(F(4)*F(0)^3+F(27)*F(7)^2))
sage: j == F(0)
\end{sagecommandline}
\end{example} 
\begin{exercise}
\label{exercise:BN128-j-inv}
Consider the curve \curvename{alt\_bn128} from \examplename{} \ref{BN128}. Write a Sage program that computes the $j$-invariant for \curvename{alt\_bn128}.
\end{exercise}
\subsection{The \concept{complex multiplication method}}\label{complex-multiplication-method}
As we have seen in the previous sections, elliptic curves have various defining properties, like their order, their prime factors, the embedding degree, or the order of the base field. The \term{\concept{complex multiplication method}} (CM) provides a practical way of constructing elliptic curves with pre-defined restrictions on the order of the curve and the base field.\footnote{A detailed explanation of the complex multiplication
method and its derivation can be found, for example, in \cite{grech-2012}.}

% the detailed method is here https://arxiv.org/pdf/1207.6983.pdf
% http://users.uoa.gr/~kontogar/files/ElisavetDaras.pdf
% https://hal.inria.fr/inria-00075302/PDF/RR-1256.pdf
% https://www.ams.org/journals/mcom/2007-76-260/S0025-5718-07-01980-1/S0025-5718-07-01980-1.pdf
% https://graui.de/code/elliptic2/ //draw curves
% https://hal.inria.fr/inria-00075302/PDF/RR-1256.pdf proposition 2.1

The \concept{complex multiplication method} starts by choosing a base field $\F_{q}$ of the curve $E(\F_q)$ we want to construct such that $q = p^m$ for some prime number $p$, and  $m\in \N$. We assume $p>3$ to simplify things in what follows. 

Next, the trace of Frobenius $t\in \Z$ of the curve is chosen such that $q$ and $t$ are coprime, that is, $gcd(q,t)=1$ holds true and $|t|\leq 2\sqrt{q}$. The choice of $t$ also defines the curve's order $r$, since $r=p+1-t$ by the Hasse bound \eqref{hasse-bound}, so choosing $t$ will determine the large order subgroup as well as all small cofactors. The resulting $r$ must be such that the curve meets the application's security requirements. 

Note that the choice of $p$ and $t$ also determines the embedding degree $k$ of any prime-order subgroup of the curve, since $k$ is defined as the smallest number such that the prime order $n$ divides the number $p^k-1$.

In order for the \concept{complex multiplication method} to work, neither $q$ nor $t$ can be arbitrary, but must be chosen in such a way that two additional integers $D\in \Z$ and $v\in \Z$ exist and the following conditions hold:

\begin{equation}\label{eq:D-criteria}
\begin{split}
D<0\\
\Zmod{\left( D =0  \text{ or } D=1\right)}{4}\\
4q  = t^2 + |D|v^2 
\end{split}
\end{equation}

If such numbers exist, we call $D$ the \term{CM-discriminant}, and we know that we can construct a curve $E(\F_q)$ over a finite field $\F_q$ such that the order of the curve is $|E(\F_q)|= q+1-t$. In this case, it is the goal of the \concept{complex multiplication method} to actually construct such a curve, that is finding the parameters $a$ and $b$ from $\F_q$ in the defining Weierstrass equation such that the curve has the desired order $r$. 

Equation \ref{eq:D-criteria} has an infinite number of solutions and much research has been done to compute solution sets which result in elliptic curves with desired properties. A common approach is to fix the CM-discriminant and to define $q$, $t$ and $v$ as functions of a parameter $x$, such that the values $q(x)$, $t(x)$ and $v(x)$ give a solution to \ref{eq:D-criteria} for every parameter $x$. In fact many of those families are known under names like BLS \cite{bls-02} or NMT \cite{mnt-84} curves, indicating that the base field order $q$ as well as the trace of Frobenius $t$ of each member in such a family of curves are computed in similar ways. This is an approach taken for example in \cite{freeman-06} to compute pairing friendly elliptic curves.
\begin{example}[BLS6 curves]
\label{ex:bls6-params} To give a better understanding of how parameterized solution sets of equation \ref{eq:D-criteria} give rise to families of elliptic curves, we will look at a parametrization that was found by the authors Barreto, Lynn and Scott in 2002 \cite{bls-02}. Their approach gives rise to pairing friendly elliptic curves of various embedding degrees and in particular to curves of embedding degree $6$ with CM-discriminant $D=-3$. Members of those families are usually called BLS6 curves for short. 

To be more precise, let the polynomials $t,q\in \Q[x]$ be defined as $t(x)=x+1$ and $q(x) = \frac{1}{3}(x-1)^2(x^2-x+1) + x$. Then the following set defines a solution set to equation \ref{eq:D-criteria} for $D=-3$ and $v= \sqrt{(4 q - t^2)/3}$: 
\begin{equation}
PARAM(BLS6) = \{t(x), q(x) \;|\; x\in \NN \text{ and } q(x)\in\NN\}
\end{equation}  
\end{example}

Assuming that proper parameters $q$, $t$, $D$ and $v$ are found, we have to compute the so-called \term{Hilbert class polynomial} $H_D\in \Z[x]$ of the CM-discriminant $D$, which is a polynomial with integer coefficients. To do so, we first have to compute the following set:
\begin{multline*}
\label{def:hilbert_class_domain}
S(D)=\{(A,B,C)\;|\; A,B,C\in\Z,\; D = B^2-4AC,\; gcd(A,B,C)=1, \\
|B|\leq A \leq \sqrt{\frac{|D|}{3}},\; A\leq C, 
\text{ if } B< 0 \text{ then } |B| < A < C\}
\end{multline*}\tbds{this needs to be numbered}
One way to compute this set is to first compute the integer $A_{max}= Floor(\sqrt{\frac{|D|}{3}})$, then loop through all the integers $0\leq A\leq A_{max}$, as well as through all the integers $-A\leq B \leq A$ and check if there is an integer $C$ that satisfies the equation $D = B^2-4AC$ and the rest of the requirements from \ref{def:hilbert_class_domain}.

To compute the Hilbert class polynomial, the so-called \term{$j$-function} is needed, which is a complex function defined on the upper half $\mathbb{H}$ of the complex plane $\mathbb{C}$, usually written as follows:

\begin{equation}\label{eq:j-invariant2}
j: \mathbb{H} \to \mathbb{C}
\end{equation}

What this means is that the $j$-functions takes complex numbers $(x +y\cdot i)$ with a positive imaginary part $y>0$ as inputs and returns a complex number $j(x+i\cdot y)$ as a result.

The $j$-function is closely related to the $j$-invariant \ref{sec:j-inv} of an elliptic curve. However for the purposes of this book, it is not important to understand the $j$-function in detail. We can use Sage to compute it in a similar way that we would use Sage to compute any other well-known function, like the square root. It should be noted, however, that the computation of the $j$-function in Sage is sometimes prone to precision errors. For example, the $j$-function has a root in $\frac{-1+i\sqrt{3}}{2}$, which Sage only approximates. Therefore, when using Sage to compute the $j$-function, we need to take precision loss into account and possibly round to the nearest integer.

\begin{sagecommandline}
sage: z = ComplexField(100)(0,1)
sage: z # (0+1i)
sage: elliptic_j(z)
sage: # j-function only defined for positive imaginary arguments
sage: z = ComplexField(100)(1,-1)
sage: try:
....:     elliptic_j(z)
....: except PariError:
....:     pass
sage: # root at (-1+i sqrt(3))/2
sage: z = ComplexField(100)(-1,sqrt(3))/2
sage: elliptic_j(z)
sage: elliptic_j(z).imag().round()
sage: elliptic_j(z).real().round()
\end{sagecommandline}

With a way to compute the $j$-function and the precomputed set $S(D)$ at hand, we can now compute the Hilbert class polynomial as follows:
\begin{equation}
H_D(x) = \Pi_{(A,B,C)\in S(D)} \left(x - j\left(\frac{-B + \sqrt{D}}{2A}\right)\right)
\end{equation}

In other words, we loop over all elements $(A,B,C)$ from the set $S(D)$ and compute the $j$-function at the point $\frac{-B + \sqrt{D}}{2A}$, where $D$ is the CM-discriminant that we decided in a previous step. The result defines a factor of the Hilbert class polynomial and all factors are multiplied together.

It can be shown that the Hilbert class polynomial is an integer polynomial, but actual computations need high-precision arithmetic to avoid approximation errors that usually occur in computer approximations of the $j$-function (as shown above). So, in case the calculated Hilbert class polynomial does not have integer coefficients, we need to round the result to the nearest integer. Given that the precision we used was sufficiently high, the result will be correct.

In the next step, we use the Hilbert class polynomial $H_D\in \Z[x]$, and project it to a polynomial $H_{D,q}\in\F_q[x]$ with coefficients in the base field $\F_q$ as chosen in the first step. We do this by simply reducing the coefficients modulo $p$, that is, if $H_D(x)= a_mx^m +a_{m-1}x^{m-1}+\ldots + a_1 x + a_0$, we compute the $q$-modulus of each coefficient
$\tilde{a}_j = \Zmod{a_j}{p}$, which yields the \term{projected Hilbert class polynomial} as follows:
$$
H_{D,p}(x)=\tilde{a}_mx^m +\tilde{a}_{m-1}x^{m-1}+\ldots + \tilde{a}_1 x + \tilde{a}_0
$$
We then search for roots of $H_{D,p}$, since every root $j_0$ of $H_{D,p}$ defines a family of elliptic curves over $\F_q$, which all have a $j$-invariant \ref{eq:j-invariant2} equal to $j_0$. We can pick any root, since all of them define an elliptic curve. However, some of the curves with the correct $j$-invariant might have an order different from the one we initially decided on. Therefore, we need a way to decide on a curve with the correct order. 

To compute a curve with the correct order, we have to distinguish a few different cases based on our choice of the root $j_0\in \F_q$ and of the CM-discriminant $D\in \Z$. If $j_0\neq 0$ or $j_0\neq \Zmod{1728}{q}$, we compute $c_1=\frac{j_0}{(\Zmod{1728}{q}) -j_0}\in \F_q$, then we chose some arbitrary quadratic non-residue $c_2\in \F_q$, and some arbitrary cubic non-residue $c_3\in \F_q$. 

The following list is guaranteed to define a curve with the correct order $r= q+1 -t$ for the fields order $q$ and the trace of Frobenius $t$ we initially decided on:

\begin{itemize}
\label{def:curve-order-frobenius}
\item Case $j_0 \neq 0 $ and $j_0\neq \Zmod{1728}{q}$. A curve with the correct order is defined by one of the following equations:
\begin{equation}
y^2 = x^3 + 3c_1x + 2c_1 \text{\;\; or \;\; } y^2 = x^3 + 3c_1c_2^2x + 2c_1c_2^3
\end{equation}
\item Case $j_0 = 0 $ and $D\neq -3$. A curve with the correct order is defined by one of the following equations:
\begin{equation}
y^2 = x^3 + 1 \text{\;\; or \;\; } y^2 = x^3 + c_2^3
\end{equation}
\item Case $j_0 = 0 $ and $D= -3$. A curve with the correct order is defined by one of the following equations:
\begin{align*}
y^2 = x^3 +1 & \text{\;\; or \;\; } y^2 = x^3 + c_2^3 \text{ \;\; or}\\  
y^2 = x^3 + c_3^2 & \text{\;\; or \;\; } y^2 = c_3^2 c_2^3 \text{\;\; or}\\
y^2 = x^3 + c_3^{-2} & \text{\;\; or \;\; }  y^2 = x^3 + c_3^{-2}c_2^3 
\end{align*}
\item Case $j_0 = \Zmod{1728}{q} $ and $D\neq -4$. A curve with the correct order is defined by one of the following equations:
\begin{equation}
y^2 = x^3 + x \text{\;\; or \;\; } y^2 = x^3 + c_2^2x
\end{equation}
\item Case $j_0 = \Zmod{1728}{q} $ and $D= -4$. A curve with the correct order is defined by one of the following equations:
\begin{align*}
y^2 = x^3 +x & \text{\;\; or \;\; } y^2 = x^3 + c_2x \text{ \;\; or}\\  
y^2 = x^3 + c_2^2x & \text{\;\; or \;\; } y^2 = x^3 + c_2^3x
\end{align*}
\end{itemize} 

To decide the proper defining \concept{short Weierstrass} equation, we therefore have to compute the order of any of the potential curves above, and then choose the one that fits our initial requirements. 

To summarize, using the \concept{complex multiplication method}, it is possible to synthesize elliptic curves with predefined order over predefined base fields from scratch. However, the curves that are constructed this way are just some representatives of a larger class of curves, all of which have the same order. Therefore, in real-world applications, it is sometimes more advantageous to choose a different representative from that class. To do so recall from \ref{sec:isomorphic_curves} that any curve defined by the \concept{short Weierstrass} equation $y^2 = x^3 + ax + b$ is isomorphic to a curve of the form $y^2 = x^3 + ad^4 x + bd^6$ for some invertible field element $d\in \F^*_q$. 

In order to find a suitable representative (e.g. with small parameters $a$ and $b$), the curve designer might choose an invertible field element $d$ such that the transformed curve has small parameters.

\begin{example} Consider curve $E_{1,1}(\F_5)$ from \examplename{} \ref{E1F5}. We want to use the \concept{complex multiplication method} to derive that curve from scratch. Since $E_{1,1}(\F_5)$ is a curve of order $r=9$ over the prime field of order $q=5$, we know from \examplename{} \ref{ex:E1F5-frobenius} that its trace of Frobenius is $t=-3$, which also shows that $q$ and $t$ are coprime. 

We then have to find parameters $D,v\in\Z$ such that the criteria in \ref{eq:D-criteria} hold. We get the following:
\begin{align*}
4q & = t^2+ |D|v^2 & \Rightarrow \\
20 & = (-3)^2 + |D|v^2 & \Leftrightarrow \\
11 & = |D|v^2
\end{align*}
Now, since $11$ is a prime number, the only solution is $|D|=11$ and $v=1$ here. With $D=-11$ and  the Euclidean division of $-11$ by $4$ being $-11 = -3\cdot 4 +1$, we have $\Zmod{-11}{4}=1$, which shows that $D=-11$ is a proper choice.

In the next step, we have to compute the Hilbert class polynomial $H_{-11}$. To do so, we first have to find the set $S(D)$. To compute that set, observe that, since $\sqrt{\frac{|D|}{3}}\approx 1.915<2$, we know from $A\leq \sqrt{\frac{|D|}{3}}$ and $A\in\Z$ as well as $0\leq |B|\leq A$ that $A$ must be either $0$ or $1$. 

For $A=0$, we know $B=0$ from the constraint $|B|\leq A$. However, in this case, there could be no $C$ satisfying $-11= B^2 -4AC$. So we try $A=1$ and deduce $B\in\{-1,0,1\}$ from the constraint $|B|\leq A$. The case $B=-1$ can be excluded, since then $B<0$ has to imply $|B|<A$. The case $B=0$ can also be excluded, as there cannot be an integer $C$ with $-11 = -4C$, since $11$ is a prime number. 

This leaves the case $B=1$, and we compute $C=3$ from the equation $-11 = 1^2 -4C$, which gives the solution $(A,B,C)=(1,1,3)$. Hence:
$$
S(D)=\{(1,1,3)\}
$$

With the set $S(D)$ at hand, we can compute the Hilbert class polynomial of $D=-11$. To do so, we have to insert the term $\frac{-1+\sqrt{-11}}{2\cdot1}$ into the $j$-function. To do so, first observe that $\sqrt{-11}=i\sqrt{11}$, where $i$ is the imaginary unit, defined by $i^2=-1$. Using this, we use Sage to compute the $j$-invariant and get the following:
$$
H_{-11}(x) = x - j\left(\frac{-1+i\sqrt{11}}{2}\right) = x + 32768
$$

As we can see, in this particular case, the Hilbert class polynomial is a linear function with a single integer coefficient. In the next step, we have to project it onto a polynomial from $\F_5[x]$ by reducing the coefficients $1$ and $32768$ modulo $5$. We get $\Zmod{32768}{5}=3$, so the projected Hilbert class polynomial is
$$
H_{-11,5}(x)=x+3
$$ 
As we can see, the only root of this polynomial is $j=2$, since $H_{-11,5}(2)=2+3=0$. We therefore have a situation with $j\neq 0$ and $j\neq 3 = \Zmod{1728}{5}$, which tells us that we have to consider the first case in \ref{def:curve-order-frobenius} and compute the parameter $c_1$, where in our case division is done in in modular $5$ arithmetic:
\begin{equation}
c_1 = \frac{2}{\Zmod{1728}{5}-2} = \frac{2}{3-2}= 2
\end{equation}
In order to decide the correct equation from the first case in \ref{def:curve-order-frobenius}, we have to check if the curve $E(\F_5)$ defined by the \concept{short Weierstrass} equation  $y^2 =  x^3 + 3\cdot c_1 x + 2\cdot c_1 = x^3 + x + 4$ has the correct order. We use Sage, and find that the order is indeed $9$, so it is a curve with the required parameters. Thus, we have successfully constructed the curve with the desired properties.

Comparing our constructed curve $y^2 =  x^3 + x + 4$ to the definition of $E_{1,1}(\F_5)$ from \examplename{} \ref{E1F5}, we see that the defining equations are different. However, since both curves are of the same order, we know from \ref{sec:isomorphic_curves} that they are isomorphic. In fact we can use \ref{sec:isomorphic_curves} and the quadratic residue $4\in \F_5$, to transform the curve defined by $y^2 = x^3 +x+4$ into the curve $y^2 = x^3 + 4^2 + 4\cdot 4^3$ which gives the defining equation of $E_{1,1}(\F_5)$:
$$
y^2 = x^3 + x +1
$$
Thus, using the \concept{complex multiplication method}, we were able to derive a curve with specific properties from scratch.
\end{example}

\begin{example}
 Consider the \curvename{Tiny-jubjub} curve \TJJ{} from \examplename{} \ref{TJJ13}. We want to use the \concept{complex multiplication method} to derive that curve from scratch. We know from \examplename{} \ref{TJJ13} that \TJJ{} is a curve of order $r=20$ over the prime field of order $q=13$ and we know from \examplename{} \ref{ex:TJJ13-frobenius} that its trace of Frobenius is $t=-6$. This shows that $q$ and $t$ are coprime. 

In order to apply the \concept{complex multiplication method} we have to find parameters $D,v\in\Z$ such that \ref{eq:D-criteria} holds. We get the following:
\begin{align*}
4q & = t^2+ |D|v^2 & \Rightarrow \\
4\cdot 13 & = (-6)^2+ |D|v^2 & \Rightarrow \\
52 & = 36 + |D|v^2 & \Leftrightarrow \\
16 & = |D|v^2
\end{align*}

This equation has four solutions for $(D,v)$, namely $(-4,\pm 2)$ and $(-16,\pm 1)$. Looking at the first two solution, we show in exercise \ref{ex:Low_order_Hilbert_polys} that $D=-4$ implies $j=1728$, and from \ref{def:curve-order-frobenius} we know that in this case the constructed curve is defined by a \concept{short Weierstrass} equation \ref{def_short_weierstrass_curve} that has a vanishing parameter $b=0$. We can therefore conclude that $D=-4$ will not help us reconstructing \TJJ{}, since $D=-4$ will produce curves of order $20$, but all of those curves have $b=0$.

We therefore consider the third and fourth solution for $D=-16$. In the next step, we have to compute the Hilbert class polynomial $H_{-16}$. To do so, we first have to find the set $S(D)$. To compute that set, observe that since $\sqrt{\frac{|-16|}{3}}\approx 2.31<3$, we know from $A\leq \sqrt{\frac{|-16|}{3}}$ and $A\in\Z$ with $0\leq |B|\leq A$, that $A$ must be in the range $0..2$. So we loop through all possible values of $A$ and through all possible values of $B$ under the constraints $|B|\leq A$, and if $B<0$ then $|B|<A$.
Then we compute potential $C$'s from the equation $-16 = B^2 -4AC$. We get the following two solutions for $S(D)$:
% sage has precomputed Hilbert class polynomials 
% https://doc.sagemath.org/html/en/reference/databases/sage/databases/db_class_polynomials.html
$$
S(D)=\{(1,0,4),(2,0,2)\}
$$
With the set $S(D)$ at hand, we can compute the Hilbert class polynomial of $D=-16$. We can use Sage to compute the $j$-invariant and get the following:
\begin{align*}
H_{-16}(x) &= \left(x - j\left(\frac{i\sqrt{16}}{2}\right)\right)
 \left(x - j\left(\frac{i\sqrt{16}}{4}\right)\right) \\
           &= (x- 287496)(x-1728)
\end{align*}

As we can see, in this particular case, the Hilbert class polynomial is a quadratic function with two integer coefficients. In the next step, we have to project it onto a polynomial over $\F_{13}$ by computing the modular $13$ remainder of the coefficients $287496$ and $1728$. We get $\Zmod{287496}{13}=1$ and $\Zmod{1728}{13}=12$, which means that the projected Hilbert class polynomial is as follows:
$$
H_{-11,5}(x)=(x-1)(x-12)= (x+12)(x+1)
$$ 
This is considered a polynomial from $\F_{13}[x]$. Thus, we have two roots, namely $j=1$ and $j=12$. We already know that $j=12$ is the wrong root to construct the \curvename{Tiny-jubjub} curve, since $\Zmod{1728}{13}=12$, and that case is not compatible with a curve with $b\neq 0$. So we choose $j=1$.

Another way to decide the proper root is to compute the $j$-invariant of the \curvename{Tiny-jubjub} curve. We get the following:
\begin{align*}
j(\mathit{TJJ\_13}) & = 12\frac{4\cdot 8^3}{4\cdot 8^3+ 1\cdot 8^2}\\
                    & = 12\frac{4\cdot 5}{4\cdot 5+ 12}\\
                    & = 12\frac{7}{7+ 12}\\
                    & = 12\frac{7}{7-1}\\
                    & = 1
\end{align*}

This is equal to the root $j=1$ of the Hilbert class polynomial $H_{-16,13}$ as expected. We therefore have a situation with $j\neq 0$ and $j\neq 1728$, which tells us that we have to consider the first case in \ref{def:curve-order-frobenius} and compute the parameter $c_1$, where in our case division is done in in modular $13$ arithmetic:
$$
c_1=\frac{1}{12-1} = \frac{1}{11} =6
$$
In order to decide the correct equation from the first case in \ref{def:curve-order-frobenius}, we have to check if the curve $E(\F_{13})$ defined by the \concept{short Weierstrass} equation  $y^2 = x^3 + 3\cdot 6 x + 2\cdot 6 = x^3 + 5x +12$ has the correct order. We use Sage and find that the order is $8$ not $20$ as expected, which implies that the trace of this curve is $6$, not $-6$. So we have to consider the second equation from the first case in \ref{def:curve-order-frobenius}, and choose some quadratic non-residue $c_2\in\F_{13}$. We choose $c_2=5$ and compute the \concept{short Weierstrass} equation $y^2 = x^3 + 5 c_2^2 x + 12 c_2^3$ as follows:
$$
y^2 = x^3 + 8 x + 5
$$
We use Sage and find that the order is $20$, which is indeed the correct one. Comparing our constructed curve $y^2 =  x^3 + 8x + 5$ to the definition of \TJJ{} from \examplename{} \ref{TJJ13}, we see that the defining equations are different. However, since both curves are of the same order, we know from \ref{sec:isomorphic_curves} that they are isomorphic.

In fact we can use \ref{sec:isomorphic_curves} and the quadratic residue $12\in \F_{13}$, to transform the curve defined by $y^2 =  x^3 + 8x + 5$ into the curve $y^2 = x^3 + 12^2\cdot 8 + 5\cdot 12^3$ which gives the following:
$$
y^2 = x^3 + 8x +8
$$

This is the \curvename{Tiny-jubjub} curve that we used extensively throughout this book. So using the \concept{complex multiplication method}, we were able to derive a curve with specific properties from scratch.
\end{example}

\begin{example} To consider a real-world example, we want to use the \concept{complex multiplication method} in combination with Sage to compute \curvename{secp256k1} from scratch. So based on \examplename{} \ref{secp256k1}, we decide to compute an elliptic curve over a prime field $\F_p$ of order $r$ for the following security parameters:
\begin{align*}
p &= \scriptstyle 115792089237316195423570985008687907853269984665640564039457584007908834671663\\
r &= \scriptstyle 115792089237316195423570985008687907852837564279074904382605163141518161494337
\end{align*}
According to \examplename{} \ref{ex:secp256k1-trace}, this gives the following trace of Frobenius for any curve isomorphic to \curvename{secp256k1}:
$$t = \scriptstyle 432420386565659656852420866390673177327$$ 

We also decide that we want a curve of the form $y^2 = x^3 + b$, that is, we want the parameter $a$ to be zero. Table \ref{def:curve-order-frobenius} then implies that the $j$-invariant of our curve must be zero.

In a first step, we have to find a CM-discriminant $D$ and some integer $v$ such that the equation 
$
4p = t^2 +|D|v^2
$
is satisfied. Since we aim for a vanishing $j$-invariant, the first thing to try is $D=-3$. In this case, we can compute $v^2 = (4p -t^2)/|D|$, and if $v^2$ happens to be an integer that has a square root $v$, we are done. Invoking Sage we compute as follows:
\begin{sagecommandline}
sage: D = -3
sage: p = 115792089237316195423570985008687907853269984665640564039457584007908834671663
sage: r = 115792089237316195423570985008687907852837564279074904382605163141518161494337
sage: t = p+1-r
sage: v_sqr = (4*p - t^2)/abs(D)
sage: v_sqr.is_integer()
sage: v = sqrt(v_sqr)
sage: v.is_integer()
sage: 4*p == t^2 + abs(D)*v^2
sage: v
\end{sagecommandline}
The pair $(D,v)=(-3, 303414439467246543595250775667605759171)$ does indeed solve the equation, which tells us that there is a curve of order $r$ over a prime field of order $p$, defined by a \concept{short Weierstrass} equation $y^2 = x^3 + b$ for some $b\in \F_p$. Now we need to compute $b$.

For $D=-3$, we show in exercise \ref{ex:Low_order_Hilbert_polys} that the associated Hilbert class polynomial is given by $H_{-3}(x)=x$, which gives the projected Hilbert class polynomial as 
$H_{-3,p}=x$ and the $j$-invariant of our curve is guaranteed to be $j=0$. Now, looking at \ref{def:curve-order-frobenius}\sme{check reference}, we see that there are $6$ possible cases to construct a curve with the correct order $r$. In order to construct the curves in question, we have to choose some arbitrary quadratic and cubic non-residue. So we loop through $\F_p$ to find them, invoking Sage:

\begin{sagecommandline}
sage: F = GF(p)
sage: for c2 in F:
....:     try: # quadratic residue
....:         _ = c2.nth_root(2)
....:     except ValueError: # quadratic non-residue
....:         break
sage: c2
sage: for c3 in F:
....:     try:
....:         _ = c3.nth_root(3)
....:     except ValueError:
....:         break
sage: c3
\end{sagecommandline}

We found the quadratic non-residue $c_2=3$ and the cubic non-residue $c_3=2$. Using those numbers, we check the six cases against the the expected order $r$ of the curve we want to synthesize:
\begin{sagecommandline}
sage: C1 = EllipticCurve(F,[0,1])
sage: C1.order() == r
sage: C2 = EllipticCurve(F,[0,c2^3])
sage: C2.order() == r
sage: C3 = EllipticCurve(F,[0,c3^2])
sage: C3.order() == r
sage: C4 = EllipticCurve(F,[0,c3^2*c2^3])
sage: C4.order() == r
sage: C5 = EllipticCurve(F,[0,c3^(-2)])
sage: C5.order() == r
sage: C6 = EllipticCurve(F,[0,c3^(-2)*c2^3])
sage: C6.order() == r
\end{sagecommandline}

As expected, we found an elliptic curve of the correct order $r$ over a prime field of size $p$. In principle. we are done, as we have found a curve with the same basic properties as \curvename{secp256k1}. However, the curve is defined by the following equation, which uses a very large parameter $b_1$, and so it might perform too slowly in certain algorithms and is not very convenient for humans to handle.
$$
\scriptstyle y^2 = x^3 + 86844066927987146567678238756515930889952488499230423029593188005931626003754
$$
It might therefore be advantageous to find an isomorphic curve with the smallest possible parameter $b_2$. In order to find such a $b_2$, we can use  \ref{sec:isomorphic_curves} and choose an invertible, quadratic residue $d$ such that $b_2 = b_1\cdot d^3$ is as small as possible. To do so, we rewrite the last equation into the following form:
$$
d = \sqrt[3]{\frac{b_2}{b_1}}
$$ 

Then we use Sage to loop through values $b_2\in \F_p$ until it finds some number such that the quotient $\frac{b_2}{b_1}$ has a cube root $d$ and this cube root itself is a quadratic residue. 
\begin{sagecommandline}
sage: b1=F(86844066927987146567678238756515930889952488499230423029593188005931626003754)
sage: for b2 in F:
....:     if b2 == 0: continue
....:     try:
....:         d = (b2/b1).nth_root(3)
....:         _ = d.nth_root(2) # test
....:     except ValueError: continue
....:     break # found it
sage: b2
\end{sagecommandline}
Indeed, the smallest possible value is $b_2=7$ and the defining \concept{short Weierstrass} equation of a curve over $\F_p$ with prime order $r$ is 
$
y^2 = x^3 + 7
$,
which we might call \curvename{secp256k1}. As we have just seen, the \concept{complex multiplication method} is powerful enough to derive cryptographically secure curves like \curvename{secp256k1} from scratch.
\end{example}
%\section{twists}
%I think this has to wait for Volume 2, due to timing constraints...
\begin{exercise}
\label{ex:Low_order_Hilbert_polys} Show that the Hilbert class polynomials for the CM-discriminants $D=-3$ and $D=-4$ are given by  $H_{-3,q}(x)=x$ and $H_{-4,q}= x-(\Zmod{1728}{q})$.
\end{exercise}
\begin{exercise}
Use the complex multiplication method to construct an elliptic curve of order $7$ over the prime field $\F_{13}$.
\end{exercise}
\begin{exercise}
Use the complex multiplication method to compute all isomorphism classes of all elliptic curves of order $7$ over the prime field $\F_{13}$.
\end{exercise}
\begin{exercise}
\label{exercise:BN128-cm-method}
Consider the prime modulus $p$ of curve \curvename{alt\_bn128} from \examplename{} \ref{BN128} and its trace $t$ from exercise \ref{exercise:BN128-cm-method}. Use the complex multiplication method to synthesize an elliptic curve over $F_p$ that is isomorphic to \curvename{alt\_bn128} and compute an explicit isomorphism between these two curves.
\end{exercise}
\subsection{The $BLS6\_6$ pen-and-paper curve}\label{BLS6}

% https://arxiv.org/pdf/1207.6983.pdf
% construction 6.6 in https://eprint.iacr.org/2006/372.pdf
In this paragraph, we summarize our understanding of elliptic curves to compute the main pen-and-paper example for the rest of the book. To do so, we want to use the \concept{complex multiplication method} to derive a pairing-friendly elliptic curve that has similar properties to curves that are used in actual cryptographic protocols. However, we design the curve specifically to be useful in pen-and-paper examples, which mostly means that the curve should contain only a few points so that we are able to derive exhaustive addition and pairing tables. Specifically, we use construction 6.6 in \cite{freeman-2020}.

A well-understood family of pairing-friendly curves is given by the set of BLS curves. As explained in example \ref{ex:bls6-params}, in \cite{bls-02} the authors Barreto, Lynn and Scott found a parameterized solution set to equation \ref{eq:D-criteria}, which gives rise to pairing friendly elliptic curves of various embedding degrees with CM-discriminant $D = -3$. 

Most real world BLS curves have an embedding degree of $12$, however this degree is to large for a convenient pen-and-paper curve. Fortunately BLS curves of embedding degrees $k$ that satisfies $\kongru{k}{0}{6}$, are computed in a similar fashion and since the smallest embedding degree $k$ that satisfies this congruency is $k=6$, we aim for a BLS curve of embedding degree 6 as our main pen-and-paper example. We call such a curve a $BLS6$ curve, since it is a convention to note the embedding degree right after a descriptor that gives a hint of how the curve was constructed.  

\subsubsection{The Construction}
To apply the \concept{complex multiplication method} \ref{complex-multiplication-method} , recall that it starts with a choice of a base field $\F_{p^m}$, as well as a trace of Frobenius $t$ and a CM-discriminant $D$ of the curve. In the case of $BLS$ curves, the parameter $m$ is chosen to be $1$, which means that the curves are defined over prime fields. In addition the CM-discriminant is chosen to be $D=-3$, which implies that the curve is defined by the equation $y^2 = x^3 +b$ for some $b\in \F_p$. 

As shown in example \ref{ex:bls6-params}, for $BLS6$ curves, the relevant parameters $p$ and $t$ are themselves parameterized by the following functions:
\begin{equation}
\begin{split}
t(x) &= x+1\\
p(x) &= \frac{1}{3}(x-1)^2(x^{2}-x+1) +x
\end{split}
\end{equation}
Here $x\in\NN$ is a parameter that the designer has to choose in such a way that the evaluation of $p$ and $t$ at the point $x$ gives integers that have the proper size to meet the security requirements of the curve to be synthesized. It is guaranteed that there is an integer $v\in\Z$, such that equation \ref{eq:D-criteria} hold for the CM-discriminant $D=-3$, which implies that the \concept{complex multiplication method} can be used to compute a field element $b\in\F_p$ such that the elliptic curve $y^2 = x^3 +b$ has order $r=p+1-t$, CM-discriminant $D=-3$ and embedding degree $k=6$. 

In order to design the smallest BLS6 curve, we therefore have to find a parameter $x$ such that $t(x)$ and $p(x)$ are the smallest natural numbers that satisfy $p(x)>3$.\footnote{The smallest BLS curve will also be the most insecure BLS curve. However, since our goal with this curve is ease of pen-and-paper computation rather than security, it fits the purposes of this book.}

We therefore initiate the design process of our $BLS6$ curve by inserting small values for $x$ into the defining polynomials $t$ and $q$. We get the following results:
$$
\begin{array}{lcr}
x=1 & (t(x),p(x)) & (2,1)\\
x=2 & (t(x),p(x)) & (3,3)\\
x=3 & (t(x),p(x)) & (4,\frac{37}{3})\\
x=4 & (t(x),p(x)) & (5,43)\\
\end{array}
$$
Since $p(1)=1$ is not a prime number, the first $x$ that gives a proper curve is $x=2$. However, such a curve would be defined over a base field of characteristic $3$, and we would rather like to avoid that, since we only defined elliptic curves over fields of characteristics larger then $3$ in this book. We therefore find $x=4$, which defines a curve over the prime field of characteristic $43$ that has a trace of Frobenius $t=5$. 

Since the prime field $\F_{43}$ has $43$ elements and $43$'s binary representation is $43_2= 101011$, which consists of $6$ digits, the name of our pen-and-paper curve should be $BLS6\_6$, since its is common to name a BLS curve by its embedding degree and the bit-length of the modulus in the base field. We call $BLS6\_6$ the \term{moon-math-curve}.

Based on \ref{hasse-bound}\sme{check reference}, we know that the Hasse bound implies that $BLS6\_6$ will contain exactly $39$ elements. Since the prime factorization of $39$ is $39=3\cdot 13$, we have a ``large'' prime factor group of size $13$, and a small cofactor of $3$. Fortunately, a subgroup of order $13$ is well suited for our purposes, as $13$ elements can be easily handled in the associated addition, scalar multiplication and pairing tables in a pen-and-paper style. 

We can check that the embedding degree is indeed $6$ as expected, since $k = 6$ is the smallest number $k$ such that $r=13$ divides $43^k-1$. 
\begin{sagecommandline}
sage: k= 0
sage: for k in range(1,42): # Fermat's little theorem
....:     if (43^k-1)%13 == 0:
....:         break
sage: k
\end{sagecommandline}

To see that equation \ref{eq:D-criteria} indeed has a solution for the parameters $D=-3$, $p=43$ and $t=5$ as expected, we compute as follows:  
\begin{align*}
4p & = t^2 + |D|v^2 & \Rightarrow \\ 
4\cdot 43 & = 5^2 + 3\cdot v^2 & \Leftrightarrow \\ 
172 & = 25 + 3 v^2 & \Leftrightarrow \\ 
49 & = v^2 & \Leftarrow \\
v & = \pm 7
\end{align*}
This implies that we can use the \concept{complex multiplication method} as described in \ref{complex-multiplication-method} in order to compute the defining equation $y^2=x^3 + ax + b$ of BLS6\_6. 

Since $D=-3$ we know from exercise \ref{ex:Low_order_Hilbert_polys} that the associated Hilbert class polynomial is given by $H_{-3,43}(x) = x$, which implies that the $j$-invariant of $BLS6\_6$ is given by $j(BLS6\_6)=0$. We therefore have to look at the third case in table \ref{def:curve-order-frobenius} to deduce $a=0$ and derive parameter $b$. To apply \ref{def:curve-order-frobenius} and decide the proper equation for $j_0=0$ and $D=-3$, we have to choose some arbitrary quadratic non-residue $c_2$ and cubic non-residue $c_3$ in $\F_{43}$. We choose $c_2 =5$ and $c_3=36$. We check these with Sage:
\begin{sagecommandline}
sage: F43 = GF(43)
sage: c2 = F43(5)
....: try: # quadratic residue
....:     c2.nth_root(2)
....: except ValueError: # quadratic non-residue
....:     print("OK") 
sage: c3 =F43(36)
....: try:
....:     c3.nth_root(3)
....: except ValueError:
....:     print("OK") 
\end{sagecommandline} 

Using those numbers we check the six possible cases from \ref{def:curve-order-frobenius}\sme{check reference} against the expected order $39$ of the curve we want to synthesize:

\begin{sagecommandline}
sage: BLS61 = EllipticCurve(F43,[0,1])
sage: BLS61.order() == 39
sage: BLS62 = EllipticCurve(F43,[0,c2^3])
sage: BLS62.order() == 39
sage: BLS63 = EllipticCurve(F43,[0,c3^2])
sage: BLS63.order() == 39
sage: BLS64 = EllipticCurve(F43,[0,c3^2*c2^3])
sage: BLS64.order() == 39
sage: BLS65 = EllipticCurve(F43,[0,c3^(-2)])
sage: BLS65.order() == 39
sage: BLS66 = EllipticCurve(F43,[0,c3^(-2)*c2^3])
sage: BLS66.order() == 39
sage: BLS6 = BLS63 # our BLS6 curve in the book
\end{sagecommandline}
As expected, we found an elliptic curve of the correct order $39$ over a prime field of size $43$. Since $c_3^2=36^2=6$ in $\F_{43}$ the curve $BLS6\_6$ is defined by the following equation:

\begin{equation}
BLS6\_6 := \{(x,y)\;|\; y^2 = x^3 + 6 \text{ for all } x,y \in \F_{43}\}
\end{equation}

Since there are other choices for $c_3$ there are other choices for $b$, such as $b=10$ or $b=23$ too. However due to \ref{sec:isomorphic_curves} all these curves are isomorphic, and hence represent the same curve in different ways. We decided on $b=6$ for no particular reason.

Since BLS6\_6 contains $39$ points only, it is possible to use Sage in order to give a visual impression of the curve:

\begin{sagesilent}
BLS63p = BLS63.plot()
\end{sagesilent}
\begin{center} 
\sageplot[scale=.5]{BLS63p} 
\end{center}

As we can see, our curve has some desirable properties: it does not contain self-inverse points, that is, points with $y=0$. It follows that the addition law can be optimized, since the branch for those cases can be eliminated. 

\subsubsection{The large prime order subgroup}
Summarizing the previous procedure, we have used the method of Barreto, Lynn and Scott\sme{add reference} to construct a pairing-friendly elliptic curve of embedding degree $6$. However, in order to do elliptic curve cryptography on this curve, note that, since the order of $BLS6\_6$ is $39$, its group is not of prime order. We therefore have to find a suitable subgroup as our main target. Since $39=13\cdot 3$, we know that the curve must contain a ``large'' prime-order group of size $13$ called $BLS6\_6(\F_{43})[13]$ according to definition \ref{def:torsion_group} and a small cofactor group of order $3$. We use the following notation for the large prime order group
\begin{equation}
\label{def:BLS6613}
\G_1[13]:= BLS6\_6(\F_{43})[13]
\end{equation}

One way to compute this group is to find a generator. We can achieve this by choosing an arbitrary element of $BLS6\_6$ that is not the point at infinity, and then multiply that point with the cofactor $3$ of $13$. If the result is not the point at infinity, it will be a generator of $\G_1[13]$. If it is the point at infinity we have to choose a different element. 

In order construct such an element from $BLS6\_6$ in a pen-and-paper style, we can choose some $x\in\F_{43}$ and see if there is some $y\in\F_{43}$ that satisfies the defining \concept{short Weierstrass} equation $y^2 = x^3 + 6$. We choose $x=9$, and check that $y=2$ satisfies the curve equation for $x$:
\begin{align*}
y^2 & = x^3 + 6 & \Rightarrow \\
2^2 & = 9^3 + 6 & \Leftrightarrow \\
4 & = 4
\end{align*}   

This implies that $P=(9,2)$ is a point on $BLS6\_6$. To see if we can project this point onto a generator of the large prime order group $\G_1[13]$, we have to multiply $P$ by the cofactor $3$, that is, we have to compute $[3](9,2)$. We get $[3](9,2) = (13,15)$ (See exercise \ref{ex:BLS66-scalar-product}). Since this is not the point at infinity, we know that $(13,15)$ is a generator of $\G_1[13]$, which we will use throughout this book: 

\begin{equation}\label{gBLS6-6-13}
g_{1} = (13,15)
\end{equation}

Since $g_{1}$ is a generator, recall from \ref{exponentialmap} that there is an exponential map from the field $\F_{13}$ to $\G_1[13]$ with respect to this generator, which generates the group in logarithmic order:
$$
[\cdot]_{(13,15)}: \F_{13} \to \G_1[13]\;;\; x \mapsto [x](13,15)
$$ 
We can use this function to construct the subgroup $\G_1[13]$ by repeatedly adding the generator to itself. Using Sage, we get the following:
\begin{sagecommandline}
sage: P1 = BLS6(9,2)
sage: g1 = 3*P1 # generator
sage: g1.xy()
sage: G1_13 = [ x*g1 for x in range(0,13) ]
\end{sagecommandline}
Repeatedly adding a generator to itself generates small groups in logarithmic order with respect to the generator as explained in \ref{def:logarithmic_ordering}. This gives the following representation:
\begin{multline}
\label{BLS6-G1-log}
\G_1[13]=
\{(13,15) \rightarrow (33,34) \rightarrow  (38,15) \rightarrow  (35,28) \rightarrow (26,34) \rightarrow  (27,34) \rightarrow  \\ 
(27,9)  \rightarrow  (26,9) \rightarrow  (35,15) \rightarrow  (38,28) \rightarrow  (33,9) \rightarrow (13,28) \rightarrow  \mathcal{O}\}$$
\end{multline}
Having a logarithmic order of this group is helpful in pen-and-paper computations. To see that consider the following example:
\begin{align*}
(27,34)\oplus (33,9)  & = [6](13,15)\oplus [11](13,15)\\
                      & = [6+11](13,15)\\
                      & = [4](13,15)\\
                      & = (35,28)\\
\end{align*}
As this computation shows \ref{BLS6-G1-log} is really all we need to do computations in $\G_1[13]$ efficiently. $\G_1[13]$ is therefore suitable as a pen-and-paper cryptographic group. However, out of convenience, the following picture lists the entire addition table of the group $BLS6\_6$, as it might be useful in some pen-and-paper computations, that are not restricted to the subgroup $\G_1$:
\begingroup
    \fontsize{7pt}{7pt}\selectfont
$$
\begin{array}{c|ccccccccccccc}
\oplus & \mathcal{O}  & (13,15) & (33,34) & (38,15) & (35,28) & (26,34) & (27,34) & (27,9) & (26,9) & (35,15) & (38,28) & (33,9) & (13,28)\\
\hline
\\
\mathcal{O} & \mathcal{O}  & (13,15) & (33,34) & (38,15) & (35,28) & (26,34) & (27,34) & (27,9) & (26,9) & (35,15) & (38,28) & (33,9) & (13,28)\\
\\
(13,15) & (13,15) & (33,34) & (38,15) & (35,28) & (26,34) & (27,34) & (27,9) & (26,9) & (35,15) & (38,28) & (33,9) & (13,28) & \mathcal{O}\\
\\
(33,34) & (33,34) & (38,15) & (35,28) & (26,34) & (27,34) & (27,9) & (26,9) & (35,15) & (38,28) & (33,9) & (13,28) & \mathcal{O} & (13,15)\\
\\
(38,15) & (38,15) & (35,28) & (26,34) & (27,34) & (27,9) & (26,9) & (35,15) & (38,28) & (33,9) & (13,28) & \mathcal{O} & (13,15) & (33,34)\\
\\
(35,28) & (35,28) & (26,34) & (27,34) & (27,9) & (26,9) & (35,15) & (38,28) & (33,9) & (13,28) & \mathcal{O} & (13,15) & (33,34) & (38,15)\\
\\
(26,34) & (26,34) & (27,34) & (27,9) & (26,9) & (35,15) & (38,28) & (33,9) & (13,28) & \mathcal{O} & (13,15) & (33,34) & (38,15) & (35,28)\\
\\
(27,34) & (27,34) & (27,9) & (26,9) & (35,15) & (38,28) & (33,9) & (13,28) & \mathcal{O} & (13,15) & (33,34) & (38,15) & (35,28) & (26,34)\\
\\
(27,9) & (27,9) & (26,9) & (35,15) & (38,28) & (33,9) & (13,28) & \mathcal{O} & (13,15) & (33,34) & (38,15) & (35,28) & (26,34) & (27,34)\\
\\
(26,9) & (26,9) & (35,15) & (38,28) & (33,9) & (13,28) & \mathcal{O} & (13,15) & (33,34) & (38,15) & (35,28) & (26,34) & (27,34) & (27,9)\\
\\
(35,15) & (35,15) & (38,28) & (33,9) & (13,28) & \mathcal{O} & (13,15) & (33,34) & (38,15) & (35,28) & (26,34) & (27,34) & (27,9) & (26,9)\\
\\
(38,28) & (38,28) & (33,9) & (13,28) & \mathcal{O} & (13,15) & (33,34) & (38,15) & (35,28) & (26,34) & (27,34) & (27,9) & (26,9) & (35,15)\\
\\
(33,9) & (33,9) & (13,28) & \mathcal{O} & (13,15) & (33,34) & (38,15) & (35,28) & (26,34) & (27,34) & (27,9) & (26,9) & (35,15) & (38,28)\\
\\
(13,28) & (13,28) & \mathcal{O} & (13,15) & (33,34) & (38,15) & (35,28) & (26,34) & (27,34) & (27,9) & (26,9) & (35,15) & (38,28) & (33,9)\\
\end{array}
$$
\endgroup
\begin{exercise}
\label{ex:BLS66-scalar-product}
Consider the point $P=(9,2)$. Show that $P$ is a point on the $BLS6\_6$ curve and compute the scalar product $[3]P$. 
\end{exercise}
\begin{exercise}
Compute the following expressions: $-(26,34)$, $(26,9)\oplus(13,28)$, $(35,15)\oplus \mathcal{O}$ and $(27,9)\ominus(33,9)$.
\end{exercise}
%To see how the small prime-order group of $BLS6\_6$ looks like we can apply the same approach but for the cofactor $13$ instead (EXERCISE). We get 
%$$BLS6\_6[3]=\{\mathcal{O},(0,7),(0,36)\}$$
%Now that we ha
\subsubsection{Pairing groups}
\label{sec:bls66-pairing-groups}
We know that $BLS6\_6$ is a pairing-friendly curve by design, since it has a small embedding degree $k=6$. It is therefore possible to compute Weil pairings efficiently. However, in order to do so, we have to decide the pairing groups $\G_1$ and $\G_2$ as defined in section \ref{sec:pairing_groups}. 

Since $BLS6\_6$ has two non-trivial subgroups, it would be possible to use any of them as the $n$-torsion group. However, in cryptography, the only secure choice is to use the large prime-order subgroup, which in our case is $\G_1[13]$ as presented in \ref{BLS6-G1-log}. We therefore decide to consider the $13$-torsion for the Weil pairing and therefore use $\G_1[13]$ as its left argument.

In order to construct the domain for the right argument, we need to construct $\G_2[13]$, which, according to the general theory \ref{sec:pairing_groups}, should be defined by those elements $P$ of the full $13$-torsion group  $BLS6\_6[13]$ that are mapped to $43\cdot P$ under the Frobenius endomorphism \ref{eq:frobenius-enomorphism}. 

To compute $\G_2[13]$, we therefore have to find the full $13$-torsion group first. To do so, we use the technique from \ref{sec:full-torsion}, which tells us that the full $13$-torsion can be found in the curve extension $BLS6\_6(\F_{43^6})$ \ref{sec:curve-extensions} of $BLS6\_6$ over the extension field $\F_{43^6}$ \ref{field-extension}, since the embedding degree of $BLS6\_6$ is $6$. We therefore have to consider the following elliptic curve:

\begin{equation}
\label{def:extended-bls6}
BLS6\_6(\F_{43^6}) := \{(x,y)\;|\; y^2 = x^3 + 6 \text{ for all } x,y \in \F_{43^6}\}
\end{equation}

In order to compute this curve, we have to construct $\F_{43^6}$, a field that contains more then 6 billion elements. We use the general construction of prime field extensions from \ref{field-extension} and start by choosing a non-reducible polynomial of degree $6$ from the ring of polynomials $\F_{43}[t]$. We choose $p(t) = t^6+6$. In order to visually distinguish polynomials with coefficients in $\F_{43}$ from elements in $\F_{43^6}$, we use the symbol $v$ exclusively to represent the indeterminate in $\F_{43^6}$. Using Sage, we get the following:
\begin{sagecommandline}
sage: F43 = GF(43)
sage: F43t.<t> = F43[]
sage: p = F43t(t^6+6)
sage: p.is_irreducible()
sage: F43_6.<v> = GF(43^6, name='v', modulus=p)
sage: F43_6.order()
\end{sagecommandline}

Recall from \ref{eq:prime-extension-field} that elements $x\in\F_{43^6}$ can be seen as polynomials $a_0+a_1v + a_2v^2+\ldots + a_5 v^5$ with the usual addition of polynomials and multiplication modulo $v^6+6$. 

In order to compute $\G_2[13]$, we first have to extend $BLS6\_6$ to $\F_{43^6}$, that is, we keep the defining equation, but expand the domain from $\F_{43}$ to $\F_{43^6}$. After that, we have to find at least one element $P$ from that curve that is not the point at infinity, is in the full $13$-torsion and  satisfies the identity $\pi(P) = [43]P$, where $\pi$ is the Frobenius endomorphism \ref{eq:frobenius-enomorphism}. We can then use this element as our generator of $\G_2[13]$ and construct all other elements by repeatedly adding the generator to itself.

Since $BLS6(\F_{43^6})$ contains approximately as many elements as $\F_{43^6}$ by the Hasse bound \ref{hasse-bound}, it's not a good strategy to simply loop through all elements. Fortunately, Sage has a way to loop through elements from the torsion group directly:

\begin{sagecommandline}
sage: ExtBLS6 = EllipticCurve(F43_6,[0 ,6]) # curve extension
sage: INF = ExtBLS6(0) # point at infinity
sage: for P in INF.division_points(13): # full 13-torsion
....:     # pI(P) == [q]P
....:     if P.order() == 13: # exclude point at infinity
....:         piP = ExtBLS6([a.frobenius() for a in P])
....:         qP = 43*P
....:         if piP == qP:
....:             break
sage: P.xy()
\end{sagecommandline}

We found an element $P$ from the full $13$-torsion with the property $\pi(P) = [43]P$, which implies that it is an element of $\G_2[13]$. As $\G_2[13]$ is cyclic and of prime order, this element must be a generator:
\begin{equation}
g_{2} = (7v^2, 16v^3)
\end{equation}

We can use this generator to compute $\G_2[13]$ in logarithmic order with respect to $g_{2}$. Using Sage we get the following:
\begin{sagecommandline}
sage: g2 = ExtBLS6(7*v^2,16*v^3)
sage: G2_13 = [ x*g2 for x in range(0,13) ]
\end{sagecommandline}
Repeatedly adding a generator to itself generates small groups in logarithmic order with respect to the generator as explained in \ref{def:logarithmic_ordering}. We therefore get the following presentation:
\begin{multline}
\label{BLS6-G2-log}
\mathbb{G}_2=\{
(7v^2, 16v^3) \to
(10v^2, 28v^3)\to
(42v^2, 16v^3)\to
(37v^2, 27v^3)\to\\
(16v^2, 28v^3)\to
(17v^2, 28v^3)\to
(17v^2, 15v^3)\to
(16v^2, 15v^3)\to\\
(37v^2, 16v^3)\to
(42v^2, 27v^3)\to
(10v^2, 15v^3)\to
(7v^2, 27v^3)\to
\mathcal{O}\}
\end{multline}

Again, having a logarithmic description of $\G_2[13]$ is helpful in pen-and-paper computations, as it reduces complicated computation in the extended curve to modular $13$ arithmetic, as in the following example:
\begin{align*}
(17v^2,28v^3)\oplus (10v^2,15v^2)  & = [6](7v^2,16v^3)\oplus [11](7v^2,16v^3)\\
                      & = [6+11](7v^2,16v^3)\\
                      & = [4](7v^2,16v^3)\\
                      & = (37v^2,27v^3)\\
\end{align*}
As this computation shows \ref{BLS6-G2-log} is really all we need to do computations in $\G_2[13]$ efficiently. The groups $\G_1[13]$ and $\G_2[13]$ are therefore suitable as pen-and-paper cryptographic pairing groups. 

\subsubsection{The Weil pairing}
In \ref{sec:bls66-pairing-groups} we computed two different groups, $\G_1[13]$ and $\G_2[13]$, which are subgroups of the full $13$-torsion group of the extended moon-math-curve $BLS6\_6(\F_{43^6})$. As explained in \ref{sec:weil-pairing}, this implies that there is a Weil pairing 
\begin{equation}
e(\cdot,\cdot): \G_1[13]\times\G_2[13] \to \F_{43^6}
\end{equation}

Since we know a logarithmic order \ref{BLS6-G1-log} for $\G_1[13]$  and a logarithmic order \ref{BLS6-G2-log} for $\G_2[13]$, this Weil pairing is efficiently computable in pen-and-paper calculations, using the following identity as derived in exercise \ref{ex:pairing-arithmetics}:  
\begin{equation}\label{eq:bilinearityBLS6}
e([m]g_{1},[n]g_{2}) = 
e(g_{1},g_{2})^{\Zmod{m\cdot n}{13}}
\end{equation}
In many pairing based zero knowledge proving systems like \ref{sec:gorth_16}, it is only necessary to show equality of various pairings and in pen-and-paper computations the exact value of $e(g_{1},g_{2})$ is therefore not important. To see how this simplifies the calculation, assume that we want to proof the identity $e((27,34),(16v^2,28v^3))=e((26,9),(17v^2,15v^3))$. In this case we can compute
\begin{align*}
e((27,34),(16v^2,28v^3)) 
               & = e([6](13,15),[5](7v^2,16v^3))\\
               & = e((13,15),(7v^2,16v^3))^{6\cdot 5}\\
               & = e((13,15),(7v^2,16v^3))^{4}\\
               & = e((13,15),(7v^2,16v^3))^{8\cdot 7}\\
               & = e([8](13,15),[7](7v^2,16v^3))\\
               & = e((26,9),(17v^2,15v^3))\\
\end{align*}
If the actual value of $e(g_{1},g_{2})$ is needed, the Weil pairing can either  be computed using equation \ref{eq:weil-pairing} and execute Miller's algorithm \ref{alg:millersalgo} (Exercise \ref{ex:generator-pairing}), or Sage can be invoked:  
\begin{sagecommandline}
sage: g1 = ExtBLS6([13,15])
sage: g2 = ExtBLS6([7*v^2, 16*v^3])
sage: g1.weil_pairing(g2,13)
\end{sagecommandline}

\begin{exercise}
\label{ex:generator-pairing}
Consider the extended $BLS6\_6$ curve as defined in \ref{def:extended-bls6} and the two curve points $g1=(13,15)$ and $g2=(7v^2,16v^3)$. Compute the Weil pairing 
$e(g1,g2)$ using definition \ref{eq:weil-pairing} and Miller's algorithm \ref{alg:millersalgo}.
\end{exercise}
\begin{comment}
\subsection{Hashing to pairing groups}
We give various constructions to hash into $\mathbb{G}_1$ and $\mathbb{G}_2$. 

We start with hashing to the scalar field... \smelong{TO APPEAR}\sme{finish writing this up}

None of these techniques work for hashing into $\mathbb{G}_2$. We therefore implement Pederson's Hash for BLS6. 

We start with $\mathbb{G}_1$. Our goal is to define an $12$-bit bounded hash function:
$$
H_{1}: \{0,1\}^{12} \to \mathbb{G}_1 
$$
Since $12= 3\cdot 4$ we ``randomly'' select $4$ uniformly distributed generators $\{(38, 15), (35,28),\\ (27, 34), (38, 28)\}$ from $\mathbb{G}_1$ and use the pseudo-random function from XXX\sme{add reference}. 
Therefore, we have to choose a set of $4$ randomly generated invertible elements from $\F_{13}$ for every generator. We choose the following:
$$
\begin{array}{lcl}
(38,15) &:& \{2,7,5,9\}\\
(35,28) &:& \{11,4,7,7\}\\
(27,34) &:& \{5,3,7,12\}\\
(38,28) &:& \{6,5,1,8\}
\end{array}
$$
Our hash function is then computed as follows:

\begin{multline*}
H_1(x_{11},x_1,\ldots, x_{0})=
[2\cdot 7^{x_{11}}\cdot 5^{x_{10}}\cdot 9^{x_9}](38,15)+
[11\cdot 4^{x_8}\cdot 7^{x_7}\cdot 7^{x_6}](35,28)+\\
[5\cdot 3^{x_5}\cdot 7^{x_4}\cdot 12^{x_3}](27,34) +
[6\cdot 5^{x_2}\cdot 1^{x_{1}}\cdot 8^{x_{0}}](38,28)
\end{multline*}

Note that $a^x=1$ when $x=0$. Hence, those terms can be omitted in the computation. 
In particular, the hash of the $12$-bit zero string is given as follows:\sme{correct computations}

\begin{multline*}\smelong{WRONG-ORDERING-REDO}\\
H_1(0)= [2](38,15)+[11](35,28)+[5](27,34)+[6](38,28)= \\
(27,34)+(26,34)+(35,28)+(26,9)= (33,9) + (13,28) = (38,28)
\end{multline*}

The hash of $011010101100$ is given as follows:\sme{fill in missing parts}
\begin{multline*}
H_1(011010101100)=\smelong{WRONG-ORDERING-REDO}\\
[2\cdot 7^{0}\cdot 5^{1}\cdot 9^{1}](38,15)+
[11\cdot 4^{0}\cdot 7^{1}\cdot 7^{0}](35,28)+
[5\cdot 3^{1}\cdot 7^{0}\cdot 12^{1}](27,34) +
[6\cdot 5^{1}\cdot 1^{0}\cdot 8^{0}](38,28)=\\
[2\cdot 5\cdot 9](38,15)+
[11\cdot 7](35,28)+
[5\cdot 3\cdot 12](27,34) +
[6\cdot 5](38,28)=\\
[12](38,15)+
[12](35,28)+
[11](27,34) +
[4](38,28)=\\ 
\smelong{TO APPEAR}
\end{multline*}
We can use the same technique to define a $12$-bit bounded hash function in $\mathbb{G}_2$:  
$$
H_{2}: \{0,1\}^{12} \to \mathbb{G}_2 
$$
Again, we ``randomly'' select $4$ uniformly distributed generators $\{(7v^2 , 16v^3 ), (42v^2 , 16v^3 ), \\(17v^2 , 15v^3 ), (10v^2 , 15v^3 )\}$ from $\mathbb{G}_2$, and use the pseudo-random function from XXX\sme{add reference}. Therefore, we have to choose a set of $4$ randomly generated invertible elements from $\F_{13}$ for every generator:
$$
\begin{array}{lcl}
(7v^2 , 16v^3 ) &:& \{8,4,5,7\}\\
(42v^2 , 16v^3 ) &:& \{12,1,3,8\}\\
(17v^2 , 15v^3 ) &:& \{2,3,9,11\}\\
(10v^2 , 15v^3 ) &:& \{3,6,9,10\}
\end{array}
$$
Our hash function is then computed like this:
\begin{multline*}
H_1(x_{11},x_{10},\ldots, x_{0})=
[8\cdot 4^{x_{11}}\cdot 5^{x_{10}}\cdot 7^{x_9}](7v^2 , 16v^3)+
[12\cdot 1^{x_8}\cdot 3^{x_7}\cdot 8^{x_6}](42v^2 , 16v^3 )+\\
[2\cdot 3^{x_5}\cdot 9^{x_4}\cdot 11^{x_3}](17v^2 , 15v^3 ) +
[3\cdot 6^{x_2}\cdot 9^{x_{1}}\cdot 10^{x_{0}}](10v^2 , 15v^3 )
\end{multline*}

We extend this to a hash function that maps unbounded bitstrings to $\mathbb{G}_2$ by precomposing with an actual hash function like $MD5$, and feed the first 12 bits of its outcome into our previously defined hash function, with $TinyMD5_{\mathbb{G}_2}(s)= H_2(MD5(s)_0,\ldots MD5(s)_{11})$:
$$
TinyMD5_{\mathbb{G}_2}: \{0,1\}^* \to \mathbb{G}_2
$$
For example, since 
$MD5("")=\\ 0xd41d8cd98f00b204e9800998ecf8427e$, and the binary representation of the hexadecimal number $0x27e$ is $001001111110$, we compute $TinyMD5_{\mathbb{G}_2}$ of the empty string as follows:
$$TinyMD5_{\mathbb{G}_2}("")= H_2(MD5(s)_{11},\ldots MD5(s)_{0}) = H_2(001001111110)=$$\sme{check equation}
\end{comment}




% FOR THE SECOND VERSION OF THE BOOK
%\subsection{MNT4 MNT6 Cycles}
% https://eprint.iacr.org/2006/372.pdf theorem 5.2
%\begin{theorem}
%Let $q$ be a prime and $E/\F_q$ be an ordinary elliptic curve such that $r= |E(Fq)|$ is a prime greater than $3$.  
%\begin{itemize}
%\item $E$ has embedding  degree $k= 4$ if and only if there  exists $x\in \mathbb{Z}$ such  that $t=-x$ or $t=x+1$, and $q=x^2+x+1$.\item $E$ has  embedding  degree $k= 6$ if and only if there  exists $x\in \Z$ such that $t= 1\pm 2x$ and $q=4x^2+1$.
%\item There is an elliptic curve $E/\F_q$ with embedding degree $6$, discriminant $D$, and $|E(Fq)| = r$ if and only if there is an elliptic curve $E'/\F_r$ with embedding degree $4$, discriminant $D$, and $|E'(\F_r)| =q$.
%\end{itemize}
%\end{theorem}

%We can use this theorem to find an MNT6-MNT4 cycle over very small prime fields with characteristics $>3$: 
%\subsubsection{MNT4}
%For our MNT4 curve, we can choose $x=2$. Then $q=7$ and if we choose $t= x+1 $ then $r = q + 1 - t = 7 + 1 -3 = 5$. Therefore our MNT4 curve is a curve $y^2=x^3+ax+b$ defined over $\F_7$ that consists of $5$ points. 

%To construct the actual curve we could use the \concept{complex multiplication method} again, but since the parameters $a$ and $b$ are from $\F_7$ there are only $48$ possibilities so we simply loop through all possible $a$'s and $b$'s and count the curve points until we find a curve that has $5$ rational points. We get
%$$
%y^2 = x^3 + 4x + 1
%$$
%defined over $\F_7$, with scalar field $\F_5$. Since $7= 2^2+2+1$, we know from theorem XXX that this curve has embedding degree $4$ and hence qualifies as a pen-and-paper pairing-friendly elliptic curve. Since the curve's order is a prime and therefore has no non trivial factors, it has no non trivial subgroups. The curve has the following set of elements
%$$MNT4=\{(0,1)\to (0,6)\to (4,2)\to (4,5) \to \mathcal{O}\}$$ 
%\begin{sagecommandline}
%sage: F7 = GF(7)
%sage: MNT4 = EllipticCurve (F7,[4 ,1])
%sage: [P.xy() for P in MNT4.points() if P.order() > 1]
%\end{sagecommandline}
%The multiplication table is
%\begingroup
%    \fontsize{10pt}{10pt}\selectfont
%$$
%\begin{array}{c|ccccc}
%\cdot & \mathcal{O} & (0,1) & (4,5) & (4,2) & (0,6)\\
%\hline
%\\
%\mathcal{O} & \mathcal{O} & (0,1) & (4,5) & (4,2) & (0,6)\\
%\\
%(0,1) & (0,1) & (4,5) & (4,2) & (0,6) & \mathcal{O}\\
%\\
%(4,5) & (4,5) & (4,2) & (0,6) & \mathcal{O} & (0,1)\\
%\\
%(4,2) & (4,2) & (0,6) & \mathcal{O} & (0,1) & (4,5)\\
%\\
%(0,6) & (0,6) & \mathcal{O} & (0,1) & (4,5) & (4,2)\\
%\end{array}
%$$
%\endgroup
%In what follows we choose our generator to be $g_{MNT4}=(0,1)$.

%In what follows we want to compute type 2 pairings on our MNT4 curve. We therefore need to extract the subgroup $\mathbb{G}_1$ as well as $\mathbb{G}_2$ from the full $5$-torsion group. Since the order of MNT4 is a prime number, we already know from XXX\sme{add reference} that $\mathbb{G}_1$ is given by  $$\mathbb{G}_1=\{(0,1)\to (0,6)\to (4,2)\to (4,5) \to \mathcal{O}\}$$ 

%In type 2 pairings, the group $\mathbb{G}_2$ is defined by those elements $P$ of the full $5$-torsion group that are mapped to $7\cdot P$ under the Frobenius endomorphism XXX\sme{add reference}. Since $MNT4/\F_{7^4}$ only contains $2475$ elements, we can  loop through all elements, to find the full $5$-torsion group and extract all elements from $\mathbb{G}_2$:
%\begin{sagecommandline}
%sage: F7t.<t> = F7[]
%sage: F7_4.<u> = GF(7^4, name='u', modulus=t^4+t+1) # embedding degree is 4
%sage: MNT4 = EllipticCurve (F7_4,[4 ,1])
%sage: INF = MNT4(0) # point at infinity
%sage: for P in INF.division_points(5): # PI(P) == [q]P
%....:     if P.order() == 5: # exclude point at infinity
%....:         PiP = MNT4([a.frobenius() for a in P])
%....:         qP = 7*P
%....:         if PiP == qP:
%....:             print(P.xy())
%\end{sagecommandline}

%Choose $g_2=(2u^3 + 5u^2 + 4u + 2, 2u^3 + 3u + 5)$ as generator of $\mathbb{G}_2$, we get
%\begin{multline*}
%\mathbb{G}_2=\{ 
%(2u^3 + 5u^2 + 4u + 2, 2u^3 + 3u + 5) \to
%(5u^3 + 2u^2 + 3u + 6, 2u^2 + 3u) \to \\
%(5u^3 + 2u^2 + 3u + 6, 5u^2 + 4u) \to
%(2u^3 + 5u^2 + 4u + 2, 5u^3 + 4u + 2)\to
%\mathcal{O}\}
%\end{multline*}
%e.g. $[3]g_2= (5u^3 + 2u^2 + 3u + 6, 5u^2 + 4u)$.

%Having those groups we can do pairings. We choose the Weil pairing and use Sagemath. For example the Weil pairing between our two generators is
%$$
%e(g_1,g_2)= 5u^3 + 2u^2 + 6u
%$$
%\begin{sagecommandline}
%sage: g1 = MNT4([0,1])
%sage: g2 = MNT4(2*u^3 + 5*u^2 + 4*u + 2, 2*u^3 + 3*u + 5)
%sage: g1.weil_pairing(g2,5)
%\end{sagecommandline}
%The full pairing table can the be written as
%\begingroup
%    \fontsize{10pt}{10pt}\selectfont
    
% generate the table entries as:
% sage: for i in range(5):
% ....:     for j in range(5):
% ....:         p = (i*g1).weil_pairing((j*g2),5)
% ....:         print('e([',i,']g1,[',j,']g2=',p)         
    
    
%$$
%\begin{array}{c|lllll}
%e(\cdot,\cdot)    & \mathcal{O} & g_1            & [2]g_1         & [3]g_1         %& [4]g_1\\
%\hline
%\\
%      \mathcal{O} & 1           & 1              & 1              & 1              %& 1\\
%\\
%              g_2 & 1           & 5u^3+2u^2+6u   & 6u^3+5u^2+6    & 2u^3+u^2+2u+3  %& u^3+6u^2+6u+4\\
%\\
%\left[2\right]g_2 & 1           & 6u^3+5u^2+6    & u^3+6u^2+6u+4  & 5u^3+2u^2+6u   %& 2u^3+u^2+2u+3\\
%\\
%\left[3\right]g_2 & 1           & 2u^3+u^2+2u+3  & 5u^3+2u^2+6u   & u^3+6u^2+6u+4  & 6u^3+5u^2+6\\
%\\
%\left[4\right]g_2 & 1           & u^3+6u^2+6u+4  & 2u^3+u^2+2u+3  & 6u^3+5u^2+6    & 5u^3+2u^2+6u\\
%\end{array}
%$$
%\endgroup

%\subsubsection{MNT6}
%For our MNT6 curve, we can choose $x=1$. Then $q=5$ and if we choose $t= 1 + 2x $ then $r= q + 1 - t = 5 + 1 + 1 = 7$. Therefore our MNT6 curve is a curve $y^2=x^3+ax+b$ defined over $\F_5$ that consists of $7$ points. 

%To construct the actual curve we could use the \concept{complex multiplication method} again, but since the parameters $a$ and $b$ are from $\F_5$ there are only $24$ possibilities, we simply loop through all possible $a$'s and $b$'s and count the curve points until we find a curve that has $7$ rational points. We get
%$$
%y^2 = x^3 + 2x + 1
%$$
%defined over $\F_5$. Since $5= 4\cdot 1 + 1$, we know from theorem XXX that this curve has embedding degree $6$ and hence qualifies as a pen-and-paper pairing-friendly elliptic curve. 

%The curve has the following set of elements
%$$MNT6=\{(1,2)\to (3,3)\to (0,1)\to (0,4)\to (3,2)\to (1,3)\to \mathcal{O}\}$$
%The multiplication table is
%\begingroup
%    \fontsize{10pt}{10pt}\selectfont
%$$
%\begin{array}{c|ccccccc}
%\cdot & \mathcal{O} & (1,2) & (3,3) & (0,1) & (0,4) & (3,2) & (1,3)\\
%\hline
%\\
%\mathcal{O} & \mathcal{O} & (1,2) & (3,3) & (0,1) & (0,4) & (3,2) & (1,3)\\
%\\
%(1,2) & (1,2) & (3,3) & (0,1) & (0,4) & (3,2) & (1,3) & \mathcal{O}\\
%\\
%(3,3) & (3,3) & (0,1) & (0,4) & (3,2) & (1,3) & \mathcal{O} & (1,2)\\
%\\
%(0,1) & (0,1) & (0,4) & (3,2) & (1,3) & \mathcal{O} & (1,2) & (3,3)\\
%\\
%(0,4) & (0,4) & (3,2) & (1,3) & \mathcal{O} & (1,2) & (3,3) & (0,1)\\
%\\
%(3,2) & (3,2) & (1,3) & \mathcal{O} & (1,2) & (3,3) & (0,1) & (0,4)\\
%\\
%(1,3) & (1,3) & \mathcal{O} & (1,2) & (3,3) & (0,1) & (0,4) & (3,2)\\
%\end{array}
%$$
%\endgroup

%In what follows we choose our generator to be $g_{MNT6}=(1,2)$.

%In what follows we want to compute type 2 pairings on our MNT6 curve. We therefore need to extract the subgroup $\mathbb{G}_1$ as well as $\mathbb{G}_2$ from the full $7$-torsion group. Since the order of MNT6 is a prime number, we already know from XXX that $\mathbb{G}_1$ is given by
%$$\mathbb{G}_1=\{(1,2)\to (3,3)\to (0,1)\to (0,4)\to (3,2)\to (1,3)\to \mathcal{O}\}$$
%In type 2 pairings, the group $\mathbb{G}_2$ is defined by those elements $P$ of the full $7$-torsion group that are mapped to $5\cdot P$ under the Frobenius endomorphism XXX. Since $MNT6/\F_{5^6}$ contains $15680$ elements, we can still loop through all elements, to find the full $7$-torsion group and extract all elements from $\mathbb{G}_2$

%\begin{sagecommandline}
%sage: G.<x> = GF(5^6) # embedding degree is 6
%sage: MNT6 = EllipticCurve (G,[2 ,1])
%sage: INF = MNT6(0) # point at infinity
%sage: for P in INF.division_points(7): # PI(P) == [q]P
%....:     if P.order() == 7: # exclude point at infinity
%....:         PiP = MNT6([a.frobenius() for a in P])
%....:         qP = 5*P
%....:         if PiP == qP:
%....:             print(P.xy())
%\end{sagecommandline}

%\begin{multline*}
%\mathbb{G}_2=\{ 
%(x^3+2x^2+4x,x^5+2x^4+4x^3+3x^2+3)\to
%(x^5+4x^4+2x^3+3x^2+x+2,3x^4+2x^3+x)\to\\
%(4x^5+x^4+2x^3,3x^5+x^4+x^3+4x+4)\to
%(4x^5+x^4+2x^3,2x^5+4x^4+4x^3+x+1) \to\\
%(x^5+4x^4+2x^3+3x^2+x+2,2x^4+3x^3+4x)\to
%(x^3+2x^2+4x,4x^5+3x^4+x^3+2x^2+2)\to
%\mathcal{O}\}
%\end{multline*}
%We choose the generator $g_2 = (x^3+2x^2+4x,x^5+2x^4+4x^3+3x^2+3)$

%\begin{remark}
%Note however that our MNT6 curve discriminant $D=-16(4a^3 + 27 b^2)= -16(4\cdot 2^3 + 27\cdot 1^2)=-944$, while our MNT4 curve has discriminate XXX. Hence our example curves are not those guaranteed by theorem XXX. Those curve are both given by $y^2= x^3 + 2x +1$ over $\F_5$ and $\F_7$, respectively. However as both curves have the same defining equation, we rather choose examples that are visually distinguishable by their defining equations.




%\end{remark}


