\documentclass{article}
\usepackage[utf8]{inputenc}
\usepackage{amsfonts}

\title{Moonmath manual}
\author{}
\date{March 2021}

\begin{document}

\maketitle

Lorem \textbf{ipsum} dolor sit amet, consectetur adipiscing elit. Pellentesque semper viverra dictum.  Fusce interdum venenatis leo varius vehicula. Etiam ac massa dolor. Quisque vel massa faucibus, facilisis nulla nec, egestas lectus. Sed orci dui, egestas non felis vel, fringilla pretium odio. \textit{Aliquam} vel consectetur felis. Suspendisse justo massa, maximus eget nisi a, maximus gravida mi.

Hallo
Hi Sylvia!
\section{Introduction}

Nisi vitae suscipit tellus mauris a diam maecenas sed. Amet nisl purus in mollis nunc sed id semper. Et odio pellentesque diam volutpat commodo sed egestas egestas fringilla. Ultricies mi eget mauris pharetra et. Dictum non consectetur a erat nam at lectus urna. Elementum curabitur vitae nunc sed velit. Tincidunt nunc pulvinar sapien et ligula. Turpis cursus in hac habitasse. Hac habitasse platea dictumst quisque sagittis purus. Nam libero justo laoreet sit. Diam ut venenatis tellus in metus vulputate. Lacinia at quis risus sed vulputate. Porta nibh venenatis cras sed felis eget.
\section{Ecosystem}

Testing

Tortor vitae purus faucibus ornare suspendisse sed nisi. Tellus in hac habitasse platea dictumst vestibulum rhoncus est. Commodo elit at imperdiet dui accumsan. Leo a diam sollicitudin tempor id eu nisl. Fermentum et sollicitudin ac orci. Elementum nibh tellus molestie nunc non blandit massa. Sed lectus vestibulum mattis ullamcorper velit sed. Aliquet enim tortor at auctor. Tincidunt arcu non sodales neque sodales ut. Lectus proin nibh nisl condimentum id venenatis. Vestibulum rhoncus est  elit ullamcorper dignissim. Quam adipiscing vitae proin sagittis nisl. Sit amet tellus cras adipiscing. Semper eget duis at tellus at urna condimentum mattis pellentesque. Est velit egestas dui id ornare. Facilisis volutpat est velit egestas dui id ornare arcu. Vitae justo eget magna fermentum iaculis eu non. Gravida neque convallis a cras semper.


\subsection{Polynome, formale Potenz- und Laurent-Reihen}

In dieser Arbeit bezeichnet $\mathbb{Z}$ stets den Ring der ganzen Zahlen, $\mathbb{N}$ die Menge der \textbf{positiven} ganzen Zahlen, 
$\mathbb{N}_0$ die Menge der nicht negativen ganzen Zahlen, 
$\mathbb{R}$ die Menge der reellen Zahlen und $\mathbb{R}^+_0$ die Menge
der nicht negativen reellen Zahlen.

Ein \textit{Integritätsring} bzw. \textit{Integritätsbereich}, ist ein vom Nullring verschiedener, nullteilerfreier kommutativer Ring mit Einse

blabl

\end{document}
